\documentclass[12pt]{book}
\usepackage[usenames]{color}
\def\aaps{{\it A\&AS}}
\def\aj{{\it AJ}}
\definecolor{DarkGreen}{rgb}{0,0.5,0} 
\definecolor{myBlue}{rgb}{0,0.0,0.75} 
\definecolor{myRed}{rgb}{0.33,0.0,0.0} 
\definecolor{myOrange}{rgb}{0.6,0.2,0.0} 
\def\ttg{\tt\color{DarkGreen}}
\def\ttb{\tt\color{myBlue}}
\def\tto{\tt\color{myOrange}}
\def\itr{\it\color{myRed}}

\evensidemargin 0.0in
\oddsidemargin 0.0in
\topmargin -0.2in
\textheight 9.0in
\textwidth 6.25in

\begin{document}
\begin{titlepage}
\begin{center}
\vspace{15em}
{\Huge{\bf StarFISH}} \\
\medskip
(\underline{F}or \underline{I}nferring 
\underline{S}tar-formation \underline{H}istories) \\
\vspace{5em}
{\Huge User Manual} \\
\bigskip
{\large version 1.1, May 2004} \\
\vspace{7em}
{\large Jason Harris and Dennis Zaritsky}\\
{\ttb jharris@as.arizona.edu} \\
{\ttb dzaritsky@as.arizona.edu} \\
\end{center}

\vfill
\begin{quote}
\hrule
Permission is granted to copy, distribute and/or modify this
document under the terms of the GNU Free Documentation License,
Version 1.1 or any later version published by the Free Software
Foundation; with no Invariant Sections, no Front-Cover Texts and
no Back-Cover Texts.
\end{quote}
\end{titlepage}

\tableofcontents

\documentclass[12pt]{book}
\usepackage[usenames]{color}
\def\aaps{{\it A\&AS}}
\def\aj{{\it AJ}}
\definecolor{DarkGreen}{rgb}{0,0.5,0} 
\definecolor{myBlue}{rgb}{0,0.0,0.75} 
\definecolor{myRed}{rgb}{0.33,0.0,0.0} 
\definecolor{myOrange}{rgb}{0.6,0.2,0.0} 
\def\ttg{\tt\color{DarkGreen}}
\def\ttb{\tt\color{myBlue}}
\def\tto{\tt\color{myOrange}}
\def\itr{\it\color{myRed}}

\evensidemargin 0.0in
\oddsidemargin 0.0in
\topmargin -0.2in
\textheight 9.0in
\textwidth 6.25in

\begin{document}
\begin{titlepage}
\begin{center}
\vspace{15em}
{\Huge{\bf StarFISH}} \\
\medskip
(\underline{F}or \underline{I}nferring 
\underline{S}tar-formation \underline{H}istories) \\
\vspace{5em}
{\Huge User Manual} \\
\bigskip
{\large version 1.1, May 2004} \\
\vspace{7em}
{\large Jason Harris and Dennis Zaritsky}\\
{\ttb jharris@as.arizona.edu} \\
{\ttb dzaritsky@as.arizona.edu} \\
\end{center}

\vfill
\begin{quote}
\hrule
Permission is granted to copy, distribute and/or modify this
document under the terms of the GNU Free Documentation License,
Version 1.1 or any later version published by the Free Software
Foundation; with no Invariant Sections, no Front-Cover Texts and
no Back-Cover Texts.
\end{quote}
\end{titlepage}

\tableofcontents

\documentclass[12pt]{book}
\usepackage[usenames]{color}
\def\aaps{{\it A\&AS}}
\def\aj{{\it AJ}}
\definecolor{DarkGreen}{rgb}{0,0.5,0} 
\definecolor{myBlue}{rgb}{0,0.0,0.75} 
\definecolor{myRed}{rgb}{0.33,0.0,0.0} 
\definecolor{myOrange}{rgb}{0.6,0.2,0.0} 
\def\ttg{\tt\color{DarkGreen}}
\def\ttb{\tt\color{myBlue}}
\def\tto{\tt\color{myOrange}}
\def\itr{\it\color{myRed}}

\evensidemargin 0.0in
\oddsidemargin 0.0in
\topmargin -0.2in
\textheight 9.0in
\textwidth 6.25in

\begin{document}
\begin{titlepage}
\begin{center}
\vspace{15em}
{\Huge{\bf StarFISH}} \\
\medskip
(\underline{F}or \underline{I}nferring 
\underline{S}tar-formation \underline{H}istories) \\
\vspace{5em}
{\Huge User Manual} \\
\bigskip
{\large version 1.1, May 2004} \\
\vspace{7em}
{\large Jason Harris and Dennis Zaritsky}\\
{\ttb jharris@as.arizona.edu} \\
{\ttb dzaritsky@as.arizona.edu} \\
\end{center}

\vfill
\begin{quote}
\hrule
Permission is granted to copy, distribute and/or modify this
document under the terms of the GNU Free Documentation License,
Version 1.1 or any later version published by the Free Software
Foundation; with no Invariant Sections, no Front-Cover Texts and
no Back-Cover Texts.
\end{quote}
\end{titlepage}

\tableofcontents

\documentclass[12pt]{book}
\usepackage[usenames]{color}
\def\aaps{{\it A\&AS}}
\def\aj{{\it AJ}}
\definecolor{DarkGreen}{rgb}{0,0.5,0} 
\definecolor{myBlue}{rgb}{0,0.0,0.75} 
\definecolor{myRed}{rgb}{0.33,0.0,0.0} 
\definecolor{myOrange}{rgb}{0.6,0.2,0.0} 
\def\ttg{\tt\color{DarkGreen}}
\def\ttb{\tt\color{myBlue}}
\def\tto{\tt\color{myOrange}}
\def\itr{\it\color{myRed}}

\evensidemargin 0.0in
\oddsidemargin 0.0in
\topmargin -0.2in
\textheight 9.0in
\textwidth 6.25in

\begin{document}
\begin{titlepage}
\begin{center}
\vspace{15em}
{\Huge{\bf StarFISH}} \\
\medskip
(\underline{F}or \underline{I}nferring 
\underline{S}tar-formation \underline{H}istories) \\
\vspace{5em}
{\Huge User Manual} \\
\bigskip
{\large version 1.1, May 2004} \\
\vspace{7em}
{\large Jason Harris and Dennis Zaritsky}\\
{\ttb jharris@as.arizona.edu} \\
{\ttb dzaritsky@as.arizona.edu} \\
\end{center}

\vfill
\begin{quote}
\hrule
Permission is granted to copy, distribute and/or modify this
document under the terms of the GNU Free Documentation License,
Version 1.1 or any later version published by the Free Software
Foundation; with no Invariant Sections, no Front-Cover Texts and
no Back-Cover Texts.
\end{quote}
\end{titlepage}

\tableofcontents

\include{sfh_manual.toc}

\chapter{Introduction}\label{ch:intro}

\section{Overview}

The {\bf StarFISH} package is a suite of FORTRAN programs designed to 
determine the best-fit star formation history (SFH) of a stellar 
population.  The package constructs a library of synthetic 
color-magnitude diagrams (CMDs) based on theoretical isochrones and 
data-derived determinations of the interstellar extinction, 
photometric errors, and distance modulus.  These synthetic CMDs are 
combined linearly and statistically compared to observed photometry.  
When the best-fitting model is found, the amplitude coefficients on 
each synthetic CMD describe the star formation history of the observed 
population.

\medskip
\noindent
If you have any comments, corrections, suggestions, or complaints,
please email: \\
{\ttb jharris@as.arizona.edu}

\bigskip
\noindent
To use the package, you need:

\begin{itemize}
\item {\bf Multicolor photometry} of a stellar population. \\

\item A (preferably empirical) {\bf photometric error model}, 
typically derived from artificial star tests. \\

\item A {\bf library of isochrones} that covers relevant ranges in age 
and metallicity. \\

\item {\it a priori} determination of the {\bf interstellar 
extinction} and {\bf distance modulus}. \\

\item Estimates (or guesses) of the {\bf binary fraction} and {\bf IMF 
slope}. \\
\end{itemize}

\clearpage

The package consists of three main programs, and five support
programs: 

\begin{quote}
\bigskip\noindent
$\bullet$ {\ttg mklib}: constructs a library of isochrone 
photometry from raw isochrones.

\medskip\noindent
$\bullet$ {\ttg synth}: constructs synthetic CMDs 
from the isochrones, applying the effects of interstellar extinction, 
binarity, IMF, and photometric errors as appropriate for the data.

\medskip\noindent
$\bullet$ {\ttg sfh}:   performs a 
maximum-likelihood comparison of composite model CMDs and input data 
CMDs.  The best-fit star formation history is output. 


\bigskip\noindent
- {\ttg interp}:  interpolates between adjacent 
isochrones, to improve resolution in age and/or metallicity.  Requires 
knowledge of equivalent evolutionary points (EEPs) along each 
isochrone. 
   
\medskip\noindent
- {\ttg testpop}: constructs artificial stellar 
populations from the isochrones.  {\ttg testpop} 
incorporates the photometric errors, extinction, etc. exactly as is 
done in {\ttg synth}.  These populations can be 
input to {\ttg sfh} for testing purposes.  
({\ttg testpop} is in the subdirectory of the same 
name)

\medskip\noindent
- {\ttg repop}: same as {\ttg testpop}, except used to construct an 
artificial population from the SFH amplitudes found by {\ttg sfh}.  
This is useful for model/data comparisons. ({\ttg repop} is in the 
subdirectory of the same name)

\medskip\noindent
- {\ttg testchi}: allows one to examine the parameter space of the 
fit by hand.  SF amplitudes and other parameters can be interactively 
adjusted, and the resulting fit recomputed on the fly. ({\ttg testchi} 
is in the subdirectory of the same name) 

\medskip\noindent
- {\ttg mkimages.sh}: creates CMD images from the best-fit model.  It 
also creates CMD maps of each region's contribution to the fitting 
statistic. 
\end{quote}

\clearpage

\section{Getting Started}

To begin, you need to obtain the program code from the Magellanic 
Clouds Photometric Survey website: \\
{\ttb http://ngala.as.arizona.edu/mcsurvey/}

\medskip
As of this writing, the current version is 1.1.  Note that there is a 
full version and a compact version.  The full version includes a 
sample pre-generated synthetic CMD library for testing purposes.  If 
you download the compact version, you can still generate the sample 
synthCMD library, but it will take some time.

\medskip 
The following command will unpack the directory tree of the StarFISH 
distribution. It will create a {\ttg SFH-1.1} subdirectory in the 
current directory.

\medskip\noindent
{\ttg \% tar zxvf SFH-1.1.tar.gz} \hfill (or {\ttg tar zxvf SFH-1.1compact.tar.gz}) \\

\noindent or (on Solaris): \\
\medskip
\noindent
{\ttg \% gunzip SFH-1.1.tar.gz} \\
{\ttg \% tar xvf SFH-1.1.tar}

\medskip
The package contents are shown in {\bf Table~1}.

\bigskip
\noindent{\large {\bf Table 1: Package Contents}}\\
{\scriptsize
\begin{tabular}{|l|l|}
\hline
{\bf File/Directory Name}  & {\bf Description} \\ 
\hline
{\ttg manual/}     & The StarFISH manual (this document)\\
{\ttg libcode/}    & source code and Makefile for the {\ttg mklib} program\\
{\ttg synthcode/}  & source code and Makefile for the {\ttg synth} program\\
{\ttg sfhcode/}    & source code and Makefile for the {\ttg sfh} program\\
{\ttg commoncode/} & source code used by multiple programs\\
{\ttg interpcode/} & source code and Makefile for the {\ttg geteep} and {\ttg interp} programs\\
{\ttg input/}      & input data files for {\ttg mklib}, {\ttg synth} and {\ttg sfh}\\
\hspace{20pt}{\ttg /lib.dat}   & sample {\ttg mklib} parameter file\\
\hspace{20pt}{\ttg /synth.dat} & sample {\ttg synth} parameter file\\
\hspace{20pt}{\ttg /sfh.dat}   & sample {\ttg sfh} parameter file\\
\hspace{20pt}{\ttg /iso.dat}   & sample isochrone description files\\
\hspace{20pt}{\ttg /iso.lock}  & describes how isochrones are to be locked together\\
\hspace{20pt}{\ttg /test.cmds} & sample synthetic CMD description file\\
\hspace{20pt}{\ttg /mask.none} & file indicating CMD regions to ignore in the fit (default: null mask)\\
\hspace{20pt}{\ttg /test.hold} & file indicating which SFH amplitudes should be held fixed (default: none held)\\
\hspace{20pt}{\ttg /av/*.av}   & sample extinction data files\\
\hspace{20pt}{\ttg /crowd/*.crowd} & sample crowding table from artificial star tests\\
\hspace{20pt}{\ttg /dtime.dat} & file describing age interval covered by each isochrone\\
\hspace{20pt}{\ttg /geteep.dat} & input file for {\ttg geteep} \\
\hspace{20pt}{\ttg /interp.dat} & input file for {\ttg interp} \\
\hspace{20pt}{\ttg /testchi.dat} & input file for {\ttg testchi} \\
{\ttg testpop/} & files related to the {\ttg testpop} support program\\
{\ttg testchicode/} & files related to the {\ttg testchi} support program\\
{\ttg repop/}   & files related to the {\ttg repop} support program\\
{\ttg grid/}    & files related to the {\ttg grid} support scripts\\
{\ttg iso/}     & raw isochrone photometry from Girardi et~al. (2002, 2000) 
                  and Bertelli et~al. (1994)\\
{\ttg testlib/}\footnotemark[1]  & sample isochrone library produced by {\ttg mklib}\\
{\ttg test.syn/}\footnotemark[1] & sample synthetic CMDs based on artificial star tests
                                   from an image of NGC~1978.\\
{\ttg data/}   & input data photometry goes here (NGC~1978 photometry included)\\
{\ttg output/} & {\ttg sfh} results and the {\ttg sfh.sm} plotting script\\ \hline
\end{tabular}
}

\footnotetext[1]{Not present in the compact version of the program}

\clearpage

\section{Compiling the Principal Code}

Each principal program ({\ttg mklib}, {\ttg synth}, and {\ttg sfh}) 
has its own Makefile, in its source code directory.  To compile these 
programs, simply cd to the correct directory and type {\ttg make}:

\medskip
\noindent
{\ttg \% cd /path/to/SFH-1.1} \hfill (if not already in the SFH directory) \\
{\ttg \% cd libcode} \\
{\ttg \% make} \\
{\ttg \% cd ../synthcode} \\
{\ttg \% make} \\
{\ttg \% cd ../sfhcode} \\
{\ttg \% make}

\medskip
Each program will be copied into the {\ttg SFH-1.1} directory, which 
is where you should run them.  You are now ready to experiment with 
the code.

\vspace{2em}
\section{A Quick Run}

We will use all of the default parameter values to go from a freshly 
downloaded and compiled distribution to a successful first run of 
{\ttg sfh}.  Steps labeled with an asterisk (*) may be skipped if you 
downloaded the full version of StarFISH, because this version already 
contains the pregenerated isochrone library and synthetic CMD library.

\begin{enumerate}

\item {\ttg \% cd /path/to/SFH-1.1} \hfill (if not already in the SFH directory)

\item (*) Prepare isochrones:  (see Section~\ref{sec:iso})\\
{\ttg \% cd iso} \\
{\ttg \% tar zxvf girardi.tar.gz} \hfill (unpack the isochrones) \\
{\ttg \% gawk -f parse\_iso.awk isocz*.dat} \hfill (parse isochrones) 

\item (*) Generate isochrone library: (see Chapter~\ref{ch:lib})\\
{\ttg \% cd ..} \\
{\ttg \% mkdir testlib} \hfill (create target directory before running {\ttg mklib}!)\\
{\ttg \% ./mklib < input/lib.dat} 

\item (*) Generate Synthetic CMD library: (see Chapter~\ref{ch:synth}) \\
{\ttg \% mkdir test.syn} \hfill (create target directory before running {\ttg synth}!)\\
{\ttg \% ./synth < input/synth.dat} 

\item Perform SFH Solution: (see Chapter~\ref{ch:sfh}) \\
{\ttg \% ./sfh < input/sfh.dat} 

\item Plot solution: \\
{\ttg \% cd output} \\
{\ttg \% sm} \hfill (start supermongo plotting program) \\
{\ttg sm$>$ macro read sfh.sm } \hfill (load sm script) \\
%{\tt sm$>$ dev x11} \hfill (bring up display window) \\
{\ttg sm$>$ sfh c1978 c1978} \hfill (display SFH solution)
\end{enumerate}

\clearpage

\section{What's New in Version 1.1}

There have been several improvements made since 1.0; the most 
important changes are listed here.  See the files {\ttg ChangeLog} and 
{\ttg NEWSTUFF} for more details.

\begin{itemize}
\item Fixed random-number generator.  We found a periodicity in 
the previous one. \\

\item More robust handling of input files.  The programs are now 
much less picky about the format of input files ({\itr note that 
input files from previous versions of StarFISH are not compatible 
with this version}). \\

\item Input files can now contain comment lines.  Any line in an 
input file that begins with {\ttg \#} or {\ttg *} will be ignored by 
the parser. \\

\item Filenames in the input files can now be up to 40 characters 
long (previous maximum was 24 characters) \\

\item Option to use Dolphin's ``Poisson'' fit statistic instead of 
the chi-squared statistic. \\

\item The photometric error model used in {\ttg synth}, 
{\ttg testpop}, and {\ttg repop} now properly includes covariance 
between the color and magnitude errors. \\

\item Option to use an analytic error model instead of an 
AST-derived crowding table in {\ttg synth}, {\ttg testpop}, and 
{\ttg repop}. \\

\item {\ttg sfh} writes the current best solution to the logfile more 
frequently. \\

\item New empirical isochrone interpolation code ({\ttg geteep} and 
{\ttg interp})

\end{itemize}


\chapter{mklib}\label{ch:lib}

{\ttg mklib} is used for processing raw isochrone photometry into a 
format usable by {\ttg synth}.  {\ttg mklib} applies a distance modulus, 
interpolates along the isochrone to a uniformly high resolution, and 
computes occupation probabilities (OP) for each isochrone point.

The package currently ships with the Padua isochrones, most recently
published by Girardi et~al. (2002).  We found that the main sequence 
in these isochrones is too coarsely sampled for our purposes, 
resulting in artificially "lumpy" synthetic CMDs. Therefore,  
{\ttg mklib} includes a photometric interpolation routine for points 
fainter than the main sequence turn-off (MSTO; defined for each 
isochrone in the {\tto isofile}).  If your isochrones do not require 
this interpolation, simply set the MSTO magnitude for each isochrone 
to be fainter than any isochrone point.  

\bigskip
The {\ttg mklib} input file (named {\ttg input/lib.dat} by default) 
contains the parameters listed in Table~2 (the lines 
must be present in the listed order). \\

\noindent{\large {\bf Table 2: mklib input file parameters}} \\
{\scriptsize
\begin{tabular}{|lr|l|}
\hline
{\tto isofile} &[string]& description of each isochrone: the input filename, \\
               &        & the output filename, the age, and the MSTO magnitude. \\
{\tto faint}   & [real] & the faint magnitude limit for the output isochrone library.\\
               &        & Should be several mag fainter than the data's faint limit. \\
{\tto dmag}    & [real] & photometric distance between adjacent interpolated points.\\
{\tto dmod}    & [real] & distance modulus \\
{\tto gamma}   & [real] & logarithmic IMF slope (Salpeter = -1.35) \\
{\tto nmag}    &  [int] & number of magnitude values per isochrone point \\
{\tto mag0}    &  [int] & which magnitude to check against {\bf faint} and  {\bf msto} \\
{\tto iverb}   &  [int] & verbosity flag (0=silent, 1=screen messages, 2=extra output files)  \\
\hline
\end{tabular}
}

\clearpage

\noindent
To use {\ttg mklib}:

\begin{enumerate}
\item {\bf Create an isofile that describes your isochrones.}  This 
file will also be used by {\ttg synth}.  Each line in the file should 
contain the following space-delimited fields (the column format does 
not matter):

\medskip\noindent
{\tto log(age)}~[real number], 
{\tto input raw isochrone file}~[up to 40 chars], 
{\tto output isochrone file}~[up to 40 chars], 
{\tto msto\_mag}~[real number]

\medskip
{\tto msto\_mag} is the absolute magnitude of each isochrone's MSTO point.  
If you do not know the MSTOs for your isochrones, you'll need to identify 
them.  Use the fact that at the MSTO, the occupation probability changes 
dramatically.
\medskip

\item {\bf Preprocess the isochrone photometry files}.  Each isochrone 
must be in its own file with a name matching the isochrone's entry in 
the {\tto input raw isochrone file} list (column 2 of the isofile).  
The columns required for the isochrone files are:

\medskip\noindent
{\tto mass}, {\tto mag1}, {\tto mag2}, {\tto mag3}, ..., {\tto magN}
 
\medskip
where N is the equal to the {\tto nmag} parameter in {\ttg lib.dat}.  
Column format does not matter.  The magnitudes should be expressed as 
absolute magnitudes; {\ttg mklib} will apply the distance modulus 
specified in {\ttg lib.dat} and output apparent magnitudes.  For the 
default Padua isochrones, the input isochrone files are generated by 
the {\ttg parse\_iso.awk} script. See the next section for details on 
the default isochrone-processing.  Make sure the {\tto mag0} parameter 
indicates the pass-band which should be used for comparing to the 
{\tto faint} and {\tto msto\_mag} values.  

\medskip
\item {\bf Create the target output directory, if necessary}. The 
output directory is specfied as part of the output filenames in the 
{\tto isofile}.

\medskip
\item {\bf Run {\ttg mklib}:} \\
{\ttg \% ./mklib < input/lib.dat}

\medskip
Depending on the verbosity flag, {\ttg mklib} may provide screen 
messages regarding which isochrone it is working on.  The output 
directory will contain the processed isochrones, with the following 
columns:

\medskip\noindent
{\tto mag1}, {\tto mag2}, {\tto mag3}, ..., {\tto magN}, {\tto OP}, 
{\tto mass}

\medskip
where the {\tto magN} are the apparent magnitudes in each band, 
{\tto OP} is the relative occupation probability associated with each 
point (determined from the IMF and from the relative duration of each 
evolutionary stage), and {\tto mass} is the {\bf initial} mass of each 
isochrone point, not including any mass-loss that may have occured 
since the star was formed.
\end{enumerate}

\clearpage

\section{Pre-processing the Padua Isochrones}\label{sec:iso}

The {\ttg iso/} directory contains files related to the Padua 
isochrones, which we ship with StarFISH for convenience.  You are free 
to use any other isochrone set that you wish, as long as it can be 
placed in the format that {\ttg mklib} requires (separate files for 
each isochrone, with the column format specified in step 2 above).

The Padua isochrones are disributed as a tar archive, which we include 
in StarFISH as the file {\ttg iso/girardi.tar.gz}.  The archive contents 
are listed in Table~3.

\bigskip
\noindent{\large {\bf Table 3: Padua isochrone package contents}} \\
\begin{tabular}{|l|l|}
\hline 
{\ttg isocz0[001,004,01,04,08,19,30].dat} &  Girardi et~al. isochrones\\
{\ttg isocsummz0[001,004,01,04,08,19,30].dat} &  Tables of Equivalent 
    Evolutionary \\
    & Points (EEPs) \\ 
{\ttg read.me} &  Girardi's readme file\\ 
\hline
\end{tabular}

\vspace{3em}
Each of the isochrone files {\ttg isocz0*.dat} contains isochrones of 
all ages for a given metallicity.  StarFISH needs a separate file for 
each isochrone, so we provide the AWK script {\ttg parse\_iso.awk} to 
divide each raw isochrone collection into individual files.  We also 
provide the script {\ttg parse\_summ.awk} to divide the EEP table 
collections into individual EEP tables for each isochrone.

If you downloaded the full version of StarFISH, then you do not need 
to run the parsing scripts; you already have the processed isochrone 
files, named {\ttg iso/zNNNN\_tt.tt} (where {\ttg NNNN} is a 4-digit 
code for the metallicity, such as {\ttg 0080}; and {\ttg tt.tt} is the 
log of the age, such as {\ttg 10.00}).  If you do not have these 
files, generate them like this:

\medskip\noindent
{\ttg \% cd iso}\\
{\ttg \% tar zxvf girardi.tar.gz} \\
{\ttg \% gawk -f parse\_iso.awk isocz*.dat} \\
{\ttg \% gawk -f parse\_summ.awk isocsummz*.dat} \\

\medskip
If you are satisfied with the age and metallicity resolution provided 
by the default Padua isochrones, then you are all done here.  If you 
want to try to interpolate between isochrones to increase age or 
metallicity resolution, refer to the chapter discussing the 
{\ttg interp} support program.


\chapter{synth}\label{ch:synth}

{\ttg synth} constructs a synthetic CMD library from the isochrone 
library, incorporating the photometric effects of extinction, 
binarity, the IMF, and photometric errors.  Each synthetic CMD is 
represented as a list of ``pixel values'', equal to the relative number 
of stars found at the corresponding subregion in the CMD.  If the data 
contain more than two filters, then multiple CMDs can be created for 
each isochrone, so that all of the data can be utilized in the fit.  
For example, if your data include $U$, $B$, and $V$ photometry, you 
might use {\ttg synth} to create two CMDs per isochrone: one with 
$U-B$ vs. $B$ axes, and another with $B-V$ vs. $V$ axes.

Some of the user-provided parameters (extinction, photometric errors, 
and binarity) are first encountered in the {\ttg synth} program. 
Others (the IMF and distance modulus) have already been used to 
create the isochrones output by {\ttg mklib}.  Therefore, if you need 
to change the IMF or distance modulus you must run {\ttg mklib} first. 

Depending on the age and metallicity resolution of your isochrone 
library, and on the quality of your photometric data, it is likely
that some of the isochrones are photometrically degenerate.  If 
{\ttg synth} is allowed to produce degenerate synthetic CMDs, then the 
SFH solution determined by {\ttg sfh} will be ambiguous, and will 
suffer from large correlated uncertainties among the SFH amplitudes.  
To avoid ambiguous SFH fits, {\ttg synth} can combine 
photometrically-degenerate isochrones together into the same synthetic 
CMD.  This is referred to as {\bf locking} the isochrones. 

\clearpage

\noindent The synth input file ({\ttg input/synth.dat}) contains the 
parameters listed in Table~4. {\itr Note that synth input files from 
previous versions of StarFISH will no longer work}.

\bigskip
\noindent{\large {\bf Table 4: synth input file parameters}} \\
{\scriptsize
\begin{tabular}{|lr|l|}
\hline
\multicolumn{3}{|c|}{\bf Filenames} \\ 
\hline
{\tto isofile}  &[string]& description of the isochrones (can use the file from {\ttg lib.dat}) \\
{\tto lockfile} &[string]& file describing the combination of degenerate isochrones\\
{\tto hotfile}  &[string]& list of extinction measurements of hot (young) stars\\
{\tto coldfile} &[string]& list of extinction measurements of cool (old) stars\\
{\tto crowd1}   &[string]& crowding table from artificial star test\\
{\tto crowd2}   &[string]& output file for lookup table of binned delta-magnitudes \\ 
\hline
\multicolumn{3}{|c|}{\bf CMDs} \\ 
\hline
{\tto nmag} &  [int] & same as nmag in {\ttg lib.dat}\\
{\tto ncmd} &  [int] & number of CMDs to construct per isochrone group\\
{\tto mag0} &  [int] & same as mag0 in {\ttg lib.dat}\\ 
{\tto dpix} & [real] & size of CMD pixels (in magnitudes) {\itr[NEW]}\\
\hline
\multicolumn{3}{|c|}{\bf parameters for each of the CMDs} \\ 
\hline
{\tto xeq}    &[string]& filter equation for the x-dimension\\
{\tto yeq}    &[string]& filter equation for the y-dimension\\
{\tto xmin}   & [real] & smallest allowable x-value\\
{\tto xmax}   & [real] & largest allowable x-value\\
{\tto ymin}   & [real] & smallest allowable y-value\\
{\tto ycmax}  & [real] & largest allowable y-value, for crowding table stars\\
{\tto ypmax}  & [real] & largest allowable y-value, for data/model stars\\
{\tto suffix} &[string]& filename suffix for the synthetic CMD output files\\ 
\hline
\multicolumn{3}{|c|}{\bf Crowding} \\ 
\hline
{\tto dbinx} & [real] & width of crowding bins, in magnitudes\\
{\tto dbiny} & [real] & height of crowding bins, in magnitudes\\
{\tto emin}  & [real] & minimum delta-magnitude cutoff\\
{\tto emax}  & [real] & maximum delta-magnitude cutoff\\
{\tto dbin}  & [real] & binsize of delta-magnitude histograms\\ 
\hline
\multicolumn{3}{|c|}{\bf Reddening parameters} \\ 
\hline
{\tto red(i)} & [real] & extinction in mag(i) (relative to the filter in which extinction is measured) \\ 
\hline
\multicolumn{3}{|c|}{\bf Run-time parameters} \\ 
\hline
{\tto verb}        &  [int] & verbosity flag (higher N = more output)\\
{\tto interp\_err} & [flag]\footnotemark[2] & 1 = interpolate errors between bracketing crowding bins {\itr[NEW]}\\
{\tto fake\_errs}  & [flag]\footnotemark[2] & 1 = use analytic error model instead of crowding table {\itr[NEW]}\\
{\tto nscale}      &  [int] & number of model stars in each CMD\\
{\tto seed}        &  [int] & random seed value\\
{\tto mass1}       & [real] & minimum mass (for OP normalization)\\
{\tto mass2}       & [real] & maximum mass (for OP normalization)\\
{\tto gamma}       & [real] & logarithmic IMF slope (Salpeter = -1.35)\\
{\tto faint}       & [real] & faint limit for generated model stars\\
{\tto fbinary}     & [real] & binary fraction\\ 
\hline
\end{tabular}
}
\footnotetext[2]{A flag is an int whose value can either be 0 or 1.}

\clearpage
\noindent To use {\ttg synth}:

\begin{enumerate}
\item {\bf Construct the {\tto lockfile}.}  This file describes how the 
isochrones are to be locked together into groups.  Again, this is 
necessary if isochrones of similar age are photometrically degenerate, 
to avoid ambiguous SFH fits.  The columns of the {\tto lockfile} are as 
follows ({\itr note that lockfiles from previous versions of StarFISH 
will no longer work}):

\medskip
\noindent
{\tto group\_id}[int], 
{\tto isoname}[up to 40 characters], 
{\tto synthfilestem} [up to 40 characters]

\medskip
{\tto group\_id} identifies the locked group to which the isochrone 
belongs.  The actual {\tto group\_id} values do not matter; when the 
value of one line is different from that of the previous line, a new 
synthetic CMD is started.

{\tto isoname} is the isochrone filename.  It is only used to 
check against the filename in the {\tto isofile}.  If they don't 
match, an error is triggered; there must be a one-to-one 
correspondence of the isochrones in the {\tto isofile} and the 
{\tto lockfile}.

{\tto synthfilestem} is the filename stem for the output synthetic 
CMD pixel file.  The output filenames are constructed by concatenating 
the CMD suffixes specified in {\ttg synth.dat} onto 
{\tto synthfilestem} (these suffixes are typically the combination of 
magnitudes; for example a $U-B$ vs. $B$ CMD might have a suffix of 
``{\ttg .ub}'').

\item {\bf Construct the extinction files ({\tto hotfile}, 
{\tto coldfile}).}  The extinction files each contain a single column: 
extinction values, $A_{\lambda}$, measured in the magnitude whose 
reddening parameter in {\ttg synth.dat} is set to 1.00 (in the 
provided {\ttg synth.dat}, red(3) is set to 1.00; this is the $V$ 
band, so the extinction files list $A_V$ values).  There are two files 
({\tto hotfile} and {\tto coldfile}) to account for the different 
extinction properties of young and old stellar populations.  If you do 
not require population-dependent extinction, simply point 
{\tto hotfile} and {\tto coldfile} to the same file.  If you do not 
require extinction modeling at all, point {\tto hotfile} and 
{\tto coldfile} to the null extinction file, {\ttg input/av/zero.av}.
   
\item {\bf Construct the crowding table ({\tto crowd1})}. This is most 
likely constructed from artificial star tests.  The crowding table 
contains:

\medskip   
{\tto ra}, {\tto dec}, {\tto mag1}, {\tto dmag1}, {\tto mag2}, {\tto dmag2}, 
..., {\tto magN}, {\tto dmagN}

\medskip\noindent
column format doesn't matter.  The crowding table should contain all 
artificial stars injected into your data images, even if they were not 
detected in your photometric pipeline.  In fact, these non-recovered 
``dropout'' stars are extremely important, because {\ttg synth} uses 
them to determine the completeness rate as a function of photometric 
position in the CMDs.  For bands in which an artificial star was 
detected, the listed {\tto mag} value is its {\bf recovered} 
magnitude, and {\tto dmag} is (mag(recovered) - mag(input)).  For 
bands in which an artificial star dropped out, {\tto mag} is the 
star's {\bf input} magnitude, and {\tto dmag} is set to 9.99 in order 
to flag it as a dropout in that band.  

If you decide not to perform artificial star tests on your images 
(although such tests are {\it highly} recommended), you can either 
generate a fake crowding table by hand, based on your estimates of the 
scatter and completeness rates in each band, or you can set the 
{\tto fake\_errs} parameter in {\ttg synth.dat} to 1, which will use a
hueristic analytic error model instead of the empirical crowding 
table.  In this case, you will need to modify the code in 
{\ttg synthcode/fakecrowd.f} to generate photometric errors and 
completeness rates appropriate for your data.  The comments in that 
file will guide you in making these changes.

\item Decide how many CMDs you would like, and what their axes will 
be.  You specify the CMD axes with the {\tto xeq(icmd)} and 
{\tto yeq(icmd)} string variables in {\ttg synth.dat}.  These strings 
indicate the filter combination for each CMD axis.  For example, you 
have $VRI$ photometry, and you are using $V$=1, $R$=2, $I$=3.  Your 
first CMD could be: $V-I$ vs. $V$, so {\tto xeq(1)} = 1-3, 
{\tto yeq(1)} = 1.  Your other ``CMD" could be a two-color diagram: 
$V-R$ vs. $V-I$, so {\tto xeq(2)} = 1-2, {\tto yeq(2)} = 1-3.  The 
strings can be either the sum or the difference of two filter indices, 
or one index by itself.  Do not put whitespace in the equations; each 
must be a single ``word''.

You will also need to specify photometric limits for each CMD 
(parameters {\tto xmin}, {\tto xmax}, {\tto ymin}, {\tto ycmax}, and 
{\tto ypmax}).  Note that instead of one {\tt ymax}, there are two 
variables: {\tto ycmax} and {\tto ypmax}.  {\tto ycmax} is the faint 
limit for the crowding bins, and {\tto ypmax} is the faint limit for 
the CMD itself.  {\tto ycmax} should be at least 2 or 3 mag fainter 
than {\tto ypmax}, because very faint stars can be blended and 
detected at much brighter magnitudes than their true magnitude.  

Also specify the size of CMD ``pixels'' in the output synthetic CMD 
files ({\tto dpix}), in magnitudes.

Finally, specify a suffix string that will be concatenated with 
{\tto synthfilestem} to construct the output file name for each 
synthetic CMD.  The suffices can be up to 8 characters long, and 
typically indicate something about the CMD axes. For example, a
$B-V$ vs. $V$ CMD might have a suffix ``{\ttg .bv}''.

\item Specify the size of the crowding bins in each direction 
({\tto dbinx} and {\tto dbiny}).  Artificial stars in the crowding 
table are binned in the CMDs so that the brightness- and 
color-dependence of the photometric errors can be accounted for.
Independent $\Delta mag$ histograms will be constructed from the 
artificial stars present in each bin.  

Specify the limits for the delta-mag histograms with {\tto emin} and 
{\tto emax}.  Any $\Delta mag$ value in the crowding table outside 
these limits is treated as a photometric dropout.

The {\tto dbin} parameter sets the width of the bins in the delta-mag 
histograms ({\itr this parameter used to determine the size of the 
synthetic CMD pixels as well, but there is now a separate parameter 
for that: dpix}).

\item Specify the reddening law to use, by setting the relative 
extinction for each band: {\tto red(imag)}.  The band for which you 
have measured extinction values in {\tto hotfile} and {\tto coldfile} 
should have {\tto red(imag)} = 1.0.  The rest of the values should be 
the extinction in that band, assuming the extinction in the reference 
band is 1.0.  For example, you have $BV$ photometry, and measured 
A$_V$ extinction values.  A$_V$ = 1.0.  If we assume that 
A$_V$/E(B-V) = 3.2 = A$_V$/(A$_B$ - A$_V$), then A$_B$ = 1.3125. 

\item Specify miscellaneous parameters.  {\tto nscale} sets the number 
of model stars to include in each CMD.  {\tto nscale} should be much 
larger than the number of stars in your data CMDs, so that the Poisson 
noise in the models is much smaller than in the data.  {\tto iseed} is 
the random seed value; it can be any integer.  {\tto mass1}, 
{\tto mass2}, and {\tto gamma} are IMF parameters used for normalizing 
the pixel values.  {\ttg synth} will only attempt to add model stars 
brighter than the {\tto faint} parameter.  This is an efficiency 
feature: without this cutoff, the code would waste a lot of time 
adding undetectable stars.  Set {\tto faint} to just fainter than 
where your completeness rate reaches 0\%.

\item Once the parameter file is complete, and the other input files 
are ready, you can finally run {\ttg synth}.  It will take quite a bit 
of time to complete, depending on computing power and the number of 
isochrones.  I found it usually took several hours to process 150 
isochrones on a 700 MHz linux box.  
\end{enumerate}

You can examine the synthetic CMDs as images, using either the IRAF 
{\tt rtext} command, or the {\ttg mkgif.bat} and {\ttg pxl2ppm.awk} 
scripts I provide in the {\ttg commoncode} directory.


\chapter{sfh}\label{ch:sfh}

{\ttg sfh} performs a ``chi-squared-like'' minimization comparison 
between data photometry and a linear combination of the synthetic 
CMDs.  Each synthetic CMD is modulated by an {\bf amplitude} that is 
proportional to the number of stars with the age and metallicity of 
the synthetic CMD's parent isochrone(s).  These amplitudes form the 
multidimensional parameter space of the minimization.

I say ``chi-squared-like'' because you may actually choose among three 
statistics to determine the quality of the fit: chi-squared, 
Lorentzian, or Poisson.  The three parameters weight outliers slightly 
differently, but they are otherwise quite similar.  Hereafter, I will 
simply refer to the fitting statistic as ``chi-squared''.

The minimization analysis employs the downhill simplex ``amoeba'' 
algorithm as described in Numerical Recipes (Press {\it et al.}; 
hereafter NR).  We have written our own implementation of this 
algorithm.  It works by sampling a small region of the parameter  
space and determining the local chi-squared gradient.  It then takes a 
small step in the down-gradient direction, and determines the local 
chi-squared gradient around this new location.  The process iterates 
until a (nearly) zero gradient is found, signalling that a minimum
has been found.  

The amoeba algorithm in NR has some safeguards against finding local 
minima, but we found that it was still prone to getting stuck.  The 
simplex by definition can only sample parameter space along directions 
parallel to the N axes of the parameter space; it is blind to off-axis 
directions.  However, as it nears the minimum, lower chi-squared 
values will tend to be found when a change in one parameter is 
complemented by a change in one or more other parameters.  In other 
words, downward gradients will tend to lie along off-axis directions.

To remedy this problem with the pure simplex method, we add a 
random-search loop, which is triggered whenever the simplex claims it 
has found a minimum.  In the random-search loop, we select a random 
direction in the multidimensional parameter space, and take a small 
step along that direction.  If the step yields a smaller chi-squared
value, we continue stepping along that direction until we reach a step 
that no longer decreases chi-squared.  This is repeated for a large 
number of random directions, and the parameter space location which 
yielded the lowest overall chi-squared value is recorded.  The simplex 
is then restarted at this new parameter space location.

This iteration between the downhill simplex algorithm and the 
random-search loop is repeated until the random-search loop cannot 
find a direction which yields a lower chi-squared value.
We then accept the last simplex location as the true minimum, and 
output the best-fit SFH amplitudes.

\bigskip
\noindent The {\ttg sfh} input file ({\ttg input/sfh.dat}) contains 
the following parameters:

\medskip
\noindent
{\scriptsize
\begin{tabular}{|lr|l|} 
\hline
\multicolumn{3}{|c|}{\bf Filenames} \\ 
\hline
{\tto datpre}   &[string]& input data file prefix\\ 
{\tto cmdfile}  &[string]& Sample synthetic CMDs description file\\
{\tto maskfile} &[string]& Allows one to ignore specified regions of the CMDs\\
{\tto holdfile} &[string]& file w/ SFH amplitudes that should be held fixed\\
{\tto outfile}  &[string]& contains final best-fit SFH amplitudes\\
{\tto logfile}  &[string]& file w/ current best SFH (useful for monitoring, and for\\
                && restarting an interrupted run)\\
{\tto plgfile}  &[string]& log of all tested parameter locations \\
                && (this file will be very large; not used by default: {\tto iplg}=0)\\
{\tto chifile}  &[string]& output file containing grid populations and chi-squared values\\
\hline
\multicolumn{3}{|c|}{\bf Synthetic CMD parameters} \\ 
\hline
{\tto nfree}  & [int]& number of independent isochrone amplitudes\\
{\tto ncmd}   & [int]& number of CMDs over which to calculate chi-squared\\
{\tto npix}   & [int]& binning factor for CMDs\\
{\tto dpix}   &[real]& size of synthetic CMD pixels (same as in {\ttg synth.dat})\\
\hline
\multicolumn{3}{|c|}{\bf Parameters for each CMD} \\ 
\hline
{\tto suffix(i)} &[string]& CMD filename suffix (same as in {\ttg synth.dat})\\
{\tto   xmin(i)} & [real] & calculate chi-squared within these CMD limits\\
{\tto   xmax(i)} & [real] & calculate chi-squared within these CMD limits\\
{\tto   ymin(i)} & [real] & calculate chi-squared within these CMD limits\\
{\tto   ymax(i)} & [real] & calculate chi-squared within these CMD limits\\
{\tto   nbox(i)} &  [int] & number of grid boxes within region defined by above limits\\ 
\hline
\multicolumn{3}{|c|}{\bf Runtime parameters} \\ 
\hline
{\tto seed}      &  [int]&  random seed value \\
{\tto fit\_stat} &  [int]&  0 = use chi-squared; 1 = use Lorentzian; 2 = Use Poisson\\
{\tto uselog}    & [flag]&  1 = start searching at logged position in logfile\\
{\tto iplg}      & [flag]&  1 = generate plgfile (warning: this file will be very large!)\\
{\tto gtype}     & [flag]&  grid definition flag (0 = uniform grid; 1 = custom grid)\\
{\tto iverb}     &  [int]& verbosity flag (larger N = more screen output)\\
{\tto lambda}    & [real]& initial size of simplex\\
{\tto conf}      & [real]& confidence interval for computing errorbars (e.g., 1 sigma = 0.68)\\
{\tto cthr}      & [real]& if ($\Delta\chi^2$ $>$ cthr), consider new location an improvement\\
{\tto ptol}      & [real]& if ($|p\_new - p\_old| > ptol$), consider new location different\\
{\tto ftol}      & [real]& if simplex's chi-squared values span less than ftol, signal a zero \\
                 &       & gradient (i.e., a minimum) \\
{\tto nvec}      &  [int]& number of parameter space directions in random-search loop\\
{\tto ntry}      &  [int]& number of iterations for determining errorbars\\
\hline
\end{tabular}
}

\clearpage

\noindent
To use {\ttg sfh}:

\begin{enumerate}
\item {\bf construct the {\tto cmdfile}}.  The file's columns are:

\medskip\noindent
{\tto z\_metal}, {\tto log\_age}, {\tto filenamestem}

\medskip\noindent
Column format does not matter.  Note that some lines in the
{\tto cmdfile} may represent multiple isochrones.  In these cases, 
just choose a representative age/metallicity for the group.  These 
values are only used to identify the amplitudes in the output file.  
The {\tto filenamestem} needs to point to the synthetic CMD files 
generated with {\ttg synth}.

\item {\bf Decide on a CMD gridding strategy}.  {\ttg sfh} works by 
binning the CMDs and comparing the number of data and model stars in 
each bin.  The default is to uniformly bin the CMDs in each dimension 
with a binsize of {\tto npix}$\times${\tto dpix}.  If you want to use 
a custom grid instead, write a subroutine to assign gridbox numbers to 
each CMD pixel, add a call to your subroutine in {\ttg grid.f}, and 
set {\tto gtype} to 1 in {\ttg sfh.dat}.  You must also add your 
custom-grid source code files to {\ttg sfhcode/Makefile}.  An example 
custom grid can be reviewed in the files {\ttg ubbgrid.f}, 
{\ttg bvvgrid.f}, and {\ttg viigrid.f}.  These grids were constructed 
for our LMC data, and are fine in densely populated regions of the 
CMDs and coarse in sparsely populated regions.

The boxes in each CMD's grid are identified by a single number, the
box number.  Thus the distribution of stars in the CMDs can be
expressed as a two-dimensional array whose indices identify the CMD
and the gridbox within that CMD: {\tt nstars(icmd, ibox)}. 
{\tt icmd} refers to which CMD you are considering, {\tt ibox} is the 
number of the box in that CMD.  Indicate the total number of boxes in 
each CMD with the {\tto nbox(icmd)} parameter. 

\item {\bf Construct the {\tto maskfile}}.  the file's columns are:

\medskip\noindent
{\tto icmd},  {\tto ibox},  {\tto maskflag}

\medskip\noindent
Column format does not matter.  if {\tto maskflag}=1, the stars in 
({\tt icmd}, {\tt ibox}) will be excluded from the determination of 
the fit.

\item {\bf  Preprocess the data photometry}.  You need to have a 
separate photometry file for each CMD.  The filenames need to be 
({\tto datpre} + {\tto sffx(icmd)}), where {\tto datpre} and 
{\tto sffx(icmd)} are specified in {\ttg input/sfh.dat}.  The first 
column in each CMD file is the x-dimension magnitude, and the second 
column is the y-dimension magnitude.  Other columns are ignored, but 
could be used for (ra, dec) coordinates or other useful information.   
Note that the {\tto sffx(icmd)} values need to be the same as the 
{\tto suffix(icmd)} values in {\ttg synth.dat}, and that the CMD 
limits ({\tto xmin}, {\tto xmax}, {\tto ymin}, {\tto ymax}) need to be 
the same also.

\item {\bf Adjust the sfh runtime parameters.}  The default values 
should work well, but feel free to tweak them.

\item {\bf Run {\ttg sfh}}.  Depending on the value of {\tto iverb}, 
it may provide screen output on its progress (status messages, 
current lowest chi-squared value).  Generally, larger values of 
{\tto iverb} produces more screen output.  {\tto iverb}=0 will
print no screen messages; {\tto iverb}=3 provides ``live'' updates
of the convergence process.  If the run gets interrupted for 
some reason, you can resume at the last logged position by setting 
{\tto uselog}=1 (assuming the {\tto logfile} exists).

\item {\bf Evaluate the goodness-of-fit.}  When the run is complete, 
it will print out the lowest chi-squared value found.  Divide this by 
({\tto nboxes}-{\tto nfree}) to get the reduced chi-squared, which 
should probably be less than $\sim10$ for a good fit.  

To examine the SFH solution, you can use the {\ttg sfh.sm} supermongo 
script (in the {\ttg output/} subdirectory).  All of the customizable
parameters for this script are collected at the top of the file.  See
the comments in the file for guidance in adjusting these settings.

If there is a problem with the SFH fit (e.g., high chi-squared value), 
you can use the following diagnostics (each of these programs has its 
own Chapter in this manual):

\begin{enumerate}
\item the {\ttg repop} program produces artificial stellar photometry 
based on the best-fit amplitudes, to be compared to the data CMDs. 

\item the {\ttg mkimages.sh} script in the {\ttg grid/} subdirectory, 
which converts the gridbox populations in the {\tto chifile} into 
images, and embeds them into an HTML document for easy comparison.  

\item the {\ttg testchi} program allows you to tweak the SFH 
amplitudes and parameter values.  It then calculates a new chi-squared 
value for the new model.
\end{enumerate}

\end{enumerate}


\chapter{geteep and interp}

{\ttg geteep} and {\ttg interp} are used to interpolate between 
isochrones to increase the nominal matallicity and/or age resolution.
To compile the programs, cd to the {\ttg interpcode} directory and 
type {\ttg make}.  This will create the {\ttg geteep} and 
{\ttg interp} programs in the root {\ttg SFH-1.1} directory.

{\ttg geteep} identifies empirical ``equivalent evolutionary points''
(EEPs) in each isochrone.  True EEPs are stellar evolution events 
which can be identified in the isochrones, such as ``the main sequence 
turn-off'' or ``the tip of the red giant branch''.  These are very 
useful for interpolating between isochrones.  In fact, the older 
Padua isochrones provide EEP tables for this purpose, but the new
isochrones published in 2002 did not have this information (StarFISH 
includes the older EEP tables, but we don't make use of them since 
they are missing for some isochrones).

Since we don't have EEP tables for all of the isochrones, we need a 
way to identify EEPs (or something like them), which is what 
{\ttg geteep} does.  It identifies empirical EEPs (eEEPs) as local 
extrema in color or brightness, along the isochrone track in one of 
the CMDs.  Many of these points correspond to ``true'' EEPs (e.g., the 
TRGB is a local brightness maximum).  The point is to identify points 
which represent the same evolutionary state in two similar isochrones; 
we feel that the color/brightness extrema do this well.

The {\ttg geteep} input file has a two-line header.  The first line 
contains an integer equal to the number of magnitudes representing 
each point in the isochrones.  The second line contains an integer 
identifying the magnitude which will be the vertical ``brightness'' 
axis in the target CMD (in which eEEPs will be identified).  The 
horizontal axis is automatically set to be the color index 
(mag(i-1) - mag(i)), so don't choose ``1'' for the vertical axis.
The remainder of the file lists the isochrone files to be processed,
and the list of output isochrones in which eEEPs have been identified.

Once you have run {\ttg geteep} to identify and label the eEEPs in 
each isochrone, you can run {\ttg interp} to interpolate between the 
isochrones.  The {\ttg interp} input file has three header lines at 
the top which identify the number of magnitudes, the vertical-axis 
magnitude, and the fractional distance between the bracketing 
isochrones for the target interpolated isochrones.  For example, if 
the bracketing isochrones have metallicity 0.004 and 0.008, and the 
target isochrones are to have metallicity 0.005, the fractional 
distance should be 0.25. (This assumes linear variation with 
metallicity, which is probably Ok over small changes in metallicity).

The rest of the {\ttg interp} input file is the list of isochrones to 
be interpolated.  In each row, column 1 is the first bracketing 
isochrone, column 2 is the second bracketing isochrone, and column 3 
is the target isochrone.  Again, there is no strict column format; 
each filename can be up to 40 characters long.


\chapter{testpop}\label{ch:testpop}

{\ttg testpop} constructs artificial stellar photometry based on a 
user-specified star formation history.  It is basically a different 
front-end to the {\ttg synth} code base.  Build {\ttg testpop} by 
typing {\ttg make} in the {\ttg testpop} directory.  {\ttg testpop} 
reads a (probably modified) version of the {\ttg synth} input file, 
which must be named {\ttg testpop/synth.dat}.  It also reads its 
own input parameter file, which is specified on the command line:

\medskip
\noindent 
{\ttg \% ./testpop < input\_file}

\medskip
\noindent 
The input file should contain the following 8 header lines:

\medskip
\noindent
\begin{tabular}{|lr|l|} \hline
{\tto pre}      &[string]& a prefix for the output photometry files (up to 8 characters) \\
{\tto lockflag} &[flag]& 1 = use locked amplitudes (use {\tto lockfile} in synth.dat) \\
{\tto nstars}   &[int]& an amp of 1.0 will draw {\tto nstars} masses from the isochrone \\
{\tto dmod}     &[real]& delta-distance modulus.  For adjusting the DM of the isochrones \\
{\tto fext}     &[real]& additional extinction multiplier \\
{\tto gamma}    &[real]& IMF slope ({\ttg testpop} recalculates OPs) \\
{\tto fbinary}  &[real]& Binary fraction \\ 
{\tto sfrflag}  &[flag]& 1 = amps are expressed as $M_\odot/yr$; 0 = amps are Nstars/bin \\ \hline
\end{tabular}

\medskip
\noindent
The values for {\tto gamma} and {\tto fbinary} supercede the values in 
{\ttg synth.dat}.  Immediately following these 8 parameters, the input 
file should contain N lines with the following columns:

\medskip
\noindent
{\ttg $Amp_{SFH}$, $Z_{metal}$, $log(age)$, $num_{iso}$}

\medskip
\noindent
There should be one line for each isochrone in the isofile (if 
{\tto lockflag}=0), or one line for each independent isochrone group 
(if {\tto lockflag}=1).  {\ttg testpop} determines the number of stars 
to generate for isochrone $i$ according to the formula:

$$N(i) = nstars \times Amp(i) \times frac(i) \times \Delta t(i) / num_{iso}(i)$$

\noindent where: $frac(i)$ is the fraction of the full mass range that 
is represented on the isochrone ($frac(i)$ decreases with age, as more 
stars have evolved to their non-luminous end-states); $\Delta t(i)$ is 
the duration of the age bin, in Gyr; and $num_{iso}(i)$ is the number 
of isochrones in the current isochrone group.  If {\tto sfrflag}=0, 
then $\Delta t(i)=1.0$ for all $i$.  If {\tto lockflag}=0, then
$num_{iso}(i)=1$ for all $i$.

Be aware that, in general, many more stars are generated than actually 
appear in the CMDs, because a large fraction of the generated stars 
will be too faint to detect.  

\medskip
\noindent
{\ttg testpop} outputs {\tt ncmd} output photometry files, each 
containing the following data:

\medskip
\noindent
{\ttg $mag_x$},  {\ttg $mag_y$}, {\ttg $\Delta_x$}, {\ttg $\Delta_y$}

\medskip
\noindent
These files may be used directly as input data to the {\ttg sfh} 
program.


\chapter{repop}\label{ch:repop}

\noindent
{\ttg repop} is nearly identical to {\ttg testpop}.  The only 
difference is that the input SFH amplitudes are generated from the SFH 
amplitudes output by {\ttg sfh}.  {\ttg repop} constructs an 
artificial stellar population based on the best-fit SFH solution, so 
that it can be compared to the input population.

Before using {\ttg repop}, you must first generate the input SFH 
amplitudes file using {\ttg mkinput.awk}:

\medskip
\noindent
{\ttg \% cd repop} \\
{\ttg \% gawk -f mkinput.awk -v pre=file\_prefix sfh\_output\_file}

\medskip
You may need to modify this script to make it suitable to your data.
Specifically, the number of isochrone groups and number of isochrones 
per group are hard-coded, as are the metallicity strings.

Now you can actually run {\ttg repop}, specifying the new input file 
on the command line:
\medskip
\noindent
{\ttg \% repop $<$ file\_prefix.input}

\medskip
The number of stars constructed by {\ttg repop} is probably different 
from the number of data stars.  To generate the same number of stars, 
follow these steps:

\begin{enumerate}
\item use {\ttg wc} to count the number of stars in one of repop's 
output CMD files, and the corresponding input data CMD file.
 
\item multiply {\tto nscale} (in the {\ttg repop} input file) by 
N(data)/N(repop)

\item run {\ttg repop} again
\end{enumerate}

The {\ttg repop.sm} script can be used to display the CMDs of the 
generated population.  Again, you may need to modify this script.  
This script as written displays the repopulated B-V,V CMD alongside 
the original B-V,V CMD for easy comparison. It takes two arguments, 
the prefixes of the repopulated and original photometry files.


\chapter{testchi}\label{ch:testchi}

{\ttg testchi} provides an interactive front-end to the 
{\ttg fitstat.f} subroutine, which calculates the fitting statistic of 
a given model SFH.  Compile testchi by typing {\ttg make} in the 
{\ttg testchi} directory (the executable is placed in the root 
{\ttg SFH-1.1} directory).

The {\ttg testchi} input file is a simplified version of the 
{\ttg sfh} input file, with several parameters removed.  The 
{\ttg testchi} input file contains the following parameters:
 
\medskip
\noindent
{\scriptsize
\begin{tabular}{|lr|l|} 
\hline
\multicolumn{3}{|c|}{\bf Filenames} \\ 
\hline
{\tto datpre}   &[string]& input data file prefix\\ 
{\tto cmdfile}  &[string]& Sample synthetic CMDs description file\\
{\tto maskfile} &[string]& Allows one to ignore specified regions of the CMDs\\
{\tto holdfile} &[string]& file w/ SFH amplitudes that should be held fixed\\
{\tto ampfile}  &[string]& the input model SFH amplitudes to test\\
{\tto chifile}  &[string]& output file containing grid populations and chi-squared values\\
\hline
\multicolumn{3}{|c|}{\bf Synthetic CMD parameters} \\ 
\hline
{\tto nfree}  &[int]& number of independent isochrone amplitudes\\
{\tto ncmd}   &[int]& number of CMDs over which to calculate chi-squared\\
{\tto npix}   &[int]& binning factor for CMDs\\
{\tto dpix}   &[real]& size of synthetic CMD pixels (same as in {\ttg synth.dat})\\
\hline
\multicolumn{3}{|c|}{\bf Parameters for each CMD} \\
\hline
{\tto suffix(i)} &[string]& CMD filename suffix (same as in {\ttg synth.dat})\\
{\tto   xmin(i)} &[real]& calculate chi-squared within these CMD limits\\
{\tto   xmax(i)} &[real]& calculate chi-squared within these CMD limits\\
{\tto   ymin(i)} &[real]& calculate chi-squared within these CMD limits\\
{\tto   ymax(i)} &[real]& calculate chi-squared within these CMD limits\\
{\tto   nbox(i)} &[int]& number of grid boxes within region defined by above limits\\ 
\hline
\multicolumn{3}{|c|}{\bf Runtime parameters} \\ 
\hline
{\tto fit\_stat} &[int]&  0 = use chi-squared; 1 = use Lorentzian; 2 = Use Poisson\\
{\tto gtype}     &[flag]&  grid definition flag (0 = uniform grid; 1 = custom grid)\\
{\tto iverb}     &[int]& verbosity flag (larger N = more screen output)\\
\hline
\end{tabular}
}

\medskip
To use {\ttg testchi}, adjust the parameters in {\ttg testchi.dat} (or 
create a new input file) and run the program:

\medskip\noindent
{\ttg \% ./testchi < input/testchi.dat}

\medskip
The chi-squared value resulting from a comparison of the data and the 
model described by the {\tto outfile} is printed to the screen, and 
the gridbox populations are written to the {\tto chifile}.  

{\ttg testchi} is very useful for quickly seeing changes in the 
chi-squared value when the SFH amplitudes (or other parameters) are 
changed.  You can examine the distribution of chi-squared in the CMDs 
using the {\tto chifile}, and the {\ttg mkimages.sh} script.


\chapter{mkimages.sh}\label{ch:grid}

The {\ttg grid} subdirectory contains the script {\ttg mkimages.sh},
which generates images of the CMDs from the grid box populations in 
the {\tto chifile}, and generates an HTML document which displays the 
images in a grid.  The HTML document provides a very convenient way to 
examine the quality of the model fit, and to diagnose what may be 
causing a poor fit.

The script requires files named {\ttg grid/gridN.index} which 
contains a description of the grid boxes in each of your CMDs
(each CMD has its own index file). The columns are: 

\medskip\noindent
{\ttg ix}, {\ttg iy}, {\ttg ibox}

\medskip
\noindent
where ({\ttg ix}, {\ttg iy}) are the synthetic CMD pixel coordinates, 
and {\ttg ibox} is the grid box to which that pixel belongs.  You can 
modify the program {\ttg mkgrid.f} to generate the gridN.index files 
for your CMDs.

Once you have valid {\ttg gridN.index} files, you can use the 
{\ttg mkimages.sh} script to generate the images and HTML page.  
Before running the script, there are several parameters which should 
be adjusted; see the "CUSTOMIZATION" section of the {\ttg mkimages.sh} 
file.

When running the script, you specify a filename prefix for the HTML 
document and images, and the name of the {\tto chifile} on the command 
line:

\medskip\noindent
{\ttg \% ./mkimages.sh c1978 ../output/c1978.chi}

\medskip
The script will generate 4 images per CMD, which will be placed in the 
{\ttg images/} subdirectory.  The images represent the grid box 
populations of the model and of the data; the difference between model 
and data, and the contribution of each grid box to the fitting 
statistic.  You can view all of the images using the HTML document 
that is created by the script.


\begin{thebibliography}{75}

\bibitem[Bertelli et~al. (1994)]{ber94}
{Bertelli}, G., {Bressan}, A., {Chiosi}, C., {Fagotto}, F., \& {Nasi}, E. 1994,
  \aaps, 106, 275

\bibitem[Girardi et~al. (2000)]{gir00}
{Girardi}, L., {Bressan}, A., {Bertelli}, G., \& {Chiosi}, C. 2000,
  \aaps, 141, 371
  
\end{thebibliography}
\end{document}


\chapter{Introduction}\label{ch:intro}

\section{Overview}

The {\bf StarFISH} package is a suite of FORTRAN programs designed to 
determine the best-fit star formation history (SFH) of a stellar 
population.  The package constructs a library of synthetic 
color-magnitude diagrams (CMDs) based on theoretical isochrones and 
data-derived determinations of the interstellar extinction, 
photometric errors, and distance modulus.  These synthetic CMDs are 
combined linearly and statistically compared to observed photometry.  
When the best-fitting model is found, the amplitude coefficients on 
each synthetic CMD describe the star formation history of the observed 
population.

\medskip
\noindent
If you have any comments, corrections, suggestions, or complaints,
please email: \\
{\ttb jharris@as.arizona.edu}

\bigskip
\noindent
To use the package, you need:

\begin{itemize}
\item {\bf Multicolor photometry} of a stellar population. \\

\item A (preferably empirical) {\bf photometric error model}, 
typically derived from artificial star tests. \\

\item A {\bf library of isochrones} that covers relevant ranges in age 
and metallicity. \\

\item {\it a priori} determination of the {\bf interstellar 
extinction} and {\bf distance modulus}. \\

\item Estimates (or guesses) of the {\bf binary fraction} and {\bf IMF 
slope}. \\
\end{itemize}

\clearpage

The package consists of three main programs, and five support
programs: 

\begin{quote}
\bigskip\noindent
$\bullet$ {\ttg mklib}: constructs a library of isochrone 
photometry from raw isochrones.

\medskip\noindent
$\bullet$ {\ttg synth}: constructs synthetic CMDs 
from the isochrones, applying the effects of interstellar extinction, 
binarity, IMF, and photometric errors as appropriate for the data.

\medskip\noindent
$\bullet$ {\ttg sfh}:   performs a 
maximum-likelihood comparison of composite model CMDs and input data 
CMDs.  The best-fit star formation history is output. 


\bigskip\noindent
- {\ttg interp}:  interpolates between adjacent 
isochrones, to improve resolution in age and/or metallicity.  Requires 
knowledge of equivalent evolutionary points (EEPs) along each 
isochrone. 
   
\medskip\noindent
- {\ttg testpop}: constructs artificial stellar 
populations from the isochrones.  {\ttg testpop} 
incorporates the photometric errors, extinction, etc. exactly as is 
done in {\ttg synth}.  These populations can be 
input to {\ttg sfh} for testing purposes.  
({\ttg testpop} is in the subdirectory of the same 
name)

\medskip\noindent
- {\ttg repop}: same as {\ttg testpop}, except used to construct an 
artificial population from the SFH amplitudes found by {\ttg sfh}.  
This is useful for model/data comparisons. ({\ttg repop} is in the 
subdirectory of the same name)

\medskip\noindent
- {\ttg testchi}: allows one to examine the parameter space of the 
fit by hand.  SF amplitudes and other parameters can be interactively 
adjusted, and the resulting fit recomputed on the fly. ({\ttg testchi} 
is in the subdirectory of the same name) 

\medskip\noindent
- {\ttg mkimages.sh}: creates CMD images from the best-fit model.  It 
also creates CMD maps of each region's contribution to the fitting 
statistic. 
\end{quote}

\clearpage

\section{Getting Started}

To begin, you need to obtain the program code from the Magellanic 
Clouds Photometric Survey website: \\
{\ttb http://ngala.as.arizona.edu/mcsurvey/}

\medskip
As of this writing, the current version is 1.1.  Note that there is a 
full version and a compact version.  The full version includes a 
sample pre-generated synthetic CMD library for testing purposes.  If 
you download the compact version, you can still generate the sample 
synthCMD library, but it will take some time.

\medskip 
The following command will unpack the directory tree of the StarFISH 
distribution. It will create a {\ttg SFH-1.1} subdirectory in the 
current directory.

\medskip\noindent
{\ttg \% tar zxvf SFH-1.1.tar.gz} \hfill (or {\ttg tar zxvf SFH-1.1compact.tar.gz}) \\

\noindent or (on Solaris): \\
\medskip
\noindent
{\ttg \% gunzip SFH-1.1.tar.gz} \\
{\ttg \% tar xvf SFH-1.1.tar}

\medskip
The package contents are shown in {\bf Table~1}.

\bigskip
\noindent{\large {\bf Table 1: Package Contents}}\\
{\scriptsize
\begin{tabular}{|l|l|}
\hline
{\bf File/Directory Name}  & {\bf Description} \\ 
\hline
{\ttg manual/}     & The StarFISH manual (this document)\\
{\ttg libcode/}    & source code and Makefile for the {\ttg mklib} program\\
{\ttg synthcode/}  & source code and Makefile for the {\ttg synth} program\\
{\ttg sfhcode/}    & source code and Makefile for the {\ttg sfh} program\\
{\ttg commoncode/} & source code used by multiple programs\\
{\ttg interpcode/} & source code and Makefile for the {\ttg geteep} and {\ttg interp} programs\\
{\ttg input/}      & input data files for {\ttg mklib}, {\ttg synth} and {\ttg sfh}\\
\hspace{20pt}{\ttg /lib.dat}   & sample {\ttg mklib} parameter file\\
\hspace{20pt}{\ttg /synth.dat} & sample {\ttg synth} parameter file\\
\hspace{20pt}{\ttg /sfh.dat}   & sample {\ttg sfh} parameter file\\
\hspace{20pt}{\ttg /iso.dat}   & sample isochrone description files\\
\hspace{20pt}{\ttg /iso.lock}  & describes how isochrones are to be locked together\\
\hspace{20pt}{\ttg /test.cmds} & sample synthetic CMD description file\\
\hspace{20pt}{\ttg /mask.none} & file indicating CMD regions to ignore in the fit (default: null mask)\\
\hspace{20pt}{\ttg /test.hold} & file indicating which SFH amplitudes should be held fixed (default: none held)\\
\hspace{20pt}{\ttg /av/*.av}   & sample extinction data files\\
\hspace{20pt}{\ttg /crowd/*.crowd} & sample crowding table from artificial star tests\\
\hspace{20pt}{\ttg /dtime.dat} & file describing age interval covered by each isochrone\\
\hspace{20pt}{\ttg /geteep.dat} & input file for {\ttg geteep} \\
\hspace{20pt}{\ttg /interp.dat} & input file for {\ttg interp} \\
\hspace{20pt}{\ttg /testchi.dat} & input file for {\ttg testchi} \\
{\ttg testpop/} & files related to the {\ttg testpop} support program\\
{\ttg testchicode/} & files related to the {\ttg testchi} support program\\
{\ttg repop/}   & files related to the {\ttg repop} support program\\
{\ttg grid/}    & files related to the {\ttg grid} support scripts\\
{\ttg iso/}     & raw isochrone photometry from Girardi et~al. (2002, 2000) 
                  and Bertelli et~al. (1994)\\
{\ttg testlib/}\footnotemark[1]  & sample isochrone library produced by {\ttg mklib}\\
{\ttg test.syn/}\footnotemark[1] & sample synthetic CMDs based on artificial star tests
                                   from an image of NGC~1978.\\
{\ttg data/}   & input data photometry goes here (NGC~1978 photometry included)\\
{\ttg output/} & {\ttg sfh} results and the {\ttg sfh.sm} plotting script\\ \hline
\end{tabular}
}

\footnotetext[1]{Not present in the compact version of the program}

\clearpage

\section{Compiling the Principal Code}

Each principal program ({\ttg mklib}, {\ttg synth}, and {\ttg sfh}) 
has its own Makefile, in its source code directory.  To compile these 
programs, simply cd to the correct directory and type {\ttg make}:

\medskip
\noindent
{\ttg \% cd /path/to/SFH-1.1} \hfill (if not already in the SFH directory) \\
{\ttg \% cd libcode} \\
{\ttg \% make} \\
{\ttg \% cd ../synthcode} \\
{\ttg \% make} \\
{\ttg \% cd ../sfhcode} \\
{\ttg \% make}

\medskip
Each program will be copied into the {\ttg SFH-1.1} directory, which 
is where you should run them.  You are now ready to experiment with 
the code.

\vspace{2em}
\section{A Quick Run}

We will use all of the default parameter values to go from a freshly 
downloaded and compiled distribution to a successful first run of 
{\ttg sfh}.  Steps labeled with an asterisk (*) may be skipped if you 
downloaded the full version of StarFISH, because this version already 
contains the pregenerated isochrone library and synthetic CMD library.

\begin{enumerate}

\item {\ttg \% cd /path/to/SFH-1.1} \hfill (if not already in the SFH directory)

\item (*) Prepare isochrones:  (see Section~\ref{sec:iso})\\
{\ttg \% cd iso} \\
{\ttg \% tar zxvf girardi.tar.gz} \hfill (unpack the isochrones) \\
{\ttg \% gawk -f parse\_iso.awk isocz*.dat} \hfill (parse isochrones) 

\item (*) Generate isochrone library: (see Chapter~\ref{ch:lib})\\
{\ttg \% cd ..} \\
{\ttg \% mkdir testlib} \hfill (create target directory before running {\ttg mklib}!)\\
{\ttg \% ./mklib < input/lib.dat} 

\item (*) Generate Synthetic CMD library: (see Chapter~\ref{ch:synth}) \\
{\ttg \% mkdir test.syn} \hfill (create target directory before running {\ttg synth}!)\\
{\ttg \% ./synth < input/synth.dat} 

\item Perform SFH Solution: (see Chapter~\ref{ch:sfh}) \\
{\ttg \% ./sfh < input/sfh.dat} 

\item Plot solution: \\
{\ttg \% cd output} \\
{\ttg \% sm} \hfill (start supermongo plotting program) \\
{\ttg sm$>$ macro read sfh.sm } \hfill (load sm script) \\
%{\tt sm$>$ dev x11} \hfill (bring up display window) \\
{\ttg sm$>$ sfh c1978 c1978} \hfill (display SFH solution)
\end{enumerate}

\clearpage

\section{What's New in Version 1.1}

There have been several improvements made since 1.0; the most 
important changes are listed here.  See the files {\ttg ChangeLog} and 
{\ttg NEWSTUFF} for more details.

\begin{itemize}
\item Fixed random-number generator.  We found a periodicity in 
the previous one. \\

\item More robust handling of input files.  The programs are now 
much less picky about the format of input files ({\itr note that 
input files from previous versions of StarFISH are not compatible 
with this version}). \\

\item Input files can now contain comment lines.  Any line in an 
input file that begins with {\ttg \#} or {\ttg *} will be ignored by 
the parser. \\

\item Filenames in the input files can now be up to 40 characters 
long (previous maximum was 24 characters) \\

\item Option to use Dolphin's ``Poisson'' fit statistic instead of 
the chi-squared statistic. \\

\item The photometric error model used in {\ttg synth}, 
{\ttg testpop}, and {\ttg repop} now properly includes covariance 
between the color and magnitude errors. \\

\item Option to use an analytic error model instead of an 
AST-derived crowding table in {\ttg synth}, {\ttg testpop}, and 
{\ttg repop}. \\

\item {\ttg sfh} writes the current best solution to the logfile more 
frequently. \\

\item New empirical isochrone interpolation code ({\ttg geteep} and 
{\ttg interp})

\end{itemize}


\chapter{mklib}\label{ch:lib}

{\ttg mklib} is used for processing raw isochrone photometry into a 
format usable by {\ttg synth}.  {\ttg mklib} applies a distance modulus, 
interpolates along the isochrone to a uniformly high resolution, and 
computes occupation probabilities (OP) for each isochrone point.

The package currently ships with the Padua isochrones, most recently
published by Girardi et~al. (2002).  We found that the main sequence 
in these isochrones is too coarsely sampled for our purposes, 
resulting in artificially "lumpy" synthetic CMDs. Therefore,  
{\ttg mklib} includes a photometric interpolation routine for points 
fainter than the main sequence turn-off (MSTO; defined for each 
isochrone in the {\tto isofile}).  If your isochrones do not require 
this interpolation, simply set the MSTO magnitude for each isochrone 
to be fainter than any isochrone point.  

\bigskip
The {\ttg mklib} input file (named {\ttg input/lib.dat} by default) 
contains the parameters listed in Table~2 (the lines 
must be present in the listed order). \\

\noindent{\large {\bf Table 2: mklib input file parameters}} \\
{\scriptsize
\begin{tabular}{|lr|l|}
\hline
{\tto isofile} &[string]& description of each isochrone: the input filename, \\
               &        & the output filename, the age, and the MSTO magnitude. \\
{\tto faint}   & [real] & the faint magnitude limit for the output isochrone library.\\
               &        & Should be several mag fainter than the data's faint limit. \\
{\tto dmag}    & [real] & photometric distance between adjacent interpolated points.\\
{\tto dmod}    & [real] & distance modulus \\
{\tto gamma}   & [real] & logarithmic IMF slope (Salpeter = -1.35) \\
{\tto nmag}    &  [int] & number of magnitude values per isochrone point \\
{\tto mag0}    &  [int] & which magnitude to check against {\bf faint} and  {\bf msto} \\
{\tto iverb}   &  [int] & verbosity flag (0=silent, 1=screen messages, 2=extra output files)  \\
\hline
\end{tabular}
}

\clearpage

\noindent
To use {\ttg mklib}:

\begin{enumerate}
\item {\bf Create an isofile that describes your isochrones.}  This 
file will also be used by {\ttg synth}.  Each line in the file should 
contain the following space-delimited fields (the column format does 
not matter):

\medskip\noindent
{\tto log(age)}~[real number], 
{\tto input raw isochrone file}~[up to 40 chars], 
{\tto output isochrone file}~[up to 40 chars], 
{\tto msto\_mag}~[real number]

\medskip
{\tto msto\_mag} is the absolute magnitude of each isochrone's MSTO point.  
If you do not know the MSTOs for your isochrones, you'll need to identify 
them.  Use the fact that at the MSTO, the occupation probability changes 
dramatically.
\medskip

\item {\bf Preprocess the isochrone photometry files}.  Each isochrone 
must be in its own file with a name matching the isochrone's entry in 
the {\tto input raw isochrone file} list (column 2 of the isofile).  
The columns required for the isochrone files are:

\medskip\noindent
{\tto mass}, {\tto mag1}, {\tto mag2}, {\tto mag3}, ..., {\tto magN}
 
\medskip
where N is the equal to the {\tto nmag} parameter in {\ttg lib.dat}.  
Column format does not matter.  The magnitudes should be expressed as 
absolute magnitudes; {\ttg mklib} will apply the distance modulus 
specified in {\ttg lib.dat} and output apparent magnitudes.  For the 
default Padua isochrones, the input isochrone files are generated by 
the {\ttg parse\_iso.awk} script. See the next section for details on 
the default isochrone-processing.  Make sure the {\tto mag0} parameter 
indicates the pass-band which should be used for comparing to the 
{\tto faint} and {\tto msto\_mag} values.  

\medskip
\item {\bf Create the target output directory, if necessary}. The 
output directory is specfied as part of the output filenames in the 
{\tto isofile}.

\medskip
\item {\bf Run {\ttg mklib}:} \\
{\ttg \% ./mklib < input/lib.dat}

\medskip
Depending on the verbosity flag, {\ttg mklib} may provide screen 
messages regarding which isochrone it is working on.  The output 
directory will contain the processed isochrones, with the following 
columns:

\medskip\noindent
{\tto mag1}, {\tto mag2}, {\tto mag3}, ..., {\tto magN}, {\tto OP}, 
{\tto mass}

\medskip
where the {\tto magN} are the apparent magnitudes in each band, 
{\tto OP} is the relative occupation probability associated with each 
point (determined from the IMF and from the relative duration of each 
evolutionary stage), and {\tto mass} is the {\bf initial} mass of each 
isochrone point, not including any mass-loss that may have occured 
since the star was formed.
\end{enumerate}

\clearpage

\section{Pre-processing the Padua Isochrones}\label{sec:iso}

The {\ttg iso/} directory contains files related to the Padua 
isochrones, which we ship with StarFISH for convenience.  You are free 
to use any other isochrone set that you wish, as long as it can be 
placed in the format that {\ttg mklib} requires (separate files for 
each isochrone, with the column format specified in step 2 above).

The Padua isochrones are disributed as a tar archive, which we include 
in StarFISH as the file {\ttg iso/girardi.tar.gz}.  The archive contents 
are listed in Table~3.

\bigskip
\noindent{\large {\bf Table 3: Padua isochrone package contents}} \\
\begin{tabular}{|l|l|}
\hline 
{\ttg isocz0[001,004,01,04,08,19,30].dat} &  Girardi et~al. isochrones\\
{\ttg isocsummz0[001,004,01,04,08,19,30].dat} &  Tables of Equivalent 
    Evolutionary \\
    & Points (EEPs) \\ 
{\ttg read.me} &  Girardi's readme file\\ 
\hline
\end{tabular}

\vspace{3em}
Each of the isochrone files {\ttg isocz0*.dat} contains isochrones of 
all ages for a given metallicity.  StarFISH needs a separate file for 
each isochrone, so we provide the AWK script {\ttg parse\_iso.awk} to 
divide each raw isochrone collection into individual files.  We also 
provide the script {\ttg parse\_summ.awk} to divide the EEP table 
collections into individual EEP tables for each isochrone.

If you downloaded the full version of StarFISH, then you do not need 
to run the parsing scripts; you already have the processed isochrone 
files, named {\ttg iso/zNNNN\_tt.tt} (where {\ttg NNNN} is a 4-digit 
code for the metallicity, such as {\ttg 0080}; and {\ttg tt.tt} is the 
log of the age, such as {\ttg 10.00}).  If you do not have these 
files, generate them like this:

\medskip\noindent
{\ttg \% cd iso}\\
{\ttg \% tar zxvf girardi.tar.gz} \\
{\ttg \% gawk -f parse\_iso.awk isocz*.dat} \\
{\ttg \% gawk -f parse\_summ.awk isocsummz*.dat} \\

\medskip
If you are satisfied with the age and metallicity resolution provided 
by the default Padua isochrones, then you are all done here.  If you 
want to try to interpolate between isochrones to increase age or 
metallicity resolution, refer to the chapter discussing the 
{\ttg interp} support program.


\chapter{synth}\label{ch:synth}

{\ttg synth} constructs a synthetic CMD library from the isochrone 
library, incorporating the photometric effects of extinction, 
binarity, the IMF, and photometric errors.  Each synthetic CMD is 
represented as a list of ``pixel values'', equal to the relative number 
of stars found at the corresponding subregion in the CMD.  If the data 
contain more than two filters, then multiple CMDs can be created for 
each isochrone, so that all of the data can be utilized in the fit.  
For example, if your data include $U$, $B$, and $V$ photometry, you 
might use {\ttg synth} to create two CMDs per isochrone: one with 
$U-B$ vs. $B$ axes, and another with $B-V$ vs. $V$ axes.

Some of the user-provided parameters (extinction, photometric errors, 
and binarity) are first encountered in the {\ttg synth} program. 
Others (the IMF and distance modulus) have already been used to 
create the isochrones output by {\ttg mklib}.  Therefore, if you need 
to change the IMF or distance modulus you must run {\ttg mklib} first. 

Depending on the age and metallicity resolution of your isochrone 
library, and on the quality of your photometric data, it is likely
that some of the isochrones are photometrically degenerate.  If 
{\ttg synth} is allowed to produce degenerate synthetic CMDs, then the 
SFH solution determined by {\ttg sfh} will be ambiguous, and will 
suffer from large correlated uncertainties among the SFH amplitudes.  
To avoid ambiguous SFH fits, {\ttg synth} can combine 
photometrically-degenerate isochrones together into the same synthetic 
CMD.  This is referred to as {\bf locking} the isochrones. 

\clearpage

\noindent The synth input file ({\ttg input/synth.dat}) contains the 
parameters listed in Table~4. {\itr Note that synth input files from 
previous versions of StarFISH will no longer work}.

\bigskip
\noindent{\large {\bf Table 4: synth input file parameters}} \\
{\scriptsize
\begin{tabular}{|lr|l|}
\hline
\multicolumn{3}{|c|}{\bf Filenames} \\ 
\hline
{\tto isofile}  &[string]& description of the isochrones (can use the file from {\ttg lib.dat}) \\
{\tto lockfile} &[string]& file describing the combination of degenerate isochrones\\
{\tto hotfile}  &[string]& list of extinction measurements of hot (young) stars\\
{\tto coldfile} &[string]& list of extinction measurements of cool (old) stars\\
{\tto crowd1}   &[string]& crowding table from artificial star test\\
{\tto crowd2}   &[string]& output file for lookup table of binned delta-magnitudes \\ 
\hline
\multicolumn{3}{|c|}{\bf CMDs} \\ 
\hline
{\tto nmag} &  [int] & same as nmag in {\ttg lib.dat}\\
{\tto ncmd} &  [int] & number of CMDs to construct per isochrone group\\
{\tto mag0} &  [int] & same as mag0 in {\ttg lib.dat}\\ 
{\tto dpix} & [real] & size of CMD pixels (in magnitudes) {\itr[NEW]}\\
\hline
\multicolumn{3}{|c|}{\bf parameters for each of the CMDs} \\ 
\hline
{\tto xeq}    &[string]& filter equation for the x-dimension\\
{\tto yeq}    &[string]& filter equation for the y-dimension\\
{\tto xmin}   & [real] & smallest allowable x-value\\
{\tto xmax}   & [real] & largest allowable x-value\\
{\tto ymin}   & [real] & smallest allowable y-value\\
{\tto ycmax}  & [real] & largest allowable y-value, for crowding table stars\\
{\tto ypmax}  & [real] & largest allowable y-value, for data/model stars\\
{\tto suffix} &[string]& filename suffix for the synthetic CMD output files\\ 
\hline
\multicolumn{3}{|c|}{\bf Crowding} \\ 
\hline
{\tto dbinx} & [real] & width of crowding bins, in magnitudes\\
{\tto dbiny} & [real] & height of crowding bins, in magnitudes\\
{\tto emin}  & [real] & minimum delta-magnitude cutoff\\
{\tto emax}  & [real] & maximum delta-magnitude cutoff\\
{\tto dbin}  & [real] & binsize of delta-magnitude histograms\\ 
\hline
\multicolumn{3}{|c|}{\bf Reddening parameters} \\ 
\hline
{\tto red(i)} & [real] & extinction in mag(i) (relative to the filter in which extinction is measured) \\ 
\hline
\multicolumn{3}{|c|}{\bf Run-time parameters} \\ 
\hline
{\tto verb}        &  [int] & verbosity flag (higher N = more output)\\
{\tto interp\_err} & [flag]\footnotemark[2] & 1 = interpolate errors between bracketing crowding bins {\itr[NEW]}\\
{\tto fake\_errs}  & [flag]\footnotemark[2] & 1 = use analytic error model instead of crowding table {\itr[NEW]}\\
{\tto nscale}      &  [int] & number of model stars in each CMD\\
{\tto seed}        &  [int] & random seed value\\
{\tto mass1}       & [real] & minimum mass (for OP normalization)\\
{\tto mass2}       & [real] & maximum mass (for OP normalization)\\
{\tto gamma}       & [real] & logarithmic IMF slope (Salpeter = -1.35)\\
{\tto faint}       & [real] & faint limit for generated model stars\\
{\tto fbinary}     & [real] & binary fraction\\ 
\hline
\end{tabular}
}
\footnotetext[2]{A flag is an int whose value can either be 0 or 1.}

\clearpage
\noindent To use {\ttg synth}:

\begin{enumerate}
\item {\bf Construct the {\tto lockfile}.}  This file describes how the 
isochrones are to be locked together into groups.  Again, this is 
necessary if isochrones of similar age are photometrically degenerate, 
to avoid ambiguous SFH fits.  The columns of the {\tto lockfile} are as 
follows ({\itr note that lockfiles from previous versions of StarFISH 
will no longer work}):

\medskip
\noindent
{\tto group\_id}[int], 
{\tto isoname}[up to 40 characters], 
{\tto synthfilestem} [up to 40 characters]

\medskip
{\tto group\_id} identifies the locked group to which the isochrone 
belongs.  The actual {\tto group\_id} values do not matter; when the 
value of one line is different from that of the previous line, a new 
synthetic CMD is started.

{\tto isoname} is the isochrone filename.  It is only used to 
check against the filename in the {\tto isofile}.  If they don't 
match, an error is triggered; there must be a one-to-one 
correspondence of the isochrones in the {\tto isofile} and the 
{\tto lockfile}.

{\tto synthfilestem} is the filename stem for the output synthetic 
CMD pixel file.  The output filenames are constructed by concatenating 
the CMD suffixes specified in {\ttg synth.dat} onto 
{\tto synthfilestem} (these suffixes are typically the combination of 
magnitudes; for example a $U-B$ vs. $B$ CMD might have a suffix of 
``{\ttg .ub}'').

\item {\bf Construct the extinction files ({\tto hotfile}, 
{\tto coldfile}).}  The extinction files each contain a single column: 
extinction values, $A_{\lambda}$, measured in the magnitude whose 
reddening parameter in {\ttg synth.dat} is set to 1.00 (in the 
provided {\ttg synth.dat}, red(3) is set to 1.00; this is the $V$ 
band, so the extinction files list $A_V$ values).  There are two files 
({\tto hotfile} and {\tto coldfile}) to account for the different 
extinction properties of young and old stellar populations.  If you do 
not require population-dependent extinction, simply point 
{\tto hotfile} and {\tto coldfile} to the same file.  If you do not 
require extinction modeling at all, point {\tto hotfile} and 
{\tto coldfile} to the null extinction file, {\ttg input/av/zero.av}.
   
\item {\bf Construct the crowding table ({\tto crowd1})}. This is most 
likely constructed from artificial star tests.  The crowding table 
contains:

\medskip   
{\tto ra}, {\tto dec}, {\tto mag1}, {\tto dmag1}, {\tto mag2}, {\tto dmag2}, 
..., {\tto magN}, {\tto dmagN}

\medskip\noindent
column format doesn't matter.  The crowding table should contain all 
artificial stars injected into your data images, even if they were not 
detected in your photometric pipeline.  In fact, these non-recovered 
``dropout'' stars are extremely important, because {\ttg synth} uses 
them to determine the completeness rate as a function of photometric 
position in the CMDs.  For bands in which an artificial star was 
detected, the listed {\tto mag} value is its {\bf recovered} 
magnitude, and {\tto dmag} is (mag(recovered) - mag(input)).  For 
bands in which an artificial star dropped out, {\tto mag} is the 
star's {\bf input} magnitude, and {\tto dmag} is set to 9.99 in order 
to flag it as a dropout in that band.  

If you decide not to perform artificial star tests on your images 
(although such tests are {\it highly} recommended), you can either 
generate a fake crowding table by hand, based on your estimates of the 
scatter and completeness rates in each band, or you can set the 
{\tto fake\_errs} parameter in {\ttg synth.dat} to 1, which will use a
hueristic analytic error model instead of the empirical crowding 
table.  In this case, you will need to modify the code in 
{\ttg synthcode/fakecrowd.f} to generate photometric errors and 
completeness rates appropriate for your data.  The comments in that 
file will guide you in making these changes.

\item Decide how many CMDs you would like, and what their axes will 
be.  You specify the CMD axes with the {\tto xeq(icmd)} and 
{\tto yeq(icmd)} string variables in {\ttg synth.dat}.  These strings 
indicate the filter combination for each CMD axis.  For example, you 
have $VRI$ photometry, and you are using $V$=1, $R$=2, $I$=3.  Your 
first CMD could be: $V-I$ vs. $V$, so {\tto xeq(1)} = 1-3, 
{\tto yeq(1)} = 1.  Your other ``CMD" could be a two-color diagram: 
$V-R$ vs. $V-I$, so {\tto xeq(2)} = 1-2, {\tto yeq(2)} = 1-3.  The 
strings can be either the sum or the difference of two filter indices, 
or one index by itself.  Do not put whitespace in the equations; each 
must be a single ``word''.

You will also need to specify photometric limits for each CMD 
(parameters {\tto xmin}, {\tto xmax}, {\tto ymin}, {\tto ycmax}, and 
{\tto ypmax}).  Note that instead of one {\tt ymax}, there are two 
variables: {\tto ycmax} and {\tto ypmax}.  {\tto ycmax} is the faint 
limit for the crowding bins, and {\tto ypmax} is the faint limit for 
the CMD itself.  {\tto ycmax} should be at least 2 or 3 mag fainter 
than {\tto ypmax}, because very faint stars can be blended and 
detected at much brighter magnitudes than their true magnitude.  

Also specify the size of CMD ``pixels'' in the output synthetic CMD 
files ({\tto dpix}), in magnitudes.

Finally, specify a suffix string that will be concatenated with 
{\tto synthfilestem} to construct the output file name for each 
synthetic CMD.  The suffices can be up to 8 characters long, and 
typically indicate something about the CMD axes. For example, a
$B-V$ vs. $V$ CMD might have a suffix ``{\ttg .bv}''.

\item Specify the size of the crowding bins in each direction 
({\tto dbinx} and {\tto dbiny}).  Artificial stars in the crowding 
table are binned in the CMDs so that the brightness- and 
color-dependence of the photometric errors can be accounted for.
Independent $\Delta mag$ histograms will be constructed from the 
artificial stars present in each bin.  

Specify the limits for the delta-mag histograms with {\tto emin} and 
{\tto emax}.  Any $\Delta mag$ value in the crowding table outside 
these limits is treated as a photometric dropout.

The {\tto dbin} parameter sets the width of the bins in the delta-mag 
histograms ({\itr this parameter used to determine the size of the 
synthetic CMD pixels as well, but there is now a separate parameter 
for that: dpix}).

\item Specify the reddening law to use, by setting the relative 
extinction for each band: {\tto red(imag)}.  The band for which you 
have measured extinction values in {\tto hotfile} and {\tto coldfile} 
should have {\tto red(imag)} = 1.0.  The rest of the values should be 
the extinction in that band, assuming the extinction in the reference 
band is 1.0.  For example, you have $BV$ photometry, and measured 
A$_V$ extinction values.  A$_V$ = 1.0.  If we assume that 
A$_V$/E(B-V) = 3.2 = A$_V$/(A$_B$ - A$_V$), then A$_B$ = 1.3125. 

\item Specify miscellaneous parameters.  {\tto nscale} sets the number 
of model stars to include in each CMD.  {\tto nscale} should be much 
larger than the number of stars in your data CMDs, so that the Poisson 
noise in the models is much smaller than in the data.  {\tto iseed} is 
the random seed value; it can be any integer.  {\tto mass1}, 
{\tto mass2}, and {\tto gamma} are IMF parameters used for normalizing 
the pixel values.  {\ttg synth} will only attempt to add model stars 
brighter than the {\tto faint} parameter.  This is an efficiency 
feature: without this cutoff, the code would waste a lot of time 
adding undetectable stars.  Set {\tto faint} to just fainter than 
where your completeness rate reaches 0\%.

\item Once the parameter file is complete, and the other input files 
are ready, you can finally run {\ttg synth}.  It will take quite a bit 
of time to complete, depending on computing power and the number of 
isochrones.  I found it usually took several hours to process 150 
isochrones on a 700 MHz linux box.  
\end{enumerate}

You can examine the synthetic CMDs as images, using either the IRAF 
{\tt rtext} command, or the {\ttg mkgif.bat} and {\ttg pxl2ppm.awk} 
scripts I provide in the {\ttg commoncode} directory.


\chapter{sfh}\label{ch:sfh}

{\ttg sfh} performs a ``chi-squared-like'' minimization comparison 
between data photometry and a linear combination of the synthetic 
CMDs.  Each synthetic CMD is modulated by an {\bf amplitude} that is 
proportional to the number of stars with the age and metallicity of 
the synthetic CMD's parent isochrone(s).  These amplitudes form the 
multidimensional parameter space of the minimization.

I say ``chi-squared-like'' because you may actually choose among three 
statistics to determine the quality of the fit: chi-squared, 
Lorentzian, or Poisson.  The three parameters weight outliers slightly 
differently, but they are otherwise quite similar.  Hereafter, I will 
simply refer to the fitting statistic as ``chi-squared''.

The minimization analysis employs the downhill simplex ``amoeba'' 
algorithm as described in Numerical Recipes (Press {\it et al.}; 
hereafter NR).  We have written our own implementation of this 
algorithm.  It works by sampling a small region of the parameter  
space and determining the local chi-squared gradient.  It then takes a 
small step in the down-gradient direction, and determines the local 
chi-squared gradient around this new location.  The process iterates 
until a (nearly) zero gradient is found, signalling that a minimum
has been found.  

The amoeba algorithm in NR has some safeguards against finding local 
minima, but we found that it was still prone to getting stuck.  The 
simplex by definition can only sample parameter space along directions 
parallel to the N axes of the parameter space; it is blind to off-axis 
directions.  However, as it nears the minimum, lower chi-squared 
values will tend to be found when a change in one parameter is 
complemented by a change in one or more other parameters.  In other 
words, downward gradients will tend to lie along off-axis directions.

To remedy this problem with the pure simplex method, we add a 
random-search loop, which is triggered whenever the simplex claims it 
has found a minimum.  In the random-search loop, we select a random 
direction in the multidimensional parameter space, and take a small 
step along that direction.  If the step yields a smaller chi-squared
value, we continue stepping along that direction until we reach a step 
that no longer decreases chi-squared.  This is repeated for a large 
number of random directions, and the parameter space location which 
yielded the lowest overall chi-squared value is recorded.  The simplex 
is then restarted at this new parameter space location.

This iteration between the downhill simplex algorithm and the 
random-search loop is repeated until the random-search loop cannot 
find a direction which yields a lower chi-squared value.
We then accept the last simplex location as the true minimum, and 
output the best-fit SFH amplitudes.

\bigskip
\noindent The {\ttg sfh} input file ({\ttg input/sfh.dat}) contains 
the following parameters:

\medskip
\noindent
{\scriptsize
\begin{tabular}{|lr|l|} 
\hline
\multicolumn{3}{|c|}{\bf Filenames} \\ 
\hline
{\tto datpre}   &[string]& input data file prefix\\ 
{\tto cmdfile}  &[string]& Sample synthetic CMDs description file\\
{\tto maskfile} &[string]& Allows one to ignore specified regions of the CMDs\\
{\tto holdfile} &[string]& file w/ SFH amplitudes that should be held fixed\\
{\tto outfile}  &[string]& contains final best-fit SFH amplitudes\\
{\tto logfile}  &[string]& file w/ current best SFH (useful for monitoring, and for\\
                && restarting an interrupted run)\\
{\tto plgfile}  &[string]& log of all tested parameter locations \\
                && (this file will be very large; not used by default: {\tto iplg}=0)\\
{\tto chifile}  &[string]& output file containing grid populations and chi-squared values\\
\hline
\multicolumn{3}{|c|}{\bf Synthetic CMD parameters} \\ 
\hline
{\tto nfree}  & [int]& number of independent isochrone amplitudes\\
{\tto ncmd}   & [int]& number of CMDs over which to calculate chi-squared\\
{\tto npix}   & [int]& binning factor for CMDs\\
{\tto dpix}   &[real]& size of synthetic CMD pixels (same as in {\ttg synth.dat})\\
\hline
\multicolumn{3}{|c|}{\bf Parameters for each CMD} \\ 
\hline
{\tto suffix(i)} &[string]& CMD filename suffix (same as in {\ttg synth.dat})\\
{\tto   xmin(i)} & [real] & calculate chi-squared within these CMD limits\\
{\tto   xmax(i)} & [real] & calculate chi-squared within these CMD limits\\
{\tto   ymin(i)} & [real] & calculate chi-squared within these CMD limits\\
{\tto   ymax(i)} & [real] & calculate chi-squared within these CMD limits\\
{\tto   nbox(i)} &  [int] & number of grid boxes within region defined by above limits\\ 
\hline
\multicolumn{3}{|c|}{\bf Runtime parameters} \\ 
\hline
{\tto seed}      &  [int]&  random seed value \\
{\tto fit\_stat} &  [int]&  0 = use chi-squared; 1 = use Lorentzian; 2 = Use Poisson\\
{\tto uselog}    & [flag]&  1 = start searching at logged position in logfile\\
{\tto iplg}      & [flag]&  1 = generate plgfile (warning: this file will be very large!)\\
{\tto gtype}     & [flag]&  grid definition flag (0 = uniform grid; 1 = custom grid)\\
{\tto iverb}     &  [int]& verbosity flag (larger N = more screen output)\\
{\tto lambda}    & [real]& initial size of simplex\\
{\tto conf}      & [real]& confidence interval for computing errorbars (e.g., 1 sigma = 0.68)\\
{\tto cthr}      & [real]& if ($\Delta\chi^2$ $>$ cthr), consider new location an improvement\\
{\tto ptol}      & [real]& if ($|p\_new - p\_old| > ptol$), consider new location different\\
{\tto ftol}      & [real]& if simplex's chi-squared values span less than ftol, signal a zero \\
                 &       & gradient (i.e., a minimum) \\
{\tto nvec}      &  [int]& number of parameter space directions in random-search loop\\
{\tto ntry}      &  [int]& number of iterations for determining errorbars\\
\hline
\end{tabular}
}

\clearpage

\noindent
To use {\ttg sfh}:

\begin{enumerate}
\item {\bf construct the {\tto cmdfile}}.  The file's columns are:

\medskip\noindent
{\tto z\_metal}, {\tto log\_age}, {\tto filenamestem}

\medskip\noindent
Column format does not matter.  Note that some lines in the
{\tto cmdfile} may represent multiple isochrones.  In these cases, 
just choose a representative age/metallicity for the group.  These 
values are only used to identify the amplitudes in the output file.  
The {\tto filenamestem} needs to point to the synthetic CMD files 
generated with {\ttg synth}.

\item {\bf Decide on a CMD gridding strategy}.  {\ttg sfh} works by 
binning the CMDs and comparing the number of data and model stars in 
each bin.  The default is to uniformly bin the CMDs in each dimension 
with a binsize of {\tto npix}$\times${\tto dpix}.  If you want to use 
a custom grid instead, write a subroutine to assign gridbox numbers to 
each CMD pixel, add a call to your subroutine in {\ttg grid.f}, and 
set {\tto gtype} to 1 in {\ttg sfh.dat}.  You must also add your 
custom-grid source code files to {\ttg sfhcode/Makefile}.  An example 
custom grid can be reviewed in the files {\ttg ubbgrid.f}, 
{\ttg bvvgrid.f}, and {\ttg viigrid.f}.  These grids were constructed 
for our LMC data, and are fine in densely populated regions of the 
CMDs and coarse in sparsely populated regions.

The boxes in each CMD's grid are identified by a single number, the
box number.  Thus the distribution of stars in the CMDs can be
expressed as a two-dimensional array whose indices identify the CMD
and the gridbox within that CMD: {\tt nstars(icmd, ibox)}. 
{\tt icmd} refers to which CMD you are considering, {\tt ibox} is the 
number of the box in that CMD.  Indicate the total number of boxes in 
each CMD with the {\tto nbox(icmd)} parameter. 

\item {\bf Construct the {\tto maskfile}}.  the file's columns are:

\medskip\noindent
{\tto icmd},  {\tto ibox},  {\tto maskflag}

\medskip\noindent
Column format does not matter.  if {\tto maskflag}=1, the stars in 
({\tt icmd}, {\tt ibox}) will be excluded from the determination of 
the fit.

\item {\bf  Preprocess the data photometry}.  You need to have a 
separate photometry file for each CMD.  The filenames need to be 
({\tto datpre} + {\tto sffx(icmd)}), where {\tto datpre} and 
{\tto sffx(icmd)} are specified in {\ttg input/sfh.dat}.  The first 
column in each CMD file is the x-dimension magnitude, and the second 
column is the y-dimension magnitude.  Other columns are ignored, but 
could be used for (ra, dec) coordinates or other useful information.   
Note that the {\tto sffx(icmd)} values need to be the same as the 
{\tto suffix(icmd)} values in {\ttg synth.dat}, and that the CMD 
limits ({\tto xmin}, {\tto xmax}, {\tto ymin}, {\tto ymax}) need to be 
the same also.

\item {\bf Adjust the sfh runtime parameters.}  The default values 
should work well, but feel free to tweak them.

\item {\bf Run {\ttg sfh}}.  Depending on the value of {\tto iverb}, 
it may provide screen output on its progress (status messages, 
current lowest chi-squared value).  Generally, larger values of 
{\tto iverb} produces more screen output.  {\tto iverb}=0 will
print no screen messages; {\tto iverb}=3 provides ``live'' updates
of the convergence process.  If the run gets interrupted for 
some reason, you can resume at the last logged position by setting 
{\tto uselog}=1 (assuming the {\tto logfile} exists).

\item {\bf Evaluate the goodness-of-fit.}  When the run is complete, 
it will print out the lowest chi-squared value found.  Divide this by 
({\tto nboxes}-{\tto nfree}) to get the reduced chi-squared, which 
should probably be less than $\sim10$ for a good fit.  

To examine the SFH solution, you can use the {\ttg sfh.sm} supermongo 
script (in the {\ttg output/} subdirectory).  All of the customizable
parameters for this script are collected at the top of the file.  See
the comments in the file for guidance in adjusting these settings.

If there is a problem with the SFH fit (e.g., high chi-squared value), 
you can use the following diagnostics (each of these programs has its 
own Chapter in this manual):

\begin{enumerate}
\item the {\ttg repop} program produces artificial stellar photometry 
based on the best-fit amplitudes, to be compared to the data CMDs. 

\item the {\ttg mkimages.sh} script in the {\ttg grid/} subdirectory, 
which converts the gridbox populations in the {\tto chifile} into 
images, and embeds them into an HTML document for easy comparison.  

\item the {\ttg testchi} program allows you to tweak the SFH 
amplitudes and parameter values.  It then calculates a new chi-squared 
value for the new model.
\end{enumerate}

\end{enumerate}


\chapter{geteep and interp}

{\ttg geteep} and {\ttg interp} are used to interpolate between 
isochrones to increase the nominal matallicity and/or age resolution.
To compile the programs, cd to the {\ttg interpcode} directory and 
type {\ttg make}.  This will create the {\ttg geteep} and 
{\ttg interp} programs in the root {\ttg SFH-1.1} directory.

{\ttg geteep} identifies empirical ``equivalent evolutionary points''
(EEPs) in each isochrone.  True EEPs are stellar evolution events 
which can be identified in the isochrones, such as ``the main sequence 
turn-off'' or ``the tip of the red giant branch''.  These are very 
useful for interpolating between isochrones.  In fact, the older 
Padua isochrones provide EEP tables for this purpose, but the new
isochrones published in 2002 did not have this information (StarFISH 
includes the older EEP tables, but we don't make use of them since 
they are missing for some isochrones).

Since we don't have EEP tables for all of the isochrones, we need a 
way to identify EEPs (or something like them), which is what 
{\ttg geteep} does.  It identifies empirical EEPs (eEEPs) as local 
extrema in color or brightness, along the isochrone track in one of 
the CMDs.  Many of these points correspond to ``true'' EEPs (e.g., the 
TRGB is a local brightness maximum).  The point is to identify points 
which represent the same evolutionary state in two similar isochrones; 
we feel that the color/brightness extrema do this well.

The {\ttg geteep} input file has a two-line header.  The first line 
contains an integer equal to the number of magnitudes representing 
each point in the isochrones.  The second line contains an integer 
identifying the magnitude which will be the vertical ``brightness'' 
axis in the target CMD (in which eEEPs will be identified).  The 
horizontal axis is automatically set to be the color index 
(mag(i-1) - mag(i)), so don't choose ``1'' for the vertical axis.
The remainder of the file lists the isochrone files to be processed,
and the list of output isochrones in which eEEPs have been identified.

Once you have run {\ttg geteep} to identify and label the eEEPs in 
each isochrone, you can run {\ttg interp} to interpolate between the 
isochrones.  The {\ttg interp} input file has three header lines at 
the top which identify the number of magnitudes, the vertical-axis 
magnitude, and the fractional distance between the bracketing 
isochrones for the target interpolated isochrones.  For example, if 
the bracketing isochrones have metallicity 0.004 and 0.008, and the 
target isochrones are to have metallicity 0.005, the fractional 
distance should be 0.25. (This assumes linear variation with 
metallicity, which is probably Ok over small changes in metallicity).

The rest of the {\ttg interp} input file is the list of isochrones to 
be interpolated.  In each row, column 1 is the first bracketing 
isochrone, column 2 is the second bracketing isochrone, and column 3 
is the target isochrone.  Again, there is no strict column format; 
each filename can be up to 40 characters long.


\chapter{testpop}\label{ch:testpop}

{\ttg testpop} constructs artificial stellar photometry based on a 
user-specified star formation history.  It is basically a different 
front-end to the {\ttg synth} code base.  Build {\ttg testpop} by 
typing {\ttg make} in the {\ttg testpop} directory.  {\ttg testpop} 
reads a (probably modified) version of the {\ttg synth} input file, 
which must be named {\ttg testpop/synth.dat}.  It also reads its 
own input parameter file, which is specified on the command line:

\medskip
\noindent 
{\ttg \% ./testpop < input\_file}

\medskip
\noindent 
The input file should contain the following 8 header lines:

\medskip
\noindent
\begin{tabular}{|lr|l|} \hline
{\tto pre}      &[string]& a prefix for the output photometry files (up to 8 characters) \\
{\tto lockflag} &[flag]& 1 = use locked amplitudes (use {\tto lockfile} in synth.dat) \\
{\tto nstars}   &[int]& an amp of 1.0 will draw {\tto nstars} masses from the isochrone \\
{\tto dmod}     &[real]& delta-distance modulus.  For adjusting the DM of the isochrones \\
{\tto fext}     &[real]& additional extinction multiplier \\
{\tto gamma}    &[real]& IMF slope ({\ttg testpop} recalculates OPs) \\
{\tto fbinary}  &[real]& Binary fraction \\ 
{\tto sfrflag}  &[flag]& 1 = amps are expressed as $M_\odot/yr$; 0 = amps are Nstars/bin \\ \hline
\end{tabular}

\medskip
\noindent
The values for {\tto gamma} and {\tto fbinary} supercede the values in 
{\ttg synth.dat}.  Immediately following these 8 parameters, the input 
file should contain N lines with the following columns:

\medskip
\noindent
{\ttg $Amp_{SFH}$, $Z_{metal}$, $log(age)$, $num_{iso}$}

\medskip
\noindent
There should be one line for each isochrone in the isofile (if 
{\tto lockflag}=0), or one line for each independent isochrone group 
(if {\tto lockflag}=1).  {\ttg testpop} determines the number of stars 
to generate for isochrone $i$ according to the formula:

$$N(i) = nstars \times Amp(i) \times frac(i) \times \Delta t(i) / num_{iso}(i)$$

\noindent where: $frac(i)$ is the fraction of the full mass range that 
is represented on the isochrone ($frac(i)$ decreases with age, as more 
stars have evolved to their non-luminous end-states); $\Delta t(i)$ is 
the duration of the age bin, in Gyr; and $num_{iso}(i)$ is the number 
of isochrones in the current isochrone group.  If {\tto sfrflag}=0, 
then $\Delta t(i)=1.0$ for all $i$.  If {\tto lockflag}=0, then
$num_{iso}(i)=1$ for all $i$.

Be aware that, in general, many more stars are generated than actually 
appear in the CMDs, because a large fraction of the generated stars 
will be too faint to detect.  

\medskip
\noindent
{\ttg testpop} outputs {\tt ncmd} output photometry files, each 
containing the following data:

\medskip
\noindent
{\ttg $mag_x$},  {\ttg $mag_y$}, {\ttg $\Delta_x$}, {\ttg $\Delta_y$}

\medskip
\noindent
These files may be used directly as input data to the {\ttg sfh} 
program.


\chapter{repop}\label{ch:repop}

\noindent
{\ttg repop} is nearly identical to {\ttg testpop}.  The only 
difference is that the input SFH amplitudes are generated from the SFH 
amplitudes output by {\ttg sfh}.  {\ttg repop} constructs an 
artificial stellar population based on the best-fit SFH solution, so 
that it can be compared to the input population.

Before using {\ttg repop}, you must first generate the input SFH 
amplitudes file using {\ttg mkinput.awk}:

\medskip
\noindent
{\ttg \% cd repop} \\
{\ttg \% gawk -f mkinput.awk -v pre=file\_prefix sfh\_output\_file}

\medskip
You may need to modify this script to make it suitable to your data.
Specifically, the number of isochrone groups and number of isochrones 
per group are hard-coded, as are the metallicity strings.

Now you can actually run {\ttg repop}, specifying the new input file 
on the command line:
\medskip
\noindent
{\ttg \% repop $<$ file\_prefix.input}

\medskip
The number of stars constructed by {\ttg repop} is probably different 
from the number of data stars.  To generate the same number of stars, 
follow these steps:

\begin{enumerate}
\item use {\ttg wc} to count the number of stars in one of repop's 
output CMD files, and the corresponding input data CMD file.
 
\item multiply {\tto nscale} (in the {\ttg repop} input file) by 
N(data)/N(repop)

\item run {\ttg repop} again
\end{enumerate}

The {\ttg repop.sm} script can be used to display the CMDs of the 
generated population.  Again, you may need to modify this script.  
This script as written displays the repopulated B-V,V CMD alongside 
the original B-V,V CMD for easy comparison. It takes two arguments, 
the prefixes of the repopulated and original photometry files.


\chapter{testchi}\label{ch:testchi}

{\ttg testchi} provides an interactive front-end to the 
{\ttg fitstat.f} subroutine, which calculates the fitting statistic of 
a given model SFH.  Compile testchi by typing {\ttg make} in the 
{\ttg testchi} directory (the executable is placed in the root 
{\ttg SFH-1.1} directory).

The {\ttg testchi} input file is a simplified version of the 
{\ttg sfh} input file, with several parameters removed.  The 
{\ttg testchi} input file contains the following parameters:
 
\medskip
\noindent
{\scriptsize
\begin{tabular}{|lr|l|} 
\hline
\multicolumn{3}{|c|}{\bf Filenames} \\ 
\hline
{\tto datpre}   &[string]& input data file prefix\\ 
{\tto cmdfile}  &[string]& Sample synthetic CMDs description file\\
{\tto maskfile} &[string]& Allows one to ignore specified regions of the CMDs\\
{\tto holdfile} &[string]& file w/ SFH amplitudes that should be held fixed\\
{\tto ampfile}  &[string]& the input model SFH amplitudes to test\\
{\tto chifile}  &[string]& output file containing grid populations and chi-squared values\\
\hline
\multicolumn{3}{|c|}{\bf Synthetic CMD parameters} \\ 
\hline
{\tto nfree}  &[int]& number of independent isochrone amplitudes\\
{\tto ncmd}   &[int]& number of CMDs over which to calculate chi-squared\\
{\tto npix}   &[int]& binning factor for CMDs\\
{\tto dpix}   &[real]& size of synthetic CMD pixels (same as in {\ttg synth.dat})\\
\hline
\multicolumn{3}{|c|}{\bf Parameters for each CMD} \\
\hline
{\tto suffix(i)} &[string]& CMD filename suffix (same as in {\ttg synth.dat})\\
{\tto   xmin(i)} &[real]& calculate chi-squared within these CMD limits\\
{\tto   xmax(i)} &[real]& calculate chi-squared within these CMD limits\\
{\tto   ymin(i)} &[real]& calculate chi-squared within these CMD limits\\
{\tto   ymax(i)} &[real]& calculate chi-squared within these CMD limits\\
{\tto   nbox(i)} &[int]& number of grid boxes within region defined by above limits\\ 
\hline
\multicolumn{3}{|c|}{\bf Runtime parameters} \\ 
\hline
{\tto fit\_stat} &[int]&  0 = use chi-squared; 1 = use Lorentzian; 2 = Use Poisson\\
{\tto gtype}     &[flag]&  grid definition flag (0 = uniform grid; 1 = custom grid)\\
{\tto iverb}     &[int]& verbosity flag (larger N = more screen output)\\
\hline
\end{tabular}
}

\medskip
To use {\ttg testchi}, adjust the parameters in {\ttg testchi.dat} (or 
create a new input file) and run the program:

\medskip\noindent
{\ttg \% ./testchi < input/testchi.dat}

\medskip
The chi-squared value resulting from a comparison of the data and the 
model described by the {\tto outfile} is printed to the screen, and 
the gridbox populations are written to the {\tto chifile}.  

{\ttg testchi} is very useful for quickly seeing changes in the 
chi-squared value when the SFH amplitudes (or other parameters) are 
changed.  You can examine the distribution of chi-squared in the CMDs 
using the {\tto chifile}, and the {\ttg mkimages.sh} script.


\chapter{mkimages.sh}\label{ch:grid}

The {\ttg grid} subdirectory contains the script {\ttg mkimages.sh},
which generates images of the CMDs from the grid box populations in 
the {\tto chifile}, and generates an HTML document which displays the 
images in a grid.  The HTML document provides a very convenient way to 
examine the quality of the model fit, and to diagnose what may be 
causing a poor fit.

The script requires files named {\ttg grid/gridN.index} which 
contains a description of the grid boxes in each of your CMDs
(each CMD has its own index file). The columns are: 

\medskip\noindent
{\ttg ix}, {\ttg iy}, {\ttg ibox}

\medskip
\noindent
where ({\ttg ix}, {\ttg iy}) are the synthetic CMD pixel coordinates, 
and {\ttg ibox} is the grid box to which that pixel belongs.  You can 
modify the program {\ttg mkgrid.f} to generate the gridN.index files 
for your CMDs.

Once you have valid {\ttg gridN.index} files, you can use the 
{\ttg mkimages.sh} script to generate the images and HTML page.  
Before running the script, there are several parameters which should 
be adjusted; see the "CUSTOMIZATION" section of the {\ttg mkimages.sh} 
file.

When running the script, you specify a filename prefix for the HTML 
document and images, and the name of the {\tto chifile} on the command 
line:

\medskip\noindent
{\ttg \% ./mkimages.sh c1978 ../output/c1978.chi}

\medskip
The script will generate 4 images per CMD, which will be placed in the 
{\ttg images/} subdirectory.  The images represent the grid box 
populations of the model and of the data; the difference between model 
and data, and the contribution of each grid box to the fitting 
statistic.  You can view all of the images using the HTML document 
that is created by the script.


\begin{thebibliography}{75}

\bibitem[Bertelli et~al. (1994)]{ber94}
{Bertelli}, G., {Bressan}, A., {Chiosi}, C., {Fagotto}, F., \& {Nasi}, E. 1994,
  \aaps, 106, 275

\bibitem[Girardi et~al. (2000)]{gir00}
{Girardi}, L., {Bressan}, A., {Bertelli}, G., \& {Chiosi}, C. 2000,
  \aaps, 141, 371
  
\end{thebibliography}
\end{document}


\chapter{Introduction}\label{ch:intro}

\section{Overview}

The {\bf StarFISH} package is a suite of FORTRAN programs designed to 
determine the best-fit star formation history (SFH) of a stellar 
population.  The package constructs a library of synthetic 
color-magnitude diagrams (CMDs) based on theoretical isochrones and 
data-derived determinations of the interstellar extinction, 
photometric errors, and distance modulus.  These synthetic CMDs are 
combined linearly and statistically compared to observed photometry.  
When the best-fitting model is found, the amplitude coefficients on 
each synthetic CMD describe the star formation history of the observed 
population.

\medskip
\noindent
If you have any comments, corrections, suggestions, or complaints,
please email: \\
{\ttb jharris@as.arizona.edu}

\bigskip
\noindent
To use the package, you need:

\begin{itemize}
\item {\bf Multicolor photometry} of a stellar population. \\

\item A (preferably empirical) {\bf photometric error model}, 
typically derived from artificial star tests. \\

\item A {\bf library of isochrones} that covers relevant ranges in age 
and metallicity. \\

\item {\it a priori} determination of the {\bf interstellar 
extinction} and {\bf distance modulus}. \\

\item Estimates (or guesses) of the {\bf binary fraction} and {\bf IMF 
slope}. \\
\end{itemize}

\clearpage

The package consists of three main programs, and five support
programs: 

\begin{quote}
\bigskip\noindent
$\bullet$ {\ttg mklib}: constructs a library of isochrone 
photometry from raw isochrones.

\medskip\noindent
$\bullet$ {\ttg synth}: constructs synthetic CMDs 
from the isochrones, applying the effects of interstellar extinction, 
binarity, IMF, and photometric errors as appropriate for the data.

\medskip\noindent
$\bullet$ {\ttg sfh}:   performs a 
maximum-likelihood comparison of composite model CMDs and input data 
CMDs.  The best-fit star formation history is output. 


\bigskip\noindent
- {\ttg interp}:  interpolates between adjacent 
isochrones, to improve resolution in age and/or metallicity.  Requires 
knowledge of equivalent evolutionary points (EEPs) along each 
isochrone. 
   
\medskip\noindent
- {\ttg testpop}: constructs artificial stellar 
populations from the isochrones.  {\ttg testpop} 
incorporates the photometric errors, extinction, etc. exactly as is 
done in {\ttg synth}.  These populations can be 
input to {\ttg sfh} for testing purposes.  
({\ttg testpop} is in the subdirectory of the same 
name)

\medskip\noindent
- {\ttg repop}: same as {\ttg testpop}, except used to construct an 
artificial population from the SFH amplitudes found by {\ttg sfh}.  
This is useful for model/data comparisons. ({\ttg repop} is in the 
subdirectory of the same name)

\medskip\noindent
- {\ttg testchi}: allows one to examine the parameter space of the 
fit by hand.  SF amplitudes and other parameters can be interactively 
adjusted, and the resulting fit recomputed on the fly. ({\ttg testchi} 
is in the subdirectory of the same name) 

\medskip\noindent
- {\ttg mkimages.sh}: creates CMD images from the best-fit model.  It 
also creates CMD maps of each region's contribution to the fitting 
statistic. 
\end{quote}

\clearpage

\section{Getting Started}

To begin, you need to obtain the program code from the Magellanic 
Clouds Photometric Survey website: \\
{\ttb http://ngala.as.arizona.edu/mcsurvey/}

\medskip
As of this writing, the current version is 1.1.  Note that there is a 
full version and a compact version.  The full version includes a 
sample pre-generated synthetic CMD library for testing purposes.  If 
you download the compact version, you can still generate the sample 
synthCMD library, but it will take some time.

\medskip 
The following command will unpack the directory tree of the StarFISH 
distribution. It will create a {\ttg SFH-1.1} subdirectory in the 
current directory.

\medskip\noindent
{\ttg \% tar zxvf SFH-1.1.tar.gz} \hfill (or {\ttg tar zxvf SFH-1.1compact.tar.gz}) \\

\noindent or (on Solaris): \\
\medskip
\noindent
{\ttg \% gunzip SFH-1.1.tar.gz} \\
{\ttg \% tar xvf SFH-1.1.tar}

\medskip
The package contents are shown in {\bf Table~1}.

\bigskip
\noindent{\large {\bf Table 1: Package Contents}}\\
{\scriptsize
\begin{tabular}{|l|l|}
\hline
{\bf File/Directory Name}  & {\bf Description} \\ 
\hline
{\ttg manual/}     & The StarFISH manual (this document)\\
{\ttg libcode/}    & source code and Makefile for the {\ttg mklib} program\\
{\ttg synthcode/}  & source code and Makefile for the {\ttg synth} program\\
{\ttg sfhcode/}    & source code and Makefile for the {\ttg sfh} program\\
{\ttg commoncode/} & source code used by multiple programs\\
{\ttg interpcode/} & source code and Makefile for the {\ttg geteep} and {\ttg interp} programs\\
{\ttg input/}      & input data files for {\ttg mklib}, {\ttg synth} and {\ttg sfh}\\
\hspace{20pt}{\ttg /lib.dat}   & sample {\ttg mklib} parameter file\\
\hspace{20pt}{\ttg /synth.dat} & sample {\ttg synth} parameter file\\
\hspace{20pt}{\ttg /sfh.dat}   & sample {\ttg sfh} parameter file\\
\hspace{20pt}{\ttg /iso.dat}   & sample isochrone description files\\
\hspace{20pt}{\ttg /iso.lock}  & describes how isochrones are to be locked together\\
\hspace{20pt}{\ttg /test.cmds} & sample synthetic CMD description file\\
\hspace{20pt}{\ttg /mask.none} & file indicating CMD regions to ignore in the fit (default: null mask)\\
\hspace{20pt}{\ttg /test.hold} & file indicating which SFH amplitudes should be held fixed (default: none held)\\
\hspace{20pt}{\ttg /av/*.av}   & sample extinction data files\\
\hspace{20pt}{\ttg /crowd/*.crowd} & sample crowding table from artificial star tests\\
\hspace{20pt}{\ttg /dtime.dat} & file describing age interval covered by each isochrone\\
\hspace{20pt}{\ttg /geteep.dat} & input file for {\ttg geteep} \\
\hspace{20pt}{\ttg /interp.dat} & input file for {\ttg interp} \\
\hspace{20pt}{\ttg /testchi.dat} & input file for {\ttg testchi} \\
{\ttg testpop/} & files related to the {\ttg testpop} support program\\
{\ttg testchicode/} & files related to the {\ttg testchi} support program\\
{\ttg repop/}   & files related to the {\ttg repop} support program\\
{\ttg grid/}    & files related to the {\ttg grid} support scripts\\
{\ttg iso/}     & raw isochrone photometry from Girardi et~al. (2002, 2000) 
                  and Bertelli et~al. (1994)\\
{\ttg testlib/}\footnotemark[1]  & sample isochrone library produced by {\ttg mklib}\\
{\ttg test.syn/}\footnotemark[1] & sample synthetic CMDs based on artificial star tests
                                   from an image of NGC~1978.\\
{\ttg data/}   & input data photometry goes here (NGC~1978 photometry included)\\
{\ttg output/} & {\ttg sfh} results and the {\ttg sfh.sm} plotting script\\ \hline
\end{tabular}
}

\footnotetext[1]{Not present in the compact version of the program}

\clearpage

\section{Compiling the Principal Code}

Each principal program ({\ttg mklib}, {\ttg synth}, and {\ttg sfh}) 
has its own Makefile, in its source code directory.  To compile these 
programs, simply cd to the correct directory and type {\ttg make}:

\medskip
\noindent
{\ttg \% cd /path/to/SFH-1.1} \hfill (if not already in the SFH directory) \\
{\ttg \% cd libcode} \\
{\ttg \% make} \\
{\ttg \% cd ../synthcode} \\
{\ttg \% make} \\
{\ttg \% cd ../sfhcode} \\
{\ttg \% make}

\medskip
Each program will be copied into the {\ttg SFH-1.1} directory, which 
is where you should run them.  You are now ready to experiment with 
the code.

\vspace{2em}
\section{A Quick Run}

We will use all of the default parameter values to go from a freshly 
downloaded and compiled distribution to a successful first run of 
{\ttg sfh}.  Steps labeled with an asterisk (*) may be skipped if you 
downloaded the full version of StarFISH, because this version already 
contains the pregenerated isochrone library and synthetic CMD library.

\begin{enumerate}

\item {\ttg \% cd /path/to/SFH-1.1} \hfill (if not already in the SFH directory)

\item (*) Prepare isochrones:  (see Section~\ref{sec:iso})\\
{\ttg \% cd iso} \\
{\ttg \% tar zxvf girardi.tar.gz} \hfill (unpack the isochrones) \\
{\ttg \% gawk -f parse\_iso.awk isocz*.dat} \hfill (parse isochrones) 

\item (*) Generate isochrone library: (see Chapter~\ref{ch:lib})\\
{\ttg \% cd ..} \\
{\ttg \% mkdir testlib} \hfill (create target directory before running {\ttg mklib}!)\\
{\ttg \% ./mklib < input/lib.dat} 

\item (*) Generate Synthetic CMD library: (see Chapter~\ref{ch:synth}) \\
{\ttg \% mkdir test.syn} \hfill (create target directory before running {\ttg synth}!)\\
{\ttg \% ./synth < input/synth.dat} 

\item Perform SFH Solution: (see Chapter~\ref{ch:sfh}) \\
{\ttg \% ./sfh < input/sfh.dat} 

\item Plot solution: \\
{\ttg \% cd output} \\
{\ttg \% sm} \hfill (start supermongo plotting program) \\
{\ttg sm$>$ macro read sfh.sm } \hfill (load sm script) \\
%{\tt sm$>$ dev x11} \hfill (bring up display window) \\
{\ttg sm$>$ sfh c1978 c1978} \hfill (display SFH solution)
\end{enumerate}

\clearpage

\section{What's New in Version 1.1}

There have been several improvements made since 1.0; the most 
important changes are listed here.  See the files {\ttg ChangeLog} and 
{\ttg NEWSTUFF} for more details.

\begin{itemize}
\item Fixed random-number generator.  We found a periodicity in 
the previous one. \\

\item More robust handling of input files.  The programs are now 
much less picky about the format of input files ({\itr note that 
input files from previous versions of StarFISH are not compatible 
with this version}). \\

\item Input files can now contain comment lines.  Any line in an 
input file that begins with {\ttg \#} or {\ttg *} will be ignored by 
the parser. \\

\item Filenames in the input files can now be up to 40 characters 
long (previous maximum was 24 characters) \\

\item Option to use Dolphin's ``Poisson'' fit statistic instead of 
the chi-squared statistic. \\

\item The photometric error model used in {\ttg synth}, 
{\ttg testpop}, and {\ttg repop} now properly includes covariance 
between the color and magnitude errors. \\

\item Option to use an analytic error model instead of an 
AST-derived crowding table in {\ttg synth}, {\ttg testpop}, and 
{\ttg repop}. \\

\item {\ttg sfh} writes the current best solution to the logfile more 
frequently. \\

\item New empirical isochrone interpolation code ({\ttg geteep} and 
{\ttg interp})

\end{itemize}


\chapter{mklib}\label{ch:lib}

{\ttg mklib} is used for processing raw isochrone photometry into a 
format usable by {\ttg synth}.  {\ttg mklib} applies a distance modulus, 
interpolates along the isochrone to a uniformly high resolution, and 
computes occupation probabilities (OP) for each isochrone point.

The package currently ships with the Padua isochrones, most recently
published by Girardi et~al. (2002).  We found that the main sequence 
in these isochrones is too coarsely sampled for our purposes, 
resulting in artificially "lumpy" synthetic CMDs. Therefore,  
{\ttg mklib} includes a photometric interpolation routine for points 
fainter than the main sequence turn-off (MSTO; defined for each 
isochrone in the {\tto isofile}).  If your isochrones do not require 
this interpolation, simply set the MSTO magnitude for each isochrone 
to be fainter than any isochrone point.  

\bigskip
The {\ttg mklib} input file (named {\ttg input/lib.dat} by default) 
contains the parameters listed in Table~2 (the lines 
must be present in the listed order). \\

\noindent{\large {\bf Table 2: mklib input file parameters}} \\
{\scriptsize
\begin{tabular}{|lr|l|}
\hline
{\tto isofile} &[string]& description of each isochrone: the input filename, \\
               &        & the output filename, the age, and the MSTO magnitude. \\
{\tto faint}   & [real] & the faint magnitude limit for the output isochrone library.\\
               &        & Should be several mag fainter than the data's faint limit. \\
{\tto dmag}    & [real] & photometric distance between adjacent interpolated points.\\
{\tto dmod}    & [real] & distance modulus \\
{\tto gamma}   & [real] & logarithmic IMF slope (Salpeter = -1.35) \\
{\tto nmag}    &  [int] & number of magnitude values per isochrone point \\
{\tto mag0}    &  [int] & which magnitude to check against {\bf faint} and  {\bf msto} \\
{\tto iverb}   &  [int] & verbosity flag (0=silent, 1=screen messages, 2=extra output files)  \\
\hline
\end{tabular}
}

\clearpage

\noindent
To use {\ttg mklib}:

\begin{enumerate}
\item {\bf Create an isofile that describes your isochrones.}  This 
file will also be used by {\ttg synth}.  Each line in the file should 
contain the following space-delimited fields (the column format does 
not matter):

\medskip\noindent
{\tto log(age)}~[real number], 
{\tto input raw isochrone file}~[up to 40 chars], 
{\tto output isochrone file}~[up to 40 chars], 
{\tto msto\_mag}~[real number]

\medskip
{\tto msto\_mag} is the absolute magnitude of each isochrone's MSTO point.  
If you do not know the MSTOs for your isochrones, you'll need to identify 
them.  Use the fact that at the MSTO, the occupation probability changes 
dramatically.
\medskip

\item {\bf Preprocess the isochrone photometry files}.  Each isochrone 
must be in its own file with a name matching the isochrone's entry in 
the {\tto input raw isochrone file} list (column 2 of the isofile).  
The columns required for the isochrone files are:

\medskip\noindent
{\tto mass}, {\tto mag1}, {\tto mag2}, {\tto mag3}, ..., {\tto magN}
 
\medskip
where N is the equal to the {\tto nmag} parameter in {\ttg lib.dat}.  
Column format does not matter.  The magnitudes should be expressed as 
absolute magnitudes; {\ttg mklib} will apply the distance modulus 
specified in {\ttg lib.dat} and output apparent magnitudes.  For the 
default Padua isochrones, the input isochrone files are generated by 
the {\ttg parse\_iso.awk} script. See the next section for details on 
the default isochrone-processing.  Make sure the {\tto mag0} parameter 
indicates the pass-band which should be used for comparing to the 
{\tto faint} and {\tto msto\_mag} values.  

\medskip
\item {\bf Create the target output directory, if necessary}. The 
output directory is specfied as part of the output filenames in the 
{\tto isofile}.

\medskip
\item {\bf Run {\ttg mklib}:} \\
{\ttg \% ./mklib < input/lib.dat}

\medskip
Depending on the verbosity flag, {\ttg mklib} may provide screen 
messages regarding which isochrone it is working on.  The output 
directory will contain the processed isochrones, with the following 
columns:

\medskip\noindent
{\tto mag1}, {\tto mag2}, {\tto mag3}, ..., {\tto magN}, {\tto OP}, 
{\tto mass}

\medskip
where the {\tto magN} are the apparent magnitudes in each band, 
{\tto OP} is the relative occupation probability associated with each 
point (determined from the IMF and from the relative duration of each 
evolutionary stage), and {\tto mass} is the {\bf initial} mass of each 
isochrone point, not including any mass-loss that may have occured 
since the star was formed.
\end{enumerate}

\clearpage

\section{Pre-processing the Padua Isochrones}\label{sec:iso}

The {\ttg iso/} directory contains files related to the Padua 
isochrones, which we ship with StarFISH for convenience.  You are free 
to use any other isochrone set that you wish, as long as it can be 
placed in the format that {\ttg mklib} requires (separate files for 
each isochrone, with the column format specified in step 2 above).

The Padua isochrones are disributed as a tar archive, which we include 
in StarFISH as the file {\ttg iso/girardi.tar.gz}.  The archive contents 
are listed in Table~3.

\bigskip
\noindent{\large {\bf Table 3: Padua isochrone package contents}} \\
\begin{tabular}{|l|l|}
\hline 
{\ttg isocz0[001,004,01,04,08,19,30].dat} &  Girardi et~al. isochrones\\
{\ttg isocsummz0[001,004,01,04,08,19,30].dat} &  Tables of Equivalent 
    Evolutionary \\
    & Points (EEPs) \\ 
{\ttg read.me} &  Girardi's readme file\\ 
\hline
\end{tabular}

\vspace{3em}
Each of the isochrone files {\ttg isocz0*.dat} contains isochrones of 
all ages for a given metallicity.  StarFISH needs a separate file for 
each isochrone, so we provide the AWK script {\ttg parse\_iso.awk} to 
divide each raw isochrone collection into individual files.  We also 
provide the script {\ttg parse\_summ.awk} to divide the EEP table 
collections into individual EEP tables for each isochrone.

If you downloaded the full version of StarFISH, then you do not need 
to run the parsing scripts; you already have the processed isochrone 
files, named {\ttg iso/zNNNN\_tt.tt} (where {\ttg NNNN} is a 4-digit 
code for the metallicity, such as {\ttg 0080}; and {\ttg tt.tt} is the 
log of the age, such as {\ttg 10.00}).  If you do not have these 
files, generate them like this:

\medskip\noindent
{\ttg \% cd iso}\\
{\ttg \% tar zxvf girardi.tar.gz} \\
{\ttg \% gawk -f parse\_iso.awk isocz*.dat} \\
{\ttg \% gawk -f parse\_summ.awk isocsummz*.dat} \\

\medskip
If you are satisfied with the age and metallicity resolution provided 
by the default Padua isochrones, then you are all done here.  If you 
want to try to interpolate between isochrones to increase age or 
metallicity resolution, refer to the chapter discussing the 
{\ttg interp} support program.


\chapter{synth}\label{ch:synth}

{\ttg synth} constructs a synthetic CMD library from the isochrone 
library, incorporating the photometric effects of extinction, 
binarity, the IMF, and photometric errors.  Each synthetic CMD is 
represented as a list of ``pixel values'', equal to the relative number 
of stars found at the corresponding subregion in the CMD.  If the data 
contain more than two filters, then multiple CMDs can be created for 
each isochrone, so that all of the data can be utilized in the fit.  
For example, if your data include $U$, $B$, and $V$ photometry, you 
might use {\ttg synth} to create two CMDs per isochrone: one with 
$U-B$ vs. $B$ axes, and another with $B-V$ vs. $V$ axes.

Some of the user-provided parameters (extinction, photometric errors, 
and binarity) are first encountered in the {\ttg synth} program. 
Others (the IMF and distance modulus) have already been used to 
create the isochrones output by {\ttg mklib}.  Therefore, if you need 
to change the IMF or distance modulus you must run {\ttg mklib} first. 

Depending on the age and metallicity resolution of your isochrone 
library, and on the quality of your photometric data, it is likely
that some of the isochrones are photometrically degenerate.  If 
{\ttg synth} is allowed to produce degenerate synthetic CMDs, then the 
SFH solution determined by {\ttg sfh} will be ambiguous, and will 
suffer from large correlated uncertainties among the SFH amplitudes.  
To avoid ambiguous SFH fits, {\ttg synth} can combine 
photometrically-degenerate isochrones together into the same synthetic 
CMD.  This is referred to as {\bf locking} the isochrones. 

\clearpage

\noindent The synth input file ({\ttg input/synth.dat}) contains the 
parameters listed in Table~4. {\itr Note that synth input files from 
previous versions of StarFISH will no longer work}.

\bigskip
\noindent{\large {\bf Table 4: synth input file parameters}} \\
{\scriptsize
\begin{tabular}{|lr|l|}
\hline
\multicolumn{3}{|c|}{\bf Filenames} \\ 
\hline
{\tto isofile}  &[string]& description of the isochrones (can use the file from {\ttg lib.dat}) \\
{\tto lockfile} &[string]& file describing the combination of degenerate isochrones\\
{\tto hotfile}  &[string]& list of extinction measurements of hot (young) stars\\
{\tto coldfile} &[string]& list of extinction measurements of cool (old) stars\\
{\tto crowd1}   &[string]& crowding table from artificial star test\\
{\tto crowd2}   &[string]& output file for lookup table of binned delta-magnitudes \\ 
\hline
\multicolumn{3}{|c|}{\bf CMDs} \\ 
\hline
{\tto nmag} &  [int] & same as nmag in {\ttg lib.dat}\\
{\tto ncmd} &  [int] & number of CMDs to construct per isochrone group\\
{\tto mag0} &  [int] & same as mag0 in {\ttg lib.dat}\\ 
{\tto dpix} & [real] & size of CMD pixels (in magnitudes) {\itr[NEW]}\\
\hline
\multicolumn{3}{|c|}{\bf parameters for each of the CMDs} \\ 
\hline
{\tto xeq}    &[string]& filter equation for the x-dimension\\
{\tto yeq}    &[string]& filter equation for the y-dimension\\
{\tto xmin}   & [real] & smallest allowable x-value\\
{\tto xmax}   & [real] & largest allowable x-value\\
{\tto ymin}   & [real] & smallest allowable y-value\\
{\tto ycmax}  & [real] & largest allowable y-value, for crowding table stars\\
{\tto ypmax}  & [real] & largest allowable y-value, for data/model stars\\
{\tto suffix} &[string]& filename suffix for the synthetic CMD output files\\ 
\hline
\multicolumn{3}{|c|}{\bf Crowding} \\ 
\hline
{\tto dbinx} & [real] & width of crowding bins, in magnitudes\\
{\tto dbiny} & [real] & height of crowding bins, in magnitudes\\
{\tto emin}  & [real] & minimum delta-magnitude cutoff\\
{\tto emax}  & [real] & maximum delta-magnitude cutoff\\
{\tto dbin}  & [real] & binsize of delta-magnitude histograms\\ 
\hline
\multicolumn{3}{|c|}{\bf Reddening parameters} \\ 
\hline
{\tto red(i)} & [real] & extinction in mag(i) (relative to the filter in which extinction is measured) \\ 
\hline
\multicolumn{3}{|c|}{\bf Run-time parameters} \\ 
\hline
{\tto verb}        &  [int] & verbosity flag (higher N = more output)\\
{\tto interp\_err} & [flag]\footnotemark[2] & 1 = interpolate errors between bracketing crowding bins {\itr[NEW]}\\
{\tto fake\_errs}  & [flag]\footnotemark[2] & 1 = use analytic error model instead of crowding table {\itr[NEW]}\\
{\tto nscale}      &  [int] & number of model stars in each CMD\\
{\tto seed}        &  [int] & random seed value\\
{\tto mass1}       & [real] & minimum mass (for OP normalization)\\
{\tto mass2}       & [real] & maximum mass (for OP normalization)\\
{\tto gamma}       & [real] & logarithmic IMF slope (Salpeter = -1.35)\\
{\tto faint}       & [real] & faint limit for generated model stars\\
{\tto fbinary}     & [real] & binary fraction\\ 
\hline
\end{tabular}
}
\footnotetext[2]{A flag is an int whose value can either be 0 or 1.}

\clearpage
\noindent To use {\ttg synth}:

\begin{enumerate}
\item {\bf Construct the {\tto lockfile}.}  This file describes how the 
isochrones are to be locked together into groups.  Again, this is 
necessary if isochrones of similar age are photometrically degenerate, 
to avoid ambiguous SFH fits.  The columns of the {\tto lockfile} are as 
follows ({\itr note that lockfiles from previous versions of StarFISH 
will no longer work}):

\medskip
\noindent
{\tto group\_id}[int], 
{\tto isoname}[up to 40 characters], 
{\tto synthfilestem} [up to 40 characters]

\medskip
{\tto group\_id} identifies the locked group to which the isochrone 
belongs.  The actual {\tto group\_id} values do not matter; when the 
value of one line is different from that of the previous line, a new 
synthetic CMD is started.

{\tto isoname} is the isochrone filename.  It is only used to 
check against the filename in the {\tto isofile}.  If they don't 
match, an error is triggered; there must be a one-to-one 
correspondence of the isochrones in the {\tto isofile} and the 
{\tto lockfile}.

{\tto synthfilestem} is the filename stem for the output synthetic 
CMD pixel file.  The output filenames are constructed by concatenating 
the CMD suffixes specified in {\ttg synth.dat} onto 
{\tto synthfilestem} (these suffixes are typically the combination of 
magnitudes; for example a $U-B$ vs. $B$ CMD might have a suffix of 
``{\ttg .ub}'').

\item {\bf Construct the extinction files ({\tto hotfile}, 
{\tto coldfile}).}  The extinction files each contain a single column: 
extinction values, $A_{\lambda}$, measured in the magnitude whose 
reddening parameter in {\ttg synth.dat} is set to 1.00 (in the 
provided {\ttg synth.dat}, red(3) is set to 1.00; this is the $V$ 
band, so the extinction files list $A_V$ values).  There are two files 
({\tto hotfile} and {\tto coldfile}) to account for the different 
extinction properties of young and old stellar populations.  If you do 
not require population-dependent extinction, simply point 
{\tto hotfile} and {\tto coldfile} to the same file.  If you do not 
require extinction modeling at all, point {\tto hotfile} and 
{\tto coldfile} to the null extinction file, {\ttg input/av/zero.av}.
   
\item {\bf Construct the crowding table ({\tto crowd1})}. This is most 
likely constructed from artificial star tests.  The crowding table 
contains:

\medskip   
{\tto ra}, {\tto dec}, {\tto mag1}, {\tto dmag1}, {\tto mag2}, {\tto dmag2}, 
..., {\tto magN}, {\tto dmagN}

\medskip\noindent
column format doesn't matter.  The crowding table should contain all 
artificial stars injected into your data images, even if they were not 
detected in your photometric pipeline.  In fact, these non-recovered 
``dropout'' stars are extremely important, because {\ttg synth} uses 
them to determine the completeness rate as a function of photometric 
position in the CMDs.  For bands in which an artificial star was 
detected, the listed {\tto mag} value is its {\bf recovered} 
magnitude, and {\tto dmag} is (mag(recovered) - mag(input)).  For 
bands in which an artificial star dropped out, {\tto mag} is the 
star's {\bf input} magnitude, and {\tto dmag} is set to 9.99 in order 
to flag it as a dropout in that band.  

If you decide not to perform artificial star tests on your images 
(although such tests are {\it highly} recommended), you can either 
generate a fake crowding table by hand, based on your estimates of the 
scatter and completeness rates in each band, or you can set the 
{\tto fake\_errs} parameter in {\ttg synth.dat} to 1, which will use a
hueristic analytic error model instead of the empirical crowding 
table.  In this case, you will need to modify the code in 
{\ttg synthcode/fakecrowd.f} to generate photometric errors and 
completeness rates appropriate for your data.  The comments in that 
file will guide you in making these changes.

\item Decide how many CMDs you would like, and what their axes will 
be.  You specify the CMD axes with the {\tto xeq(icmd)} and 
{\tto yeq(icmd)} string variables in {\ttg synth.dat}.  These strings 
indicate the filter combination for each CMD axis.  For example, you 
have $VRI$ photometry, and you are using $V$=1, $R$=2, $I$=3.  Your 
first CMD could be: $V-I$ vs. $V$, so {\tto xeq(1)} = 1-3, 
{\tto yeq(1)} = 1.  Your other ``CMD" could be a two-color diagram: 
$V-R$ vs. $V-I$, so {\tto xeq(2)} = 1-2, {\tto yeq(2)} = 1-3.  The 
strings can be either the sum or the difference of two filter indices, 
or one index by itself.  Do not put whitespace in the equations; each 
must be a single ``word''.

You will also need to specify photometric limits for each CMD 
(parameters {\tto xmin}, {\tto xmax}, {\tto ymin}, {\tto ycmax}, and 
{\tto ypmax}).  Note that instead of one {\tt ymax}, there are two 
variables: {\tto ycmax} and {\tto ypmax}.  {\tto ycmax} is the faint 
limit for the crowding bins, and {\tto ypmax} is the faint limit for 
the CMD itself.  {\tto ycmax} should be at least 2 or 3 mag fainter 
than {\tto ypmax}, because very faint stars can be blended and 
detected at much brighter magnitudes than their true magnitude.  

Also specify the size of CMD ``pixels'' in the output synthetic CMD 
files ({\tto dpix}), in magnitudes.

Finally, specify a suffix string that will be concatenated with 
{\tto synthfilestem} to construct the output file name for each 
synthetic CMD.  The suffices can be up to 8 characters long, and 
typically indicate something about the CMD axes. For example, a
$B-V$ vs. $V$ CMD might have a suffix ``{\ttg .bv}''.

\item Specify the size of the crowding bins in each direction 
({\tto dbinx} and {\tto dbiny}).  Artificial stars in the crowding 
table are binned in the CMDs so that the brightness- and 
color-dependence of the photometric errors can be accounted for.
Independent $\Delta mag$ histograms will be constructed from the 
artificial stars present in each bin.  

Specify the limits for the delta-mag histograms with {\tto emin} and 
{\tto emax}.  Any $\Delta mag$ value in the crowding table outside 
these limits is treated as a photometric dropout.

The {\tto dbin} parameter sets the width of the bins in the delta-mag 
histograms ({\itr this parameter used to determine the size of the 
synthetic CMD pixels as well, but there is now a separate parameter 
for that: dpix}).

\item Specify the reddening law to use, by setting the relative 
extinction for each band: {\tto red(imag)}.  The band for which you 
have measured extinction values in {\tto hotfile} and {\tto coldfile} 
should have {\tto red(imag)} = 1.0.  The rest of the values should be 
the extinction in that band, assuming the extinction in the reference 
band is 1.0.  For example, you have $BV$ photometry, and measured 
A$_V$ extinction values.  A$_V$ = 1.0.  If we assume that 
A$_V$/E(B-V) = 3.2 = A$_V$/(A$_B$ - A$_V$), then A$_B$ = 1.3125. 

\item Specify miscellaneous parameters.  {\tto nscale} sets the number 
of model stars to include in each CMD.  {\tto nscale} should be much 
larger than the number of stars in your data CMDs, so that the Poisson 
noise in the models is much smaller than in the data.  {\tto iseed} is 
the random seed value; it can be any integer.  {\tto mass1}, 
{\tto mass2}, and {\tto gamma} are IMF parameters used for normalizing 
the pixel values.  {\ttg synth} will only attempt to add model stars 
brighter than the {\tto faint} parameter.  This is an efficiency 
feature: without this cutoff, the code would waste a lot of time 
adding undetectable stars.  Set {\tto faint} to just fainter than 
where your completeness rate reaches 0\%.

\item Once the parameter file is complete, and the other input files 
are ready, you can finally run {\ttg synth}.  It will take quite a bit 
of time to complete, depending on computing power and the number of 
isochrones.  I found it usually took several hours to process 150 
isochrones on a 700 MHz linux box.  
\end{enumerate}

You can examine the synthetic CMDs as images, using either the IRAF 
{\tt rtext} command, or the {\ttg mkgif.bat} and {\ttg pxl2ppm.awk} 
scripts I provide in the {\ttg commoncode} directory.


\chapter{sfh}\label{ch:sfh}

{\ttg sfh} performs a ``chi-squared-like'' minimization comparison 
between data photometry and a linear combination of the synthetic 
CMDs.  Each synthetic CMD is modulated by an {\bf amplitude} that is 
proportional to the number of stars with the age and metallicity of 
the synthetic CMD's parent isochrone(s).  These amplitudes form the 
multidimensional parameter space of the minimization.

I say ``chi-squared-like'' because you may actually choose among three 
statistics to determine the quality of the fit: chi-squared, 
Lorentzian, or Poisson.  The three parameters weight outliers slightly 
differently, but they are otherwise quite similar.  Hereafter, I will 
simply refer to the fitting statistic as ``chi-squared''.

The minimization analysis employs the downhill simplex ``amoeba'' 
algorithm as described in Numerical Recipes (Press {\it et al.}; 
hereafter NR).  We have written our own implementation of this 
algorithm.  It works by sampling a small region of the parameter  
space and determining the local chi-squared gradient.  It then takes a 
small step in the down-gradient direction, and determines the local 
chi-squared gradient around this new location.  The process iterates 
until a (nearly) zero gradient is found, signalling that a minimum
has been found.  

The amoeba algorithm in NR has some safeguards against finding local 
minima, but we found that it was still prone to getting stuck.  The 
simplex by definition can only sample parameter space along directions 
parallel to the N axes of the parameter space; it is blind to off-axis 
directions.  However, as it nears the minimum, lower chi-squared 
values will tend to be found when a change in one parameter is 
complemented by a change in one or more other parameters.  In other 
words, downward gradients will tend to lie along off-axis directions.

To remedy this problem with the pure simplex method, we add a 
random-search loop, which is triggered whenever the simplex claims it 
has found a minimum.  In the random-search loop, we select a random 
direction in the multidimensional parameter space, and take a small 
step along that direction.  If the step yields a smaller chi-squared
value, we continue stepping along that direction until we reach a step 
that no longer decreases chi-squared.  This is repeated for a large 
number of random directions, and the parameter space location which 
yielded the lowest overall chi-squared value is recorded.  The simplex 
is then restarted at this new parameter space location.

This iteration between the downhill simplex algorithm and the 
random-search loop is repeated until the random-search loop cannot 
find a direction which yields a lower chi-squared value.
We then accept the last simplex location as the true minimum, and 
output the best-fit SFH amplitudes.

\bigskip
\noindent The {\ttg sfh} input file ({\ttg input/sfh.dat}) contains 
the following parameters:

\medskip
\noindent
{\scriptsize
\begin{tabular}{|lr|l|} 
\hline
\multicolumn{3}{|c|}{\bf Filenames} \\ 
\hline
{\tto datpre}   &[string]& input data file prefix\\ 
{\tto cmdfile}  &[string]& Sample synthetic CMDs description file\\
{\tto maskfile} &[string]& Allows one to ignore specified regions of the CMDs\\
{\tto holdfile} &[string]& file w/ SFH amplitudes that should be held fixed\\
{\tto outfile}  &[string]& contains final best-fit SFH amplitudes\\
{\tto logfile}  &[string]& file w/ current best SFH (useful for monitoring, and for\\
                && restarting an interrupted run)\\
{\tto plgfile}  &[string]& log of all tested parameter locations \\
                && (this file will be very large; not used by default: {\tto iplg}=0)\\
{\tto chifile}  &[string]& output file containing grid populations and chi-squared values\\
\hline
\multicolumn{3}{|c|}{\bf Synthetic CMD parameters} \\ 
\hline
{\tto nfree}  & [int]& number of independent isochrone amplitudes\\
{\tto ncmd}   & [int]& number of CMDs over which to calculate chi-squared\\
{\tto npix}   & [int]& binning factor for CMDs\\
{\tto dpix}   &[real]& size of synthetic CMD pixels (same as in {\ttg synth.dat})\\
\hline
\multicolumn{3}{|c|}{\bf Parameters for each CMD} \\ 
\hline
{\tto suffix(i)} &[string]& CMD filename suffix (same as in {\ttg synth.dat})\\
{\tto   xmin(i)} & [real] & calculate chi-squared within these CMD limits\\
{\tto   xmax(i)} & [real] & calculate chi-squared within these CMD limits\\
{\tto   ymin(i)} & [real] & calculate chi-squared within these CMD limits\\
{\tto   ymax(i)} & [real] & calculate chi-squared within these CMD limits\\
{\tto   nbox(i)} &  [int] & number of grid boxes within region defined by above limits\\ 
\hline
\multicolumn{3}{|c|}{\bf Runtime parameters} \\ 
\hline
{\tto seed}      &  [int]&  random seed value \\
{\tto fit\_stat} &  [int]&  0 = use chi-squared; 1 = use Lorentzian; 2 = Use Poisson\\
{\tto uselog}    & [flag]&  1 = start searching at logged position in logfile\\
{\tto iplg}      & [flag]&  1 = generate plgfile (warning: this file will be very large!)\\
{\tto gtype}     & [flag]&  grid definition flag (0 = uniform grid; 1 = custom grid)\\
{\tto iverb}     &  [int]& verbosity flag (larger N = more screen output)\\
{\tto lambda}    & [real]& initial size of simplex\\
{\tto conf}      & [real]& confidence interval for computing errorbars (e.g., 1 sigma = 0.68)\\
{\tto cthr}      & [real]& if ($\Delta\chi^2$ $>$ cthr), consider new location an improvement\\
{\tto ptol}      & [real]& if ($|p\_new - p\_old| > ptol$), consider new location different\\
{\tto ftol}      & [real]& if simplex's chi-squared values span less than ftol, signal a zero \\
                 &       & gradient (i.e., a minimum) \\
{\tto nvec}      &  [int]& number of parameter space directions in random-search loop\\
{\tto ntry}      &  [int]& number of iterations for determining errorbars\\
\hline
\end{tabular}
}

\clearpage

\noindent
To use {\ttg sfh}:

\begin{enumerate}
\item {\bf construct the {\tto cmdfile}}.  The file's columns are:

\medskip\noindent
{\tto z\_metal}, {\tto log\_age}, {\tto filenamestem}

\medskip\noindent
Column format does not matter.  Note that some lines in the
{\tto cmdfile} may represent multiple isochrones.  In these cases, 
just choose a representative age/metallicity for the group.  These 
values are only used to identify the amplitudes in the output file.  
The {\tto filenamestem} needs to point to the synthetic CMD files 
generated with {\ttg synth}.

\item {\bf Decide on a CMD gridding strategy}.  {\ttg sfh} works by 
binning the CMDs and comparing the number of data and model stars in 
each bin.  The default is to uniformly bin the CMDs in each dimension 
with a binsize of {\tto npix}$\times${\tto dpix}.  If you want to use 
a custom grid instead, write a subroutine to assign gridbox numbers to 
each CMD pixel, add a call to your subroutine in {\ttg grid.f}, and 
set {\tto gtype} to 1 in {\ttg sfh.dat}.  You must also add your 
custom-grid source code files to {\ttg sfhcode/Makefile}.  An example 
custom grid can be reviewed in the files {\ttg ubbgrid.f}, 
{\ttg bvvgrid.f}, and {\ttg viigrid.f}.  These grids were constructed 
for our LMC data, and are fine in densely populated regions of the 
CMDs and coarse in sparsely populated regions.

The boxes in each CMD's grid are identified by a single number, the
box number.  Thus the distribution of stars in the CMDs can be
expressed as a two-dimensional array whose indices identify the CMD
and the gridbox within that CMD: {\tt nstars(icmd, ibox)}. 
{\tt icmd} refers to which CMD you are considering, {\tt ibox} is the 
number of the box in that CMD.  Indicate the total number of boxes in 
each CMD with the {\tto nbox(icmd)} parameter. 

\item {\bf Construct the {\tto maskfile}}.  the file's columns are:

\medskip\noindent
{\tto icmd},  {\tto ibox},  {\tto maskflag}

\medskip\noindent
Column format does not matter.  if {\tto maskflag}=1, the stars in 
({\tt icmd}, {\tt ibox}) will be excluded from the determination of 
the fit.

\item {\bf  Preprocess the data photometry}.  You need to have a 
separate photometry file for each CMD.  The filenames need to be 
({\tto datpre} + {\tto sffx(icmd)}), where {\tto datpre} and 
{\tto sffx(icmd)} are specified in {\ttg input/sfh.dat}.  The first 
column in each CMD file is the x-dimension magnitude, and the second 
column is the y-dimension magnitude.  Other columns are ignored, but 
could be used for (ra, dec) coordinates or other useful information.   
Note that the {\tto sffx(icmd)} values need to be the same as the 
{\tto suffix(icmd)} values in {\ttg synth.dat}, and that the CMD 
limits ({\tto xmin}, {\tto xmax}, {\tto ymin}, {\tto ymax}) need to be 
the same also.

\item {\bf Adjust the sfh runtime parameters.}  The default values 
should work well, but feel free to tweak them.

\item {\bf Run {\ttg sfh}}.  Depending on the value of {\tto iverb}, 
it may provide screen output on its progress (status messages, 
current lowest chi-squared value).  Generally, larger values of 
{\tto iverb} produces more screen output.  {\tto iverb}=0 will
print no screen messages; {\tto iverb}=3 provides ``live'' updates
of the convergence process.  If the run gets interrupted for 
some reason, you can resume at the last logged position by setting 
{\tto uselog}=1 (assuming the {\tto logfile} exists).

\item {\bf Evaluate the goodness-of-fit.}  When the run is complete, 
it will print out the lowest chi-squared value found.  Divide this by 
({\tto nboxes}-{\tto nfree}) to get the reduced chi-squared, which 
should probably be less than $\sim10$ for a good fit.  

To examine the SFH solution, you can use the {\ttg sfh.sm} supermongo 
script (in the {\ttg output/} subdirectory).  All of the customizable
parameters for this script are collected at the top of the file.  See
the comments in the file for guidance in adjusting these settings.

If there is a problem with the SFH fit (e.g., high chi-squared value), 
you can use the following diagnostics (each of these programs has its 
own Chapter in this manual):

\begin{enumerate}
\item the {\ttg repop} program produces artificial stellar photometry 
based on the best-fit amplitudes, to be compared to the data CMDs. 

\item the {\ttg mkimages.sh} script in the {\ttg grid/} subdirectory, 
which converts the gridbox populations in the {\tto chifile} into 
images, and embeds them into an HTML document for easy comparison.  

\item the {\ttg testchi} program allows you to tweak the SFH 
amplitudes and parameter values.  It then calculates a new chi-squared 
value for the new model.
\end{enumerate}

\end{enumerate}


\chapter{geteep and interp}

{\ttg geteep} and {\ttg interp} are used to interpolate between 
isochrones to increase the nominal matallicity and/or age resolution.
To compile the programs, cd to the {\ttg interpcode} directory and 
type {\ttg make}.  This will create the {\ttg geteep} and 
{\ttg interp} programs in the root {\ttg SFH-1.1} directory.

{\ttg geteep} identifies empirical ``equivalent evolutionary points''
(EEPs) in each isochrone.  True EEPs are stellar evolution events 
which can be identified in the isochrones, such as ``the main sequence 
turn-off'' or ``the tip of the red giant branch''.  These are very 
useful for interpolating between isochrones.  In fact, the older 
Padua isochrones provide EEP tables for this purpose, but the new
isochrones published in 2002 did not have this information (StarFISH 
includes the older EEP tables, but we don't make use of them since 
they are missing for some isochrones).

Since we don't have EEP tables for all of the isochrones, we need a 
way to identify EEPs (or something like them), which is what 
{\ttg geteep} does.  It identifies empirical EEPs (eEEPs) as local 
extrema in color or brightness, along the isochrone track in one of 
the CMDs.  Many of these points correspond to ``true'' EEPs (e.g., the 
TRGB is a local brightness maximum).  The point is to identify points 
which represent the same evolutionary state in two similar isochrones; 
we feel that the color/brightness extrema do this well.

The {\ttg geteep} input file has a two-line header.  The first line 
contains an integer equal to the number of magnitudes representing 
each point in the isochrones.  The second line contains an integer 
identifying the magnitude which will be the vertical ``brightness'' 
axis in the target CMD (in which eEEPs will be identified).  The 
horizontal axis is automatically set to be the color index 
(mag(i-1) - mag(i)), so don't choose ``1'' for the vertical axis.
The remainder of the file lists the isochrone files to be processed,
and the list of output isochrones in which eEEPs have been identified.

Once you have run {\ttg geteep} to identify and label the eEEPs in 
each isochrone, you can run {\ttg interp} to interpolate between the 
isochrones.  The {\ttg interp} input file has three header lines at 
the top which identify the number of magnitudes, the vertical-axis 
magnitude, and the fractional distance between the bracketing 
isochrones for the target interpolated isochrones.  For example, if 
the bracketing isochrones have metallicity 0.004 and 0.008, and the 
target isochrones are to have metallicity 0.005, the fractional 
distance should be 0.25. (This assumes linear variation with 
metallicity, which is probably Ok over small changes in metallicity).

The rest of the {\ttg interp} input file is the list of isochrones to 
be interpolated.  In each row, column 1 is the first bracketing 
isochrone, column 2 is the second bracketing isochrone, and column 3 
is the target isochrone.  Again, there is no strict column format; 
each filename can be up to 40 characters long.


\chapter{testpop}\label{ch:testpop}

{\ttg testpop} constructs artificial stellar photometry based on a 
user-specified star formation history.  It is basically a different 
front-end to the {\ttg synth} code base.  Build {\ttg testpop} by 
typing {\ttg make} in the {\ttg testpop} directory.  {\ttg testpop} 
reads a (probably modified) version of the {\ttg synth} input file, 
which must be named {\ttg testpop/synth.dat}.  It also reads its 
own input parameter file, which is specified on the command line:

\medskip
\noindent 
{\ttg \% ./testpop < input\_file}

\medskip
\noindent 
The input file should contain the following 8 header lines:

\medskip
\noindent
\begin{tabular}{|lr|l|} \hline
{\tto pre}      &[string]& a prefix for the output photometry files (up to 8 characters) \\
{\tto lockflag} &[flag]& 1 = use locked amplitudes (use {\tto lockfile} in synth.dat) \\
{\tto nstars}   &[int]& an amp of 1.0 will draw {\tto nstars} masses from the isochrone \\
{\tto dmod}     &[real]& delta-distance modulus.  For adjusting the DM of the isochrones \\
{\tto fext}     &[real]& additional extinction multiplier \\
{\tto gamma}    &[real]& IMF slope ({\ttg testpop} recalculates OPs) \\
{\tto fbinary}  &[real]& Binary fraction \\ 
{\tto sfrflag}  &[flag]& 1 = amps are expressed as $M_\odot/yr$; 0 = amps are Nstars/bin \\ \hline
\end{tabular}

\medskip
\noindent
The values for {\tto gamma} and {\tto fbinary} supercede the values in 
{\ttg synth.dat}.  Immediately following these 8 parameters, the input 
file should contain N lines with the following columns:

\medskip
\noindent
{\ttg $Amp_{SFH}$, $Z_{metal}$, $log(age)$, $num_{iso}$}

\medskip
\noindent
There should be one line for each isochrone in the isofile (if 
{\tto lockflag}=0), or one line for each independent isochrone group 
(if {\tto lockflag}=1).  {\ttg testpop} determines the number of stars 
to generate for isochrone $i$ according to the formula:

$$N(i) = nstars \times Amp(i) \times frac(i) \times \Delta t(i) / num_{iso}(i)$$

\noindent where: $frac(i)$ is the fraction of the full mass range that 
is represented on the isochrone ($frac(i)$ decreases with age, as more 
stars have evolved to their non-luminous end-states); $\Delta t(i)$ is 
the duration of the age bin, in Gyr; and $num_{iso}(i)$ is the number 
of isochrones in the current isochrone group.  If {\tto sfrflag}=0, 
then $\Delta t(i)=1.0$ for all $i$.  If {\tto lockflag}=0, then
$num_{iso}(i)=1$ for all $i$.

Be aware that, in general, many more stars are generated than actually 
appear in the CMDs, because a large fraction of the generated stars 
will be too faint to detect.  

\medskip
\noindent
{\ttg testpop} outputs {\tt ncmd} output photometry files, each 
containing the following data:

\medskip
\noindent
{\ttg $mag_x$},  {\ttg $mag_y$}, {\ttg $\Delta_x$}, {\ttg $\Delta_y$}

\medskip
\noindent
These files may be used directly as input data to the {\ttg sfh} 
program.


\chapter{repop}\label{ch:repop}

\noindent
{\ttg repop} is nearly identical to {\ttg testpop}.  The only 
difference is that the input SFH amplitudes are generated from the SFH 
amplitudes output by {\ttg sfh}.  {\ttg repop} constructs an 
artificial stellar population based on the best-fit SFH solution, so 
that it can be compared to the input population.

Before using {\ttg repop}, you must first generate the input SFH 
amplitudes file using {\ttg mkinput.awk}:

\medskip
\noindent
{\ttg \% cd repop} \\
{\ttg \% gawk -f mkinput.awk -v pre=file\_prefix sfh\_output\_file}

\medskip
You may need to modify this script to make it suitable to your data.
Specifically, the number of isochrone groups and number of isochrones 
per group are hard-coded, as are the metallicity strings.

Now you can actually run {\ttg repop}, specifying the new input file 
on the command line:
\medskip
\noindent
{\ttg \% repop $<$ file\_prefix.input}

\medskip
The number of stars constructed by {\ttg repop} is probably different 
from the number of data stars.  To generate the same number of stars, 
follow these steps:

\begin{enumerate}
\item use {\ttg wc} to count the number of stars in one of repop's 
output CMD files, and the corresponding input data CMD file.
 
\item multiply {\tto nscale} (in the {\ttg repop} input file) by 
N(data)/N(repop)

\item run {\ttg repop} again
\end{enumerate}

The {\ttg repop.sm} script can be used to display the CMDs of the 
generated population.  Again, you may need to modify this script.  
This script as written displays the repopulated B-V,V CMD alongside 
the original B-V,V CMD for easy comparison. It takes two arguments, 
the prefixes of the repopulated and original photometry files.


\chapter{testchi}\label{ch:testchi}

{\ttg testchi} provides an interactive front-end to the 
{\ttg fitstat.f} subroutine, which calculates the fitting statistic of 
a given model SFH.  Compile testchi by typing {\ttg make} in the 
{\ttg testchi} directory (the executable is placed in the root 
{\ttg SFH-1.1} directory).

The {\ttg testchi} input file is a simplified version of the 
{\ttg sfh} input file, with several parameters removed.  The 
{\ttg testchi} input file contains the following parameters:
 
\medskip
\noindent
{\scriptsize
\begin{tabular}{|lr|l|} 
\hline
\multicolumn{3}{|c|}{\bf Filenames} \\ 
\hline
{\tto datpre}   &[string]& input data file prefix\\ 
{\tto cmdfile}  &[string]& Sample synthetic CMDs description file\\
{\tto maskfile} &[string]& Allows one to ignore specified regions of the CMDs\\
{\tto holdfile} &[string]& file w/ SFH amplitudes that should be held fixed\\
{\tto ampfile}  &[string]& the input model SFH amplitudes to test\\
{\tto chifile}  &[string]& output file containing grid populations and chi-squared values\\
\hline
\multicolumn{3}{|c|}{\bf Synthetic CMD parameters} \\ 
\hline
{\tto nfree}  &[int]& number of independent isochrone amplitudes\\
{\tto ncmd}   &[int]& number of CMDs over which to calculate chi-squared\\
{\tto npix}   &[int]& binning factor for CMDs\\
{\tto dpix}   &[real]& size of synthetic CMD pixels (same as in {\ttg synth.dat})\\
\hline
\multicolumn{3}{|c|}{\bf Parameters for each CMD} \\
\hline
{\tto suffix(i)} &[string]& CMD filename suffix (same as in {\ttg synth.dat})\\
{\tto   xmin(i)} &[real]& calculate chi-squared within these CMD limits\\
{\tto   xmax(i)} &[real]& calculate chi-squared within these CMD limits\\
{\tto   ymin(i)} &[real]& calculate chi-squared within these CMD limits\\
{\tto   ymax(i)} &[real]& calculate chi-squared within these CMD limits\\
{\tto   nbox(i)} &[int]& number of grid boxes within region defined by above limits\\ 
\hline
\multicolumn{3}{|c|}{\bf Runtime parameters} \\ 
\hline
{\tto fit\_stat} &[int]&  0 = use chi-squared; 1 = use Lorentzian; 2 = Use Poisson\\
{\tto gtype}     &[flag]&  grid definition flag (0 = uniform grid; 1 = custom grid)\\
{\tto iverb}     &[int]& verbosity flag (larger N = more screen output)\\
\hline
\end{tabular}
}

\medskip
To use {\ttg testchi}, adjust the parameters in {\ttg testchi.dat} (or 
create a new input file) and run the program:

\medskip\noindent
{\ttg \% ./testchi < input/testchi.dat}

\medskip
The chi-squared value resulting from a comparison of the data and the 
model described by the {\tto outfile} is printed to the screen, and 
the gridbox populations are written to the {\tto chifile}.  

{\ttg testchi} is very useful for quickly seeing changes in the 
chi-squared value when the SFH amplitudes (or other parameters) are 
changed.  You can examine the distribution of chi-squared in the CMDs 
using the {\tto chifile}, and the {\ttg mkimages.sh} script.


\chapter{mkimages.sh}\label{ch:grid}

The {\ttg grid} subdirectory contains the script {\ttg mkimages.sh},
which generates images of the CMDs from the grid box populations in 
the {\tto chifile}, and generates an HTML document which displays the 
images in a grid.  The HTML document provides a very convenient way to 
examine the quality of the model fit, and to diagnose what may be 
causing a poor fit.

The script requires files named {\ttg grid/gridN.index} which 
contains a description of the grid boxes in each of your CMDs
(each CMD has its own index file). The columns are: 

\medskip\noindent
{\ttg ix}, {\ttg iy}, {\ttg ibox}

\medskip
\noindent
where ({\ttg ix}, {\ttg iy}) are the synthetic CMD pixel coordinates, 
and {\ttg ibox} is the grid box to which that pixel belongs.  You can 
modify the program {\ttg mkgrid.f} to generate the gridN.index files 
for your CMDs.

Once you have valid {\ttg gridN.index} files, you can use the 
{\ttg mkimages.sh} script to generate the images and HTML page.  
Before running the script, there are several parameters which should 
be adjusted; see the "CUSTOMIZATION" section of the {\ttg mkimages.sh} 
file.

When running the script, you specify a filename prefix for the HTML 
document and images, and the name of the {\tto chifile} on the command 
line:

\medskip\noindent
{\ttg \% ./mkimages.sh c1978 ../output/c1978.chi}

\medskip
The script will generate 4 images per CMD, which will be placed in the 
{\ttg images/} subdirectory.  The images represent the grid box 
populations of the model and of the data; the difference between model 
and data, and the contribution of each grid box to the fitting 
statistic.  You can view all of the images using the HTML document 
that is created by the script.


\begin{thebibliography}{75}

\bibitem[Bertelli et~al. (1994)]{ber94}
{Bertelli}, G., {Bressan}, A., {Chiosi}, C., {Fagotto}, F., \& {Nasi}, E. 1994,
  \aaps, 106, 275

\bibitem[Girardi et~al. (2000)]{gir00}
{Girardi}, L., {Bressan}, A., {Bertelli}, G., \& {Chiosi}, C. 2000,
  \aaps, 141, 371
  
\end{thebibliography}
\end{document}


\chapter{Introduction}\label{ch:intro}

\section{Overview}

The {\bf StarFISH} package is a suite of FORTRAN programs designed to 
determine the best-fit star formation history (SFH) of a stellar 
population.  The package constructs a library of synthetic 
color-magnitude diagrams (CMDs) based on theoretical isochrones and 
data-derived determinations of the interstellar extinction, 
photometric errors, and distance modulus.  These synthetic CMDs are 
combined linearly and statistically compared to observed photometry.  
When the best-fitting model is found, the amplitude coefficients on 
each synthetic CMD describe the star formation history of the observed 
population.

\medskip
\noindent
If you have any comments, corrections, suggestions, or complaints,
please email: \\
{\ttb jharris@as.arizona.edu}

\bigskip
\noindent
To use the package, you need:

\begin{itemize}
\item {\bf Multicolor photometry} of a stellar population. \\

\item A (preferably empirical) {\bf photometric error model}, 
typically derived from artificial star tests. \\

\item A {\bf library of isochrones} that covers relevant ranges in age 
and metallicity. \\

\item {\it a priori} determination of the {\bf interstellar 
extinction} and {\bf distance modulus}. \\

\item Estimates (or guesses) of the {\bf binary fraction} and {\bf IMF 
slope}. \\
\end{itemize}

\clearpage

The package consists of three main programs, and five support
programs: 

\begin{quote}
\bigskip\noindent
$\bullet$ {\ttg mklib}: constructs a library of isochrone 
photometry from raw isochrones.

\medskip\noindent
$\bullet$ {\ttg synth}: constructs synthetic CMDs 
from the isochrones, applying the effects of interstellar extinction, 
binarity, IMF, and photometric errors as appropriate for the data.

\medskip\noindent
$\bullet$ {\ttg sfh}:   performs a 
maximum-likelihood comparison of composite model CMDs and input data 
CMDs.  The best-fit star formation history is output. 


\bigskip\noindent
- {\ttg interp}:  interpolates between adjacent 
isochrones, to improve resolution in age and/or metallicity.  Requires 
knowledge of equivalent evolutionary points (EEPs) along each 
isochrone. 
   
\medskip\noindent
- {\ttg testpop}: constructs artificial stellar 
populations from the isochrones.  {\ttg testpop} 
incorporates the photometric errors, extinction, etc. exactly as is 
done in {\ttg synth}.  These populations can be 
input to {\ttg sfh} for testing purposes.  
({\ttg testpop} is in the subdirectory of the same 
name)

\medskip\noindent
- {\ttg repop}: same as {\ttg testpop}, except used to construct an 
artificial population from the SFH amplitudes found by {\ttg sfh}.  
This is useful for model/data comparisons. ({\ttg repop} is in the 
subdirectory of the same name)

\medskip\noindent
- {\ttg testchi}: allows one to examine the parameter space of the 
fit by hand.  SF amplitudes and other parameters can be interactively 
adjusted, and the resulting fit recomputed on the fly. ({\ttg testchi} 
is in the subdirectory of the same name) 

\medskip\noindent
- {\ttg mkimages.sh}: creates CMD images from the best-fit model.  It 
also creates CMD maps of each region's contribution to the fitting 
statistic. 
\end{quote}

\clearpage

\section{Getting Started}

To begin, you need to obtain the program code from the Magellanic 
Clouds Photometric Survey website: \\
{\ttb http://ngala.as.arizona.edu/mcsurvey/}

\medskip
As of this writing, the current version is 1.1.  Note that there is a 
full version and a compact version.  The full version includes a 
sample pre-generated synthetic CMD library for testing purposes.  If 
you download the compact version, you can still generate the sample 
synthCMD library, but it will take some time.

\medskip 
The following command will unpack the directory tree of the StarFISH 
distribution. It will create a {\ttg SFH-1.1} subdirectory in the 
current directory.

\medskip\noindent
{\ttg \% tar zxvf SFH-1.1.tar.gz} \hfill (or {\ttg tar zxvf SFH-1.1compact.tar.gz}) \\

\noindent or (on Solaris): \\
\medskip
\noindent
{\ttg \% gunzip SFH-1.1.tar.gz} \\
{\ttg \% tar xvf SFH-1.1.tar}

\medskip
The package contents are shown in {\bf Table~1}.

\bigskip
\noindent{\large {\bf Table 1: Package Contents}}\\
{\scriptsize
\begin{tabular}{|l|l|}
\hline
{\bf File/Directory Name}  & {\bf Description} \\ 
\hline
{\ttg manual/}     & The StarFISH manual (this document)\\
{\ttg libcode/}    & source code and Makefile for the {\ttg mklib} program\\
{\ttg synthcode/}  & source code and Makefile for the {\ttg synth} program\\
{\ttg sfhcode/}    & source code and Makefile for the {\ttg sfh} program\\
{\ttg commoncode/} & source code used by multiple programs\\
{\ttg interpcode/} & source code and Makefile for the {\ttg geteep} and {\ttg interp} programs\\
{\ttg input/}      & input data files for {\ttg mklib}, {\ttg synth} and {\ttg sfh}\\
\hspace{20pt}{\ttg /lib.dat}   & sample {\ttg mklib} parameter file\\
\hspace{20pt}{\ttg /synth.dat} & sample {\ttg synth} parameter file\\
\hspace{20pt}{\ttg /sfh.dat}   & sample {\ttg sfh} parameter file\\
\hspace{20pt}{\ttg /iso.dat}   & sample isochrone description files\\
\hspace{20pt}{\ttg /iso.lock}  & describes how isochrones are to be locked together\\
\hspace{20pt}{\ttg /test.cmds} & sample synthetic CMD description file\\
\hspace{20pt}{\ttg /mask.none} & file indicating CMD regions to ignore in the fit (default: null mask)\\
\hspace{20pt}{\ttg /test.hold} & file indicating which SFH amplitudes should be held fixed (default: none held)\\
\hspace{20pt}{\ttg /av/*.av}   & sample extinction data files\\
\hspace{20pt}{\ttg /crowd/*.crowd} & sample crowding table from artificial star tests\\
\hspace{20pt}{\ttg /dtime.dat} & file describing age interval covered by each isochrone\\
\hspace{20pt}{\ttg /geteep.dat} & input file for {\ttg geteep} \\
\hspace{20pt}{\ttg /interp.dat} & input file for {\ttg interp} \\
\hspace{20pt}{\ttg /testchi.dat} & input file for {\ttg testchi} \\
{\ttg testpop/} & files related to the {\ttg testpop} support program\\
{\ttg testchicode/} & files related to the {\ttg testchi} support program\\
{\ttg repop/}   & files related to the {\ttg repop} support program\\
{\ttg grid/}    & files related to the {\ttg grid} support scripts\\
{\ttg iso/}     & raw isochrone photometry from Girardi et~al. (2002, 2000) 
                  and Bertelli et~al. (1994)\\
{\ttg testlib/}\footnotemark[1]  & sample isochrone library produced by {\ttg mklib}\\
{\ttg test.syn/}\footnotemark[1] & sample synthetic CMDs based on artificial star tests
                                   from an image of NGC~1978.\\
{\ttg data/}   & input data photometry goes here (NGC~1978 photometry included)\\
{\ttg output/} & {\ttg sfh} results and the {\ttg sfh.sm} plotting script\\ \hline
\end{tabular}
}

\footnotetext[1]{Not present in the compact version of the program}

\clearpage

\section{Compiling the Principal Code}

Each principal program ({\ttg mklib}, {\ttg synth}, and {\ttg sfh}) 
has its own Makefile, in its source code directory.  To compile these 
programs, simply cd to the correct directory and type {\ttg make}:

\medskip
\noindent
{\ttg \% cd /path/to/SFH-1.1} \hfill (if not already in the SFH directory) \\
{\ttg \% cd libcode} \\
{\ttg \% make} \\
{\ttg \% cd ../synthcode} \\
{\ttg \% make} \\
{\ttg \% cd ../sfhcode} \\
{\ttg \% make}

\medskip
Each program will be copied into the {\ttg SFH-1.1} directory, which 
is where you should run them.  You are now ready to experiment with 
the code.

\vspace{2em}
\section{A Quick Run}

We will use all of the default parameter values to go from a freshly 
downloaded and compiled distribution to a successful first run of 
{\ttg sfh}.  Steps labeled with an asterisk (*) may be skipped if you 
downloaded the full version of StarFISH, because this version already 
contains the pregenerated isochrone library and synthetic CMD library.

\begin{enumerate}

\item {\ttg \% cd /path/to/SFH-1.1} \hfill (if not already in the SFH directory)

\item (*) Prepare isochrones:  (see Section~\ref{sec:iso})\\
{\ttg \% cd iso} \\
{\ttg \% tar zxvf girardi.tar.gz} \hfill (unpack the isochrones) \\
{\ttg \% gawk -f parse\_iso.awk isocz*.dat} \hfill (parse isochrones) 

\item (*) Generate isochrone library: (see Chapter~\ref{ch:lib})\\
{\ttg \% cd ..} \\
{\ttg \% mkdir testlib} \hfill (create target directory before running {\ttg mklib}!)\\
{\ttg \% ./mklib < input/lib.dat} 

\item (*) Generate Synthetic CMD library: (see Chapter~\ref{ch:synth}) \\
{\ttg \% mkdir test.syn} \hfill (create target directory before running {\ttg synth}!)\\
{\ttg \% ./synth < input/synth.dat} 

\item Perform SFH Solution: (see Chapter~\ref{ch:sfh}) \\
{\ttg \% ./sfh < input/sfh.dat} 

\item Plot solution: \\
{\ttg \% cd output} \\
{\ttg \% sm} \hfill (start supermongo plotting program) \\
{\ttg sm$>$ macro read sfh.sm } \hfill (load sm script) \\
%{\tt sm$>$ dev x11} \hfill (bring up display window) \\
{\ttg sm$>$ sfh c1978 c1978} \hfill (display SFH solution)
\end{enumerate}

\clearpage

\section{What's New in Version 1.1}

There have been several improvements made since 1.0; the most 
important changes are listed here.  See the files {\ttg ChangeLog} and 
{\ttg NEWSTUFF} for more details.

\begin{itemize}
\item Fixed random-number generator.  We found a periodicity in 
the previous one. \\

\item More robust handling of input files.  The programs are now 
much less picky about the format of input files ({\itr note that 
input files from previous versions of StarFISH are not compatible 
with this version}). \\

\item Input files can now contain comment lines.  Any line in an 
input file that begins with {\ttg \#} or {\ttg *} will be ignored by 
the parser. \\

\item Filenames in the input files can now be up to 40 characters 
long (previous maximum was 24 characters) \\

\item Option to use Dolphin's ``Poisson'' fit statistic instead of 
the chi-squared statistic. \\

\item The photometric error model used in {\ttg synth}, 
{\ttg testpop}, and {\ttg repop} now properly includes covariance 
between the color and magnitude errors. \\

\item Option to use an analytic error model instead of an 
AST-derived crowding table in {\ttg synth}, {\ttg testpop}, and 
{\ttg repop}. \\

\item {\ttg sfh} writes the current best solution to the logfile more 
frequently. \\

\item New empirical isochrone interpolation code ({\ttg geteep} and 
{\ttg interp})

\end{itemize}


\chapter{mklib}\label{ch:lib}

{\ttg mklib} is used for processing raw isochrone photometry into a 
format usable by {\ttg synth}.  {\ttg mklib} applies a distance modulus, 
interpolates along the isochrone to a uniformly high resolution, and 
computes occupation probabilities (OP) for each isochrone point.

The package currently ships with the Padua isochrones, most recently
published by Girardi et~al. (2002).  We found that the main sequence 
in these isochrones is too coarsely sampled for our purposes, 
resulting in artificially "lumpy" synthetic CMDs. Therefore,  
{\ttg mklib} includes a photometric interpolation routine for points 
fainter than the main sequence turn-off (MSTO; defined for each 
isochrone in the {\tto isofile}).  If your isochrones do not require 
this interpolation, simply set the MSTO magnitude for each isochrone 
to be fainter than any isochrone point.  

\bigskip
The {\ttg mklib} input file (named {\ttg input/lib.dat} by default) 
contains the parameters listed in Table~2 (the lines 
must be present in the listed order). \\

\noindent{\large {\bf Table 2: mklib input file parameters}} \\
{\scriptsize
\begin{tabular}{|lr|l|}
\hline
{\tto isofile} &[string]& description of each isochrone: the input filename, \\
               &        & the output filename, the age, and the MSTO magnitude. \\
{\tto faint}   & [real] & the faint magnitude limit for the output isochrone library.\\
               &        & Should be several mag fainter than the data's faint limit. \\
{\tto dmag}    & [real] & photometric distance between adjacent interpolated points.\\
{\tto dmod}    & [real] & distance modulus \\
{\tto gamma}   & [real] & logarithmic IMF slope (Salpeter = -1.35) \\
{\tto nmag}    &  [int] & number of magnitude values per isochrone point \\
{\tto mag0}    &  [int] & which magnitude to check against {\bf faint} and  {\bf msto} \\
{\tto iverb}   &  [int] & verbosity flag (0=silent, 1=screen messages, 2=extra output files)  \\
\hline
\end{tabular}
}

\clearpage

\noindent
To use {\ttg mklib}:

\begin{enumerate}
\item {\bf Create an isofile that describes your isochrones.}  This 
file will also be used by {\ttg synth}.  Each line in the file should 
contain the following space-delimited fields (the column format does 
not matter):

\medskip\noindent
{\tto log(age)}~[real number], 
{\tto input raw isochrone file}~[up to 40 chars], 
{\tto output isochrone file}~[up to 40 chars], 
{\tto msto\_mag}~[real number]

\medskip
{\tto msto\_mag} is the absolute magnitude of each isochrone's MSTO point.  
If you do not know the MSTOs for your isochrones, you'll need to identify 
them.  Use the fact that at the MSTO, the occupation probability changes 
dramatically.
\medskip

\item {\bf Preprocess the isochrone photometry files}.  Each isochrone 
must be in its own file with a name matching the isochrone's entry in 
the {\tto input raw isochrone file} list (column 2 of the isofile).  
The columns required for the isochrone files are:

\medskip\noindent
{\tto mass}, {\tto mag1}, {\tto mag2}, {\tto mag3}, ..., {\tto magN}
 
\medskip
where N is the equal to the {\tto nmag} parameter in {\ttg lib.dat}.  
Column format does not matter.  The magnitudes should be expressed as 
absolute magnitudes; {\ttg mklib} will apply the distance modulus 
specified in {\ttg lib.dat} and output apparent magnitudes.  For the 
default Padua isochrones, the input isochrone files are generated by 
the {\ttg parse\_iso.awk} script. See the next section for details on 
the default isochrone-processing.  Make sure the {\tto mag0} parameter 
indicates the pass-band which should be used for comparing to the 
{\tto faint} and {\tto msto\_mag} values.  

\medskip
\item {\bf Create the target output directory, if necessary}. The 
output directory is specfied as part of the output filenames in the 
{\tto isofile}.

\medskip
\item {\bf Run {\ttg mklib}:} \\
{\ttg \% ./mklib < input/lib.dat}

\medskip
Depending on the verbosity flag, {\ttg mklib} may provide screen 
messages regarding which isochrone it is working on.  The output 
directory will contain the processed isochrones, with the following 
columns:

\medskip\noindent
{\tto mag1}, {\tto mag2}, {\tto mag3}, ..., {\tto magN}, {\tto OP}, 
{\tto mass}

\medskip
where the {\tto magN} are the apparent magnitudes in each band, 
{\tto OP} is the relative occupation probability associated with each 
point (determined from the IMF and from the relative duration of each 
evolutionary stage), and {\tto mass} is the {\bf initial} mass of each 
isochrone point, not including any mass-loss that may have occured 
since the star was formed.
\end{enumerate}

\clearpage

\section{Pre-processing the Padua Isochrones}\label{sec:iso}

The {\ttg iso/} directory contains files related to the Padua 
isochrones, which we ship with StarFISH for convenience.  You are free 
to use any other isochrone set that you wish, as long as it can be 
placed in the format that {\ttg mklib} requires (separate files for 
each isochrone, with the column format specified in step 2 above).

The Padua isochrones are disributed as a tar archive, which we include 
in StarFISH as the file {\ttg iso/girardi.tar.gz}.  The archive contents 
are listed in Table~3.

\bigskip
\noindent{\large {\bf Table 3: Padua isochrone package contents}} \\
\begin{tabular}{|l|l|}
\hline 
{\ttg isocz0[001,004,01,04,08,19,30].dat} &  Girardi et~al. isochrones\\
{\ttg isocsummz0[001,004,01,04,08,19,30].dat} &  Tables of Equivalent 
    Evolutionary \\
    & Points (EEPs) \\ 
{\ttg read.me} &  Girardi's readme file\\ 
\hline
\end{tabular}

\vspace{3em}
Each of the isochrone files {\ttg isocz0*.dat} contains isochrones of 
all ages for a given metallicity.  StarFISH needs a separate file for 
each isochrone, so we provide the AWK script {\ttg parse\_iso.awk} to 
divide each raw isochrone collection into individual files.  We also 
provide the script {\ttg parse\_summ.awk} to divide the EEP table 
collections into individual EEP tables for each isochrone.

If you downloaded the full version of StarFISH, then you do not need 
to run the parsing scripts; you already have the processed isochrone 
files, named {\ttg iso/zNNNN\_tt.tt} (where {\ttg NNNN} is a 4-digit 
code for the metallicity, such as {\ttg 0080}; and {\ttg tt.tt} is the 
log of the age, such as {\ttg 10.00}).  If you do not have these 
files, generate them like this:

\medskip\noindent
{\ttg \% cd iso}\\
{\ttg \% tar zxvf girardi.tar.gz} \\
{\ttg \% gawk -f parse\_iso.awk isocz*.dat} \\
{\ttg \% gawk -f parse\_summ.awk isocsummz*.dat} \\

\medskip
If you are satisfied with the age and metallicity resolution provided 
by the default Padua isochrones, then you are all done here.  If you 
want to try to interpolate between isochrones to increase age or 
metallicity resolution, refer to the chapter discussing the 
{\ttg interp} support program.


\chapter{synth}\label{ch:synth}

{\ttg synth} constructs a synthetic CMD library from the isochrone 
library, incorporating the photometric effects of extinction, 
binarity, the IMF, and photometric errors.  Each synthetic CMD is 
represented as a list of ``pixel values'', equal to the relative number 
of stars found at the corresponding subregion in the CMD.  If the data 
contain more than two filters, then multiple CMDs can be created for 
each isochrone, so that all of the data can be utilized in the fit.  
For example, if your data include $U$, $B$, and $V$ photometry, you 
might use {\ttg synth} to create two CMDs per isochrone: one with 
$U-B$ vs. $B$ axes, and another with $B-V$ vs. $V$ axes.

Some of the user-provided parameters (extinction, photometric errors, 
and binarity) are first encountered in the {\ttg synth} program. 
Others (the IMF and distance modulus) have already been used to 
create the isochrones output by {\ttg mklib}.  Therefore, if you need 
to change the IMF or distance modulus you must run {\ttg mklib} first. 

Depending on the age and metallicity resolution of your isochrone 
library, and on the quality of your photometric data, it is likely
that some of the isochrones are photometrically degenerate.  If 
{\ttg synth} is allowed to produce degenerate synthetic CMDs, then the 
SFH solution determined by {\ttg sfh} will be ambiguous, and will 
suffer from large correlated uncertainties among the SFH amplitudes.  
To avoid ambiguous SFH fits, {\ttg synth} can combine 
photometrically-degenerate isochrones together into the same synthetic 
CMD.  This is referred to as {\bf locking} the isochrones. 

\clearpage

\noindent The synth input file ({\ttg input/synth.dat}) contains the 
parameters listed in Table~4. {\itr Note that synth input files from 
previous versions of StarFISH will no longer work}.

\bigskip
\noindent{\large {\bf Table 4: synth input file parameters}} \\
{\scriptsize
\begin{tabular}{|lr|l|}
\hline
\multicolumn{3}{|c|}{\bf Filenames} \\ 
\hline
{\tto isofile}  &[string]& description of the isochrones (can use the file from {\ttg lib.dat}) \\
{\tto lockfile} &[string]& file describing the combination of degenerate isochrones\\
{\tto hotfile}  &[string]& list of extinction measurements of hot (young) stars\\
{\tto coldfile} &[string]& list of extinction measurements of cool (old) stars\\
{\tto crowd1}   &[string]& crowding table from artificial star test\\
{\tto crowd2}   &[string]& output file for lookup table of binned delta-magnitudes \\ 
\hline
\multicolumn{3}{|c|}{\bf CMDs} \\ 
\hline
{\tto nmag} &  [int] & same as nmag in {\ttg lib.dat}\\
{\tto ncmd} &  [int] & number of CMDs to construct per isochrone group\\
{\tto mag0} &  [int] & same as mag0 in {\ttg lib.dat}\\ 
{\tto dpix} & [real] & size of CMD pixels (in magnitudes) {\itr[NEW]}\\
\hline
\multicolumn{3}{|c|}{\bf parameters for each of the CMDs} \\ 
\hline
{\tto xeq}    &[string]& filter equation for the x-dimension\\
{\tto yeq}    &[string]& filter equation for the y-dimension\\
{\tto xmin}   & [real] & smallest allowable x-value\\
{\tto xmax}   & [real] & largest allowable x-value\\
{\tto ymin}   & [real] & smallest allowable y-value\\
{\tto ycmax}  & [real] & largest allowable y-value, for crowding table stars\\
{\tto ypmax}  & [real] & largest allowable y-value, for data/model stars\\
{\tto suffix} &[string]& filename suffix for the synthetic CMD output files\\ 
\hline
\multicolumn{3}{|c|}{\bf Crowding} \\ 
\hline
{\tto dbinx} & [real] & width of crowding bins, in magnitudes\\
{\tto dbiny} & [real] & height of crowding bins, in magnitudes\\
{\tto emin}  & [real] & minimum delta-magnitude cutoff\\
{\tto emax}  & [real] & maximum delta-magnitude cutoff\\
{\tto dbin}  & [real] & binsize of delta-magnitude histograms\\ 
\hline
\multicolumn{3}{|c|}{\bf Reddening parameters} \\ 
\hline
{\tto red(i)} & [real] & extinction in mag(i) (relative to the filter in which extinction is measured) \\ 
\hline
\multicolumn{3}{|c|}{\bf Run-time parameters} \\ 
\hline
{\tto verb}        &  [int] & verbosity flag (higher N = more output)\\
{\tto interp\_err} & [flag]\footnotemark[2] & 1 = interpolate errors between bracketing crowding bins {\itr[NEW]}\\
{\tto fake\_errs}  & [flag]\footnotemark[2] & 1 = use analytic error model instead of crowding table {\itr[NEW]}\\
{\tto nscale}      &  [int] & number of model stars in each CMD\\
{\tto seed}        &  [int] & random seed value\\
{\tto mass1}       & [real] & minimum mass (for OP normalization)\\
{\tto mass2}       & [real] & maximum mass (for OP normalization)\\
{\tto gamma}       & [real] & logarithmic IMF slope (Salpeter = -1.35)\\
{\tto faint}       & [real] & faint limit for generated model stars\\
{\tto fbinary}     & [real] & binary fraction\\ 
\hline
\end{tabular}
}
\footnotetext[2]{A flag is an int whose value can either be 0 or 1.}

\clearpage
\noindent To use {\ttg synth}:

\begin{enumerate}
\item {\bf Construct the {\tto lockfile}.}  This file describes how the 
isochrones are to be locked together into groups.  Again, this is 
necessary if isochrones of similar age are photometrically degenerate, 
to avoid ambiguous SFH fits.  The columns of the {\tto lockfile} are as 
follows ({\itr note that lockfiles from previous versions of StarFISH 
will no longer work}):

\medskip
\noindent
{\tto group\_id}[int], 
{\tto isoname}[up to 40 characters], 
{\tto synthfilestem} [up to 40 characters]

\medskip
{\tto group\_id} identifies the locked group to which the isochrone 
belongs.  The actual {\tto group\_id} values do not matter; when the 
value of one line is different from that of the previous line, a new 
synthetic CMD is started.

{\tto isoname} is the isochrone filename.  It is only used to 
check against the filename in the {\tto isofile}.  If they don't 
match, an error is triggered; there must be a one-to-one 
correspondence of the isochrones in the {\tto isofile} and the 
{\tto lockfile}.

{\tto synthfilestem} is the filename stem for the output synthetic 
CMD pixel file.  The output filenames are constructed by concatenating 
the CMD suffixes specified in {\ttg synth.dat} onto 
{\tto synthfilestem} (these suffixes are typically the combination of 
magnitudes; for example a $U-B$ vs. $B$ CMD might have a suffix of 
``{\ttg .ub}'').

\item {\bf Construct the extinction files ({\tto hotfile}, 
{\tto coldfile}).}  The extinction files each contain a single column: 
extinction values, $A_{\lambda}$, measured in the magnitude whose 
reddening parameter in {\ttg synth.dat} is set to 1.00 (in the 
provided {\ttg synth.dat}, red(3) is set to 1.00; this is the $V$ 
band, so the extinction files list $A_V$ values).  There are two files 
({\tto hotfile} and {\tto coldfile}) to account for the different 
extinction properties of young and old stellar populations.  If you do 
not require population-dependent extinction, simply point 
{\tto hotfile} and {\tto coldfile} to the same file.  If you do not 
require extinction modeling at all, point {\tto hotfile} and 
{\tto coldfile} to the null extinction file, {\ttg input/av/zero.av}.
   
\item {\bf Construct the crowding table ({\tto crowd1})}. This is most 
likely constructed from artificial star tests.  The crowding table 
contains:

\medskip   
{\tto ra}, {\tto dec}, {\tto mag1}, {\tto dmag1}, {\tto mag2}, {\tto dmag2}, 
..., {\tto magN}, {\tto dmagN}

\medskip\noindent
column format doesn't matter.  The crowding table should contain all 
artificial stars injected into your data images, even if they were not 
detected in your photometric pipeline.  In fact, these non-recovered 
``dropout'' stars are extremely important, because {\ttg synth} uses 
them to determine the completeness rate as a function of photometric 
position in the CMDs.  For bands in which an artificial star was 
detected, the listed {\tto mag} value is its {\bf recovered} 
magnitude, and {\tto dmag} is (mag(recovered) - mag(input)).  For 
bands in which an artificial star dropped out, {\tto mag} is the 
star's {\bf input} magnitude, and {\tto dmag} is set to 9.99 in order 
to flag it as a dropout in that band.  

If you decide not to perform artificial star tests on your images 
(although such tests are {\it highly} recommended), you can either 
generate a fake crowding table by hand, based on your estimates of the 
scatter and completeness rates in each band, or you can set the 
{\tto fake\_errs} parameter in {\ttg synth.dat} to 1, which will use a
hueristic analytic error model instead of the empirical crowding 
table.  In this case, you will need to modify the code in 
{\ttg synthcode/fakecrowd.f} to generate photometric errors and 
completeness rates appropriate for your data.  The comments in that 
file will guide you in making these changes.

\item Decide how many CMDs you would like, and what their axes will 
be.  You specify the CMD axes with the {\tto xeq(icmd)} and 
{\tto yeq(icmd)} string variables in {\ttg synth.dat}.  These strings 
indicate the filter combination for each CMD axis.  For example, you 
have $VRI$ photometry, and you are using $V$=1, $R$=2, $I$=3.  Your 
first CMD could be: $V-I$ vs. $V$, so {\tto xeq(1)} = 1-3, 
{\tto yeq(1)} = 1.  Your other ``CMD" could be a two-color diagram: 
$V-R$ vs. $V-I$, so {\tto xeq(2)} = 1-2, {\tto yeq(2)} = 1-3.  The 
strings can be either the sum or the difference of two filter indices, 
or one index by itself.  Do not put whitespace in the equations; each 
must be a single ``word''.

You will also need to specify photometric limits for each CMD 
(parameters {\tto xmin}, {\tto xmax}, {\tto ymin}, {\tto ycmax}, and 
{\tto ypmax}).  Note that instead of one {\tt ymax}, there are two 
variables: {\tto ycmax} and {\tto ypmax}.  {\tto ycmax} is the faint 
limit for the crowding bins, and {\tto ypmax} is the faint limit for 
the CMD itself.  {\tto ycmax} should be at least 2 or 3 mag fainter 
than {\tto ypmax}, because very faint stars can be blended and 
detected at much brighter magnitudes than their true magnitude.  

Also specify the size of CMD ``pixels'' in the output synthetic CMD 
files ({\tto dpix}), in magnitudes.

Finally, specify a suffix string that will be concatenated with 
{\tto synthfilestem} to construct the output file name for each 
synthetic CMD.  The suffices can be up to 8 characters long, and 
typically indicate something about the CMD axes. For example, a
$B-V$ vs. $V$ CMD might have a suffix ``{\ttg .bv}''.

\item Specify the size of the crowding bins in each direction 
({\tto dbinx} and {\tto dbiny}).  Artificial stars in the crowding 
table are binned in the CMDs so that the brightness- and 
color-dependence of the photometric errors can be accounted for.
Independent $\Delta mag$ histograms will be constructed from the 
artificial stars present in each bin.  

Specify the limits for the delta-mag histograms with {\tto emin} and 
{\tto emax}.  Any $\Delta mag$ value in the crowding table outside 
these limits is treated as a photometric dropout.

The {\tto dbin} parameter sets the width of the bins in the delta-mag 
histograms ({\itr this parameter used to determine the size of the 
synthetic CMD pixels as well, but there is now a separate parameter 
for that: dpix}).

\item Specify the reddening law to use, by setting the relative 
extinction for each band: {\tto red(imag)}.  The band for which you 
have measured extinction values in {\tto hotfile} and {\tto coldfile} 
should have {\tto red(imag)} = 1.0.  The rest of the values should be 
the extinction in that band, assuming the extinction in the reference 
band is 1.0.  For example, you have $BV$ photometry, and measured 
A$_V$ extinction values.  A$_V$ = 1.0.  If we assume that 
A$_V$/E(B-V) = 3.2 = A$_V$/(A$_B$ - A$_V$), then A$_B$ = 1.3125. 

\item Specify miscellaneous parameters.  {\tto nscale} sets the number 
of model stars to include in each CMD.  {\tto nscale} should be much 
larger than the number of stars in your data CMDs, so that the Poisson 
noise in the models is much smaller than in the data.  {\tto iseed} is 
the random seed value; it can be any integer.  {\tto mass1}, 
{\tto mass2}, and {\tto gamma} are IMF parameters used for normalizing 
the pixel values.  {\ttg synth} will only attempt to add model stars 
brighter than the {\tto faint} parameter.  This is an efficiency 
feature: without this cutoff, the code would waste a lot of time 
adding undetectable stars.  Set {\tto faint} to just fainter than 
where your completeness rate reaches 0\%.

\item Once the parameter file is complete, and the other input files 
are ready, you can finally run {\ttg synth}.  It will take quite a bit 
of time to complete, depending on computing power and the number of 
isochrones.  I found it usually took several hours to process 150 
isochrones on a 700 MHz linux box.  
\end{enumerate}

You can examine the synthetic CMDs as images, using either the IRAF 
{\tt rtext} command, or the {\ttg mkgif.bat} and {\ttg pxl2ppm.awk} 
scripts I provide in the {\ttg commoncode} directory.


\chapter{sfh}\label{ch:sfh}

{\ttg sfh} performs a ``chi-squared-like'' minimization comparison 
between data photometry and a linear combination of the synthetic 
CMDs.  Each synthetic CMD is modulated by an {\bf amplitude} that is 
proportional to the number of stars with the age and metallicity of 
the synthetic CMD's parent isochrone(s).  These amplitudes form the 
multidimensional parameter space of the minimization.

I say ``chi-squared-like'' because you may actually choose among three 
statistics to determine the quality of the fit: chi-squared, 
Lorentzian, or Poisson.  The three parameters weight outliers slightly 
differently, but they are otherwise quite similar.  Hereafter, I will 
simply refer to the fitting statistic as ``chi-squared''.

The minimization analysis employs the downhill simplex ``amoeba'' 
algorithm as described in Numerical Recipes (Press {\it et al.}; 
hereafter NR).  We have written our own implementation of this 
algorithm.  It works by sampling a small region of the parameter  
space and determining the local chi-squared gradient.  It then takes a 
small step in the down-gradient direction, and determines the local 
chi-squared gradient around this new location.  The process iterates 
until a (nearly) zero gradient is found, signalling that a minimum
has been found.  

The amoeba algorithm in NR has some safeguards against finding local 
minima, but we found that it was still prone to getting stuck.  The 
simplex by definition can only sample parameter space along directions 
parallel to the N axes of the parameter space; it is blind to off-axis 
directions.  However, as it nears the minimum, lower chi-squared 
values will tend to be found when a change in one parameter is 
complemented by a change in one or more other parameters.  In other 
words, downward gradients will tend to lie along off-axis directions.

To remedy this problem with the pure simplex method, we add a 
random-search loop, which is triggered whenever the simplex claims it 
has found a minimum.  In the random-search loop, we select a random 
direction in the multidimensional parameter space, and take a small 
step along that direction.  If the step yields a smaller chi-squared
value, we continue stepping along that direction until we reach a step 
that no longer decreases chi-squared.  This is repeated for a large 
number of random directions, and the parameter space location which 
yielded the lowest overall chi-squared value is recorded.  The simplex 
is then restarted at this new parameter space location.

This iteration between the downhill simplex algorithm and the 
random-search loop is repeated until the random-search loop cannot 
find a direction which yields a lower chi-squared value.
We then accept the last simplex location as the true minimum, and 
output the best-fit SFH amplitudes.

\bigskip
\noindent The {\ttg sfh} input file ({\ttg input/sfh.dat}) contains 
the following parameters:

\medskip
\noindent
{\scriptsize
\begin{tabular}{|lr|l|} 
\hline
\multicolumn{3}{|c|}{\bf Filenames} \\ 
\hline
{\tto datpre}   &[string]& input data file prefix\\ 
{\tto cmdfile}  &[string]& Sample synthetic CMDs description file\\
{\tto maskfile} &[string]& Allows one to ignore specified regions of the CMDs\\
{\tto holdfile} &[string]& file w/ SFH amplitudes that should be held fixed\\
{\tto outfile}  &[string]& contains final best-fit SFH amplitudes\\
{\tto logfile}  &[string]& file w/ current best SFH (useful for monitoring, and for\\
                && restarting an interrupted run)\\
{\tto plgfile}  &[string]& log of all tested parameter locations \\
                && (this file will be very large; not used by default: {\tto iplg}=0)\\
{\tto chifile}  &[string]& output file containing grid populations and chi-squared values\\
\hline
\multicolumn{3}{|c|}{\bf Synthetic CMD parameters} \\ 
\hline
{\tto nfree}  & [int]& number of independent isochrone amplitudes\\
{\tto ncmd}   & [int]& number of CMDs over which to calculate chi-squared\\
{\tto npix}   & [int]& binning factor for CMDs\\
{\tto dpix}   &[real]& size of synthetic CMD pixels (same as in {\ttg synth.dat})\\
\hline
\multicolumn{3}{|c|}{\bf Parameters for each CMD} \\ 
\hline
{\tto suffix(i)} &[string]& CMD filename suffix (same as in {\ttg synth.dat})\\
{\tto   xmin(i)} & [real] & calculate chi-squared within these CMD limits\\
{\tto   xmax(i)} & [real] & calculate chi-squared within these CMD limits\\
{\tto   ymin(i)} & [real] & calculate chi-squared within these CMD limits\\
{\tto   ymax(i)} & [real] & calculate chi-squared within these CMD limits\\
{\tto   nbox(i)} &  [int] & number of grid boxes within region defined by above limits\\ 
\hline
\multicolumn{3}{|c|}{\bf Runtime parameters} \\ 
\hline
{\tto seed}      &  [int]&  random seed value \\
{\tto fit\_stat} &  [int]&  0 = use chi-squared; 1 = use Lorentzian; 2 = Use Poisson\\
{\tto uselog}    & [flag]&  1 = start searching at logged position in logfile\\
{\tto iplg}      & [flag]&  1 = generate plgfile (warning: this file will be very large!)\\
{\tto gtype}     & [flag]&  grid definition flag (0 = uniform grid; 1 = custom grid)\\
{\tto iverb}     &  [int]& verbosity flag (larger N = more screen output)\\
{\tto lambda}    & [real]& initial size of simplex\\
{\tto conf}      & [real]& confidence interval for computing errorbars (e.g., 1 sigma = 0.68)\\
{\tto cthr}      & [real]& if ($\Delta\chi^2$ $>$ cthr), consider new location an improvement\\
{\tto ptol}      & [real]& if ($|p\_new - p\_old| > ptol$), consider new location different\\
{\tto ftol}      & [real]& if simplex's chi-squared values span less than ftol, signal a zero \\
                 &       & gradient (i.e., a minimum) \\
{\tto nvec}      &  [int]& number of parameter space directions in random-search loop\\
{\tto ntry}      &  [int]& number of iterations for determining errorbars\\
\hline
\end{tabular}
}

\clearpage

\noindent
To use {\ttg sfh}:

\begin{enumerate}
\item {\bf construct the {\tto cmdfile}}.  The file's columns are:

\medskip\noindent
{\tto z\_metal}, {\tto log\_age}, {\tto filenamestem}

\medskip\noindent
Column format does not matter.  Note that some lines in the
{\tto cmdfile} may represent multiple isochrones.  In these cases, 
just choose a representative age/metallicity for the group.  These 
values are only used to identify the amplitudes in the output file.  
The {\tto filenamestem} needs to point to the synthetic CMD files 
generated with {\ttg synth}.

\item {\bf Decide on a CMD gridding strategy}.  {\ttg sfh} works by 
binning the CMDs and comparing the number of data and model stars in 
each bin.  The default is to uniformly bin the CMDs in each dimension 
with a binsize of {\tto npix}$\times${\tto dpix}.  If you want to use 
a custom grid instead, write a subroutine to assign gridbox numbers to 
each CMD pixel, add a call to your subroutine in {\ttg grid.f}, and 
set {\tto gtype} to 1 in {\ttg sfh.dat}.  You must also add your 
custom-grid source code files to {\ttg sfhcode/Makefile}.  An example 
custom grid can be reviewed in the files {\ttg ubbgrid.f}, 
{\ttg bvvgrid.f}, and {\ttg viigrid.f}.  These grids were constructed 
for our LMC data, and are fine in densely populated regions of the 
CMDs and coarse in sparsely populated regions.

The boxes in each CMD's grid are identified by a single number, the
box number.  Thus the distribution of stars in the CMDs can be
expressed as a two-dimensional array whose indices identify the CMD
and the gridbox within that CMD: {\tt nstars(icmd, ibox)}. 
{\tt icmd} refers to which CMD you are considering, {\tt ibox} is the 
number of the box in that CMD.  Indicate the total number of boxes in 
each CMD with the {\tto nbox(icmd)} parameter. 

\item {\bf Construct the {\tto maskfile}}.  the file's columns are:

\medskip\noindent
{\tto icmd},  {\tto ibox},  {\tto maskflag}

\medskip\noindent
Column format does not matter.  if {\tto maskflag}=1, the stars in 
({\tt icmd}, {\tt ibox}) will be excluded from the determination of 
the fit.

\item {\bf  Preprocess the data photometry}.  You need to have a 
separate photometry file for each CMD.  The filenames need to be 
({\tto datpre} + {\tto sffx(icmd)}), where {\tto datpre} and 
{\tto sffx(icmd)} are specified in {\ttg input/sfh.dat}.  The first 
column in each CMD file is the x-dimension magnitude, and the second 
column is the y-dimension magnitude.  Other columns are ignored, but 
could be used for (ra, dec) coordinates or other useful information.   
Note that the {\tto sffx(icmd)} values need to be the same as the 
{\tto suffix(icmd)} values in {\ttg synth.dat}, and that the CMD 
limits ({\tto xmin}, {\tto xmax}, {\tto ymin}, {\tto ymax}) need to be 
the same also.

\item {\bf Adjust the sfh runtime parameters.}  The default values 
should work well, but feel free to tweak them.

\item {\bf Run {\ttg sfh}}.  Depending on the value of {\tto iverb}, 
it may provide screen output on its progress (status messages, 
current lowest chi-squared value).  Generally, larger values of 
{\tto iverb} produces more screen output.  {\tto iverb}=0 will
print no screen messages; {\tto iverb}=3 provides ``live'' updates
of the convergence process.  If the run gets interrupted for 
some reason, you can resume at the last logged position by setting 
{\tto uselog}=1 (assuming the {\tto logfile} exists).

\item {\bf Evaluate the goodness-of-fit.}  When the run is complete, 
it will print out the lowest chi-squared value found.  Divide this by 
({\tto nboxes}-{\tto nfree}) to get the reduced chi-squared, which 
should probably be less than $\sim10$ for a good fit.  

To examine the SFH solution, you can use the {\ttg sfh.sm} supermongo 
script (in the {\ttg output/} subdirectory).  All of the customizable
parameters for this script are collected at the top of the file.  See
the comments in the file for guidance in adjusting these settings.

If there is a problem with the SFH fit (e.g., high chi-squared value), 
you can use the following diagnostics (each of these programs has its 
own Chapter in this manual):

\begin{enumerate}
\item the {\ttg repop} program produces artificial stellar photometry 
based on the best-fit amplitudes, to be compared to the data CMDs. 

\item the {\ttg mkimages.sh} script in the {\ttg grid/} subdirectory, 
which converts the gridbox populations in the {\tto chifile} into 
images, and embeds them into an HTML document for easy comparison.  

\item the {\ttg testchi} program allows you to tweak the SFH 
amplitudes and parameter values.  It then calculates a new chi-squared 
value for the new model.
\end{enumerate}

\end{enumerate}


\chapter{geteep and interp}

{\ttg geteep} and {\ttg interp} are used to interpolate between 
isochrones to increase the nominal matallicity and/or age resolution.
To compile the programs, cd to the {\ttg interpcode} directory and 
type {\ttg make}.  This will create the {\ttg geteep} and 
{\ttg interp} programs in the root {\ttg SFH-1.1} directory.

{\ttg geteep} identifies empirical ``equivalent evolutionary points''
(EEPs) in each isochrone.  True EEPs are stellar evolution events 
which can be identified in the isochrones, such as ``the main sequence 
turn-off'' or ``the tip of the red giant branch''.  These are very 
useful for interpolating between isochrones.  In fact, the older 
Padua isochrones provide EEP tables for this purpose, but the new
isochrones published in 2002 did not have this information (StarFISH 
includes the older EEP tables, but we don't make use of them since 
they are missing for some isochrones).

Since we don't have EEP tables for all of the isochrones, we need a 
way to identify EEPs (or something like them), which is what 
{\ttg geteep} does.  It identifies empirical EEPs (eEEPs) as local 
extrema in color or brightness, along the isochrone track in one of 
the CMDs.  Many of these points correspond to ``true'' EEPs (e.g., the 
TRGB is a local brightness maximum).  The point is to identify points 
which represent the same evolutionary state in two similar isochrones; 
we feel that the color/brightness extrema do this well.

The {\ttg geteep} input file has a two-line header.  The first line 
contains an integer equal to the number of magnitudes representing 
each point in the isochrones.  The second line contains an integer 
identifying the magnitude which will be the vertical ``brightness'' 
axis in the target CMD (in which eEEPs will be identified).  The 
horizontal axis is automatically set to be the color index 
(mag(i-1) - mag(i)), so don't choose ``1'' for the vertical axis.
The remainder of the file lists the isochrone files to be processed,
and the list of output isochrones in which eEEPs have been identified.

Once you have run {\ttg geteep} to identify and label the eEEPs in 
each isochrone, you can run {\ttg interp} to interpolate between the 
isochrones.  The {\ttg interp} input file has three header lines at 
the top which identify the number of magnitudes, the vertical-axis 
magnitude, and the fractional distance between the bracketing 
isochrones for the target interpolated isochrones.  For example, if 
the bracketing isochrones have metallicity 0.004 and 0.008, and the 
target isochrones are to have metallicity 0.005, the fractional 
distance should be 0.25. (This assumes linear variation with 
metallicity, which is probably Ok over small changes in metallicity).

The rest of the {\ttg interp} input file is the list of isochrones to 
be interpolated.  In each row, column 1 is the first bracketing 
isochrone, column 2 is the second bracketing isochrone, and column 3 
is the target isochrone.  Again, there is no strict column format; 
each filename can be up to 40 characters long.


\chapter{testpop}\label{ch:testpop}

{\ttg testpop} constructs artificial stellar photometry based on a 
user-specified star formation history.  It is basically a different 
front-end to the {\ttg synth} code base.  Build {\ttg testpop} by 
typing {\ttg make} in the {\ttg testpop} directory.  {\ttg testpop} 
reads a (probably modified) version of the {\ttg synth} input file, 
which must be named {\ttg testpop/synth.dat}.  It also reads its 
own input parameter file, which is specified on the command line:

\medskip
\noindent 
{\ttg \% ./testpop < input\_file}

\medskip
\noindent 
The input file should contain the following 8 header lines:

\medskip
\noindent
\begin{tabular}{|lr|l|} \hline
{\tto pre}      &[string]& a prefix for the output photometry files (up to 8 characters) \\
{\tto lockflag} &[flag]& 1 = use locked amplitudes (use {\tto lockfile} in synth.dat) \\
{\tto nstars}   &[int]& an amp of 1.0 will draw {\tto nstars} masses from the isochrone \\
{\tto dmod}     &[real]& delta-distance modulus.  For adjusting the DM of the isochrones \\
{\tto fext}     &[real]& additional extinction multiplier \\
{\tto gamma}    &[real]& IMF slope ({\ttg testpop} recalculates OPs) \\
{\tto fbinary}  &[real]& Binary fraction \\ 
{\tto sfrflag}  &[flag]& 1 = amps are expressed as $M_\odot/yr$; 0 = amps are Nstars/bin \\ \hline
\end{tabular}

\medskip
\noindent
The values for {\tto gamma} and {\tto fbinary} supercede the values in 
{\ttg synth.dat}.  Immediately following these 8 parameters, the input 
file should contain N lines with the following columns:

\medskip
\noindent
{\ttg $Amp_{SFH}$, $Z_{metal}$, $log(age)$, $num_{iso}$}

\medskip
\noindent
There should be one line for each isochrone in the isofile (if 
{\tto lockflag}=0), or one line for each independent isochrone group 
(if {\tto lockflag}=1).  {\ttg testpop} determines the number of stars 
to generate for isochrone $i$ according to the formula:

$$N(i) = nstars \times Amp(i) \times frac(i) \times \Delta t(i) / num_{iso}(i)$$

\noindent where: $frac(i)$ is the fraction of the full mass range that 
is represented on the isochrone ($frac(i)$ decreases with age, as more 
stars have evolved to their non-luminous end-states); $\Delta t(i)$ is 
the duration of the age bin, in Gyr; and $num_{iso}(i)$ is the number 
of isochrones in the current isochrone group.  If {\tto sfrflag}=0, 
then $\Delta t(i)=1.0$ for all $i$.  If {\tto lockflag}=0, then
$num_{iso}(i)=1$ for all $i$.

Be aware that, in general, many more stars are generated than actually 
appear in the CMDs, because a large fraction of the generated stars 
will be too faint to detect.  

\medskip
\noindent
{\ttg testpop} outputs {\tt ncmd} output photometry files, each 
containing the following data:

\medskip
\noindent
{\ttg $mag_x$},  {\ttg $mag_y$}, {\ttg $\Delta_x$}, {\ttg $\Delta_y$}

\medskip
\noindent
These files may be used directly as input data to the {\ttg sfh} 
program.


\chapter{repop}\label{ch:repop}

\noindent
{\ttg repop} is nearly identical to {\ttg testpop}.  The only 
difference is that the input SFH amplitudes are generated from the SFH 
amplitudes output by {\ttg sfh}.  {\ttg repop} constructs an 
artificial stellar population based on the best-fit SFH solution, so 
that it can be compared to the input population.

Before using {\ttg repop}, you must first generate the input SFH 
amplitudes file using {\ttg mkinput.awk}:

\medskip
\noindent
{\ttg \% cd repop} \\
{\ttg \% gawk -f mkinput.awk -v pre=file\_prefix sfh\_output\_file}

\medskip
You may need to modify this script to make it suitable to your data.
Specifically, the number of isochrone groups and number of isochrones 
per group are hard-coded, as are the metallicity strings.

Now you can actually run {\ttg repop}, specifying the new input file 
on the command line:
\medskip
\noindent
{\ttg \% repop $<$ file\_prefix.input}

\medskip
The number of stars constructed by {\ttg repop} is probably different 
from the number of data stars.  To generate the same number of stars, 
follow these steps:

\begin{enumerate}
\item use {\ttg wc} to count the number of stars in one of repop's 
output CMD files, and the corresponding input data CMD file.
 
\item multiply {\tto nscale} (in the {\ttg repop} input file) by 
N(data)/N(repop)

\item run {\ttg repop} again
\end{enumerate}

The {\ttg repop.sm} script can be used to display the CMDs of the 
generated population.  Again, you may need to modify this script.  
This script as written displays the repopulated B-V,V CMD alongside 
the original B-V,V CMD for easy comparison. It takes two arguments, 
the prefixes of the repopulated and original photometry files.


\chapter{testchi}\label{ch:testchi}

{\ttg testchi} provides an interactive front-end to the 
{\ttg fitstat.f} subroutine, which calculates the fitting statistic of 
a given model SFH.  Compile testchi by typing {\ttg make} in the 
{\ttg testchi} directory (the executable is placed in the root 
{\ttg SFH-1.1} directory).

The {\ttg testchi} input file is a simplified version of the 
{\ttg sfh} input file, with several parameters removed.  The 
{\ttg testchi} input file contains the following parameters:
 
\medskip
\noindent
{\scriptsize
\begin{tabular}{|lr|l|} 
\hline
\multicolumn{3}{|c|}{\bf Filenames} \\ 
\hline
{\tto datpre}   &[string]& input data file prefix\\ 
{\tto cmdfile}  &[string]& Sample synthetic CMDs description file\\
{\tto maskfile} &[string]& Allows one to ignore specified regions of the CMDs\\
{\tto holdfile} &[string]& file w/ SFH amplitudes that should be held fixed\\
{\tto ampfile}  &[string]& the input model SFH amplitudes to test\\
{\tto chifile}  &[string]& output file containing grid populations and chi-squared values\\
\hline
\multicolumn{3}{|c|}{\bf Synthetic CMD parameters} \\ 
\hline
{\tto nfree}  &[int]& number of independent isochrone amplitudes\\
{\tto ncmd}   &[int]& number of CMDs over which to calculate chi-squared\\
{\tto npix}   &[int]& binning factor for CMDs\\
{\tto dpix}   &[real]& size of synthetic CMD pixels (same as in {\ttg synth.dat})\\
\hline
\multicolumn{3}{|c|}{\bf Parameters for each CMD} \\
\hline
{\tto suffix(i)} &[string]& CMD filename suffix (same as in {\ttg synth.dat})\\
{\tto   xmin(i)} &[real]& calculate chi-squared within these CMD limits\\
{\tto   xmax(i)} &[real]& calculate chi-squared within these CMD limits\\
{\tto   ymin(i)} &[real]& calculate chi-squared within these CMD limits\\
{\tto   ymax(i)} &[real]& calculate chi-squared within these CMD limits\\
{\tto   nbox(i)} &[int]& number of grid boxes within region defined by above limits\\ 
\hline
\multicolumn{3}{|c|}{\bf Runtime parameters} \\ 
\hline
{\tto fit\_stat} &[int]&  0 = use chi-squared; 1 = use Lorentzian; 2 = Use Poisson\\
{\tto gtype}     &[flag]&  grid definition flag (0 = uniform grid; 1 = custom grid)\\
{\tto iverb}     &[int]& verbosity flag (larger N = more screen output)\\
\hline
\end{tabular}
}

\medskip
To use {\ttg testchi}, adjust the parameters in {\ttg testchi.dat} (or 
create a new input file) and run the program:

\medskip\noindent
{\ttg \% ./testchi < input/testchi.dat}

\medskip
The chi-squared value resulting from a comparison of the data and the 
model described by the {\tto outfile} is printed to the screen, and 
the gridbox populations are written to the {\tto chifile}.  

{\ttg testchi} is very useful for quickly seeing changes in the 
chi-squared value when the SFH amplitudes (or other parameters) are 
changed.  You can examine the distribution of chi-squared in the CMDs 
using the {\tto chifile}, and the {\ttg mkimages.sh} script.


\chapter{mkimages.sh}\label{ch:grid}

The {\ttg grid} subdirectory contains the script {\ttg mkimages.sh},
which generates images of the CMDs from the grid box populations in 
the {\tto chifile}, and generates an HTML document which displays the 
images in a grid.  The HTML document provides a very convenient way to 
examine the quality of the model fit, and to diagnose what may be 
causing a poor fit.

The script requires files named {\ttg grid/gridN.index} which 
contains a description of the grid boxes in each of your CMDs
(each CMD has its own index file). The columns are: 

\medskip\noindent
{\ttg ix}, {\ttg iy}, {\ttg ibox}

\medskip
\noindent
where ({\ttg ix}, {\ttg iy}) are the synthetic CMD pixel coordinates, 
and {\ttg ibox} is the grid box to which that pixel belongs.  You can 
modify the program {\ttg mkgrid.f} to generate the gridN.index files 
for your CMDs.

Once you have valid {\ttg gridN.index} files, you can use the 
{\ttg mkimages.sh} script to generate the images and HTML page.  
Before running the script, there are several parameters which should 
be adjusted; see the "CUSTOMIZATION" section of the {\ttg mkimages.sh} 
file.

When running the script, you specify a filename prefix for the HTML 
document and images, and the name of the {\tto chifile} on the command 
line:

\medskip\noindent
{\ttg \% ./mkimages.sh c1978 ../output/c1978.chi}

\medskip
The script will generate 4 images per CMD, which will be placed in the 
{\ttg images/} subdirectory.  The images represent the grid box 
populations of the model and of the data; the difference between model 
and data, and the contribution of each grid box to the fitting 
statistic.  You can view all of the images using the HTML document 
that is created by the script.


\begin{thebibliography}{75}

\bibitem[Bertelli et~al. (1994)]{ber94}
{Bertelli}, G., {Bressan}, A., {Chiosi}, C., {Fagotto}, F., \& {Nasi}, E. 1994,
  \aaps, 106, 275

\bibitem[Girardi et~al. (2000)]{gir00}
{Girardi}, L., {Bressan}, A., {Bertelli}, G., \& {Chiosi}, C. 2000,
  \aaps, 141, 371
  
\end{thebibliography}
\end{document}
